\documentclass[12pt]{article}

\usepackage{fullpage}
% Allow TeX to stretch lines a bit to avoid overfull hboxes in long inline math/text
\emergencystretch=3em
% Enable UTF-8 input so that Unicode symbols (e.g., QED symbol) in imported proofs compile
\usepackage[utf8]{inputenc}
\usepackage{amssymb}
\usepackage{amsmath}
% Silence accidental uses of \tag in non-numbered displays
\makeatletter
\renewcommand{\tag}[1]{}
\makeatother
\usepackage{amsthm}
\usepackage{newunicodechar}
\usepackage[usenames]{color}
\usepackage[colorlinks=true,linkcolor=webgreen,filecolor=webbrown,citecolor=webgreen]{hyperref}

\definecolor{webgreen}{rgb}{0,.5,0}
\definecolor{webbrown}{rgb}{.6,0,0}

% JIS macro for OEIS entries; provide a safe fallback for local builds
\providecommand{\seqnum}[1]{#1}

 \newtheorem{theorem}{Theorem}
% Sub-lemmas tied to the current theorem: Lemma 3a, 3b, ...
\newtheorem{lemma}{Lemma}[theorem]
\renewcommand{\thelemma}{\thetheorem\alph{lemma}}
\newtheorem{proposition}[theorem]{Proposition}
\newtheorem{corollary}[theorem]{Corollary}
\theoremstyle{remark}
\newtheorem*{remark}{Remark}

\begin{document}

% Be generous with line-breaking to avoid overfull boxes in inline math/text
\sloppy

\title{Conjectures About Cyclic Numbers: Resolutions and Counterexamples}
\author{Hieu Le Duc \\
Telecom SudParis \\
\texttt{duc-hieu.le@telecom-sudparis.eu}}
\date{}
\maketitle

\begin{abstract}
We settle 22 conjectures of Cohen about cyclic numbers (positive integers $n$ with $\gcd(n,\varphi(n))=1$), proving 16 and disproving 6, and we completely resolve a related OEIS problem about sequences whose running averages are Fibonacci numbers. Highlights include: asymptotics for cyclics between consecutive squares with a second-order term (Conj.~9), Legendre- and $k$-fold Oppermann-type results in short quadratic intervals (Conj.~6, Conj.~20, and twin cyclics between cubes, Conj.~32), gap and growth analogs (Visser, Rosser, Ishikawa, and a sum-3-versus-sum-2 inequality; Conj.~47,~52,~54,~56), limiting ratios (Vrba and Hassani; Conj.~60,~61), and structure results for Sophie Germain cyclics (Conj.~36,~37). We also resolve two Firoozbakht-type conjectures for cyclics (Conj.~41--42). On the negative side we exhibit counterexamples to the Panaitopol, Dusart, and Carneiro analogs (Conj.~59,~53,~50--51). Finally, for the lexicographically least sequence of pairwise distinct positive integers whose running averages are Fibonacci numbers (\seqnum{A248982}), we give explicit closed forms for all $n$ and prove Fried's Conjecture~2 asserting the disjointness of the parity-defined value sets (equivalently, $F_{n+2}+2nF_{n+1}$ is never a Fibonacci number).
\end{abstract}

\section{Introduction}
An integer $n\ge1$ is \emph{cyclic} if every group of order $n$ is cyclic. By Szele \cite{Szele1947}, this is equivalent to $\gcd(n,\varphi(n))=1$, where $\varphi$ is Euler's totient function. Let $C$ denote the set of cyclic numbers, let $C(x):=\#\{c\in C: c\le x\}$ be its counting function, and let $(c_n)_{n\ge1}$ be the increasing enumeration of $C$.

\begin{remark}
Two immediate consequences of Szele's criterion will be used repeatedly: (i) $2$ is the only even cyclic integer (if $n$ is even and $n>2$, then $2\mid n$ and $2\mid \varphi(n)$, so $\gcd(n,\varphi(n))\ge2$); (ii) $1$ is the only square in $C$ (if $n$ is a square $>1$, then $p\mid n$ and $p\mid \varphi(n)$ for some prime $p$, so again $\gcd(n,\varphi(n))>1$). All odd primes are cyclic since $\gcd(p,\varphi(p))=\gcd(p,p-1)=1$.
\end{remark}

Many conjectures about cyclics proposed by Cohen \cite{Cohen2025} are analogs of well-known results or conjectures for the prime numbers (sequence \seqnum{A000040}), with $C$ playing the role of $\mathbb P$. In this paper we give complete proofs or counterexamples to several of these conjectures, and we give a full resolution of an OEIS problem about running averages equaling Fibonacci numbers (sequence \seqnum{A248982}). Throughout, we emphasize exactly which conjectures are settled and how.

\subsection*{Notation}
We write $\mathcal P$ for the set of primes. For $z\ge2$ let
\[
 P(z):=\prod_{p\le z}p,\qquad P^-(n):=\min\{p\in\mathcal P: p\mid n\},\quad P^-(1):=\infty.
\]
For primes $p\le q$ we use the arrow notation $p\to q$ to mean $q\equiv1\pmod p$. We also use $\log_1 x:=\log x$, $\log_2 x:=\log\log x$, and $\log_3 x:=\log\log\log x$ when needed.

\subsection*{Uniform tools and ranges}
We record the uniform forms used repeatedly in the proofs; precise references are indicated.
\begin{itemize}
  \item Linear Selberg sieve (dimension 1; consecutive integers). For $x\ge2$, $H\ge1$, $z\ge2$, letting $S((x,x+H);z):=\#\{m\in(x,x+H)\cap\mathbb N: P^-(m)\ge z\}$, one has uniformly
  \[
    S((x,x+H);z)\ge H\prod_{p<z}\Bigl(1-\frac1p\Bigr)\ -\ C\,z^2,
  \]
  for an absolute constant $C>0$; see, e.g., Halberstam--Richert \cite[Th.~2.3, Th.~2.4]{HalRich1974} or Iwaniec--Kowalski \cite[\S11.1]{IK2004}. In particular, with $z=x^{\delta}(\log x)^{-B}$ and $H\asymp x^{1/2}$ the remainder is $o(\sqrt x/\log x)$.
  \item $\beta$-sieve, dimension 2 (fundamental lemma). For two linear forms and level $D\le z^u$ ($u\ge2$), the fundamental lemma gives a main term $\asymp H\prod_{p\le z}(1-\nu(p)/p)$ with remainder $\ll D\log D$, uniformly for intervals of length $H$; see Greaves \cite[Th.~5.7]{Greaves2001} or Iwaniec--Kowalski \cite[Th.~11.13]{IK2004}.
  \item Brun--Titchmarsh in AP (weighted and counting forms). Uniformly for $p\ge2$ and $Y\ge2$,
  \[
    \sum_{\substack{q\le Y\\ q\equiv1\ (\bmod p)}}\frac1q\ \ll\ \frac{\log\log Y}{\varphi(p)}\ \ll\ \frac{\log\log Y}{p},
  \]
  by partial summation from the Brun--Titchmarsh inequality (see Montgomery--Vaughan \cite[Th.~6.6]{MV2007} or Iwaniec--Kowalski \cite[Th.~18.11]{IK2004}). Moreover, for any finite union $U$ of disjoint intervals and $z\ge2$,
  \[
    \sum_{\substack{q\in\mathcal P\cap U\\ q\equiv1\ (\bmod p)}}\log q\ \le\ \frac{2\,\operatorname{mes}(U)}{\varphi(p)},\qquad
    \#\{q\in\mathcal P\cap U: q\equiv1\ (\bmod p)\}\ \le\ \frac{2\,\operatorname{mes}(U)}{\varphi(p)\,\log z},
  \]
  uniformly for $U\subset[z,\infty)$; this is the weighted Montgomery--Vaughan form \cite[Th.~6.7]{MV2007}, applied piecewise and summed over the union. We only use the case $p\le X$ and $z\asymp (\log X)^A$ for fixed $A>0$.
\end{itemize}

\medskip
\noindent\textbf{Summary of resolved conjectures.}
We settle 22 conjectures from \cite{Cohen2025} (numbers as in that paper). Below we group the main outcomes and point to the relevant statements.
\begin{itemize}
  \item Distribution in quadratic ranges: Legendre analog (Conj.~6; Theorem~\ref{thm:legendre_cyclics}); cyclics between consecutive squares with a second-order term (Conj.~9; Theorem~\ref{thm:squares}); near-square values (Conj.~14; Theorem~\ref{thm:near_square_cyclics}); $k$-fold Oppermann for cyclics (Conj.~20; Theorem~\ref{thm:k_fold_oppermann_for_cyclics}) and for primes (Conj.~17; Theorem~\ref{thm:k_fold_oppermann_for_primes}); twin cyclics between consecutive cubes (Conj.~32; Theorem~\ref{thm:twin_cyclics_between_consecutive_cubes}).
  \item Sophie Germain cyclics: infinitely many SG cyclics (Conj.~36; Theorem~\ref{thm:infinitely_many_sg_cyclics}); equidistribution mod $3$ (Conj.~37; Theorem~\ref{thm:sg_cyclics_modulo_3}).
  \item Gaps, growth, and inequalities: Rosser lower bound (Conj.~52; Theorem~\ref{thm:rosser}); Ishikawa inequality (Conj.~54; Theorem~\ref{thm:ishikawa}); Visser-type gap decay (Conj.~47; Theorem~\ref{thm:visser_cyclics}); sum-3-versus-sum-2 (Conj.~56; Theorem~\ref{thm:sum_3_versus_sum_2_cyclics}); twin cyclics (Conj.~3; Theorem~\ref{thm:twin_cyclics}).
  \item Limits: Vrba (Conj.~60; Theorem~\ref{thm:vrba}) and Hassani (Conj.~61; Theorem~\ref{thm:hassani}).
  \item Firoozbakht-type behavior: two proven forms (Conj.~41; Theorem~\ref{thm:firoozbakht_cyclics_3} and Conj.~42; Theorem~\ref{thm:firoozbakht_cyclics_4}).
  \item Counterexamples: Panaitopol (Conj.~59; Theorem~\ref{thm:panaitopol}), Dusart (Conj.~53; Theorem~\ref{thm:dusart_cyclics}), Carneiro analogs for cyclics and SG cyclics (Conj.~50,~51; Theorems~\ref{thm:carneiro_cyclics},~\ref{thm:carneiro_sg_cyclics}), and a disproof of an asserted asymptotic for $k$-fold paired cyclics between cubes (Conj.~35; Theorem~\ref{thm:asymptotic_k_fold_cyclics_between_cubes}).
  \item Fibonacci averages (\seqnum{A248982}): complete closed forms (Theorem~\ref{thm:fib-full}) and disjointness of the parity-defined value sets (Fried's Conj.~2; Proposition~\ref{prop:disjointness}).
\end{itemize}

In connection with the Fibonacci averages problem (\seqnum{A248982}), it is natural to split the values produced by the greedy construction according to parity. Writing
\[
 S_{\mathrm{even}}:=\bigl\{\,nF_{\,\frac{n}{2}+3}-(n-1)F_{\,\frac{n}{2}+2}:\ n\ \text{even}\,\bigr\},\qquad
 S_{\mathrm{odd}}:=\bigl\{\,F_{\,\frac{n+1}{2}+2}:\ n\ \text{odd}\,\bigr\},
\]
Fried conjectured that these two sets are disjoint. Equivalently, defining $T(n):=F_{n+2}+2nF_{n+1}$, the claim is that $T(n)$ is never a Fibonacci number. We prove this disjointness in Proposition~\ref{prop:disjointness}. This removes the final obstacle identified in \cite{Fried2025} and dovetails with our complete closed forms for the sequence terms.

We use the Euler--Mascheroni constant $\gamma$, and write $\log_k x$ for the $k$-fold iterated natural logarithm (so $\log_1 x = \log x$, $\log_2 x = \log\log x$, and $\log_3 x = \log\log\log x$ for $x>e^e$). A key analytic input is the asymptotic for $C(x)$ due to Erd\H{o}s and sharpened by Pollack \cite{Pollack2022}:
\begin{equation}\label{eq:Pollack}
 C(x) \sim \frac{e^{-\gamma} x}{\log_3 x} \quad (x\to\infty),\quad C(x) = e^{-\gamma} x\left(\frac{1}{\log_3 x} - \frac{\gamma}{\log_3^2 x} + O\!\left(\frac{1}{\log_3^3 x}\right)\right).
\end{equation}
We also use standard facts from regular variation \cite{BGT1989} and asymptotic integration \cite{deBruijn1970} as indicated.

\subsection*{Motivation and related work}
Many of the statements we settle are cyclic-number analogs of classical results and conjectures for the primes. The cyclic set $\mathcal C$ behaves "prime-like" because Szele's criterion reduces $\gcd(n,\varphi(n))=1$ to local congruence obstructions among the prime factors of $n$. This leads naturally to sieve methods (Selberg, $\beta$-sieve) to enforce roughness and squarefreeness, and to use Brun--Titchmarsh in arithmetic progressions to prune the internal divisibility events $p\mid(q-1)$. On the analytic side, Pollack's refinement of Erd\H{o}s's asymptotic for $C(x)$ interacts cleanly with regular-variation tools (de Haan's $\Pi$-variation) to obtain uniform local increment asymptotics needed for short quadratic intervals. We point to specific theorem-level references in the Uniform tools above to aid verification.

\subsection*{Result map}
For quick reference, Table~\ref{tab:result-map} maps Cohen's conjecture numbers to our results.
\begin{center}
\begin{tabular}{|c|c|c|}
\hline
Conj. & Result & Status \\
\hline
3 & Thm.~\ref{thm:twin_cyclics} & proved \\
6 & Thm.~\ref{thm:legendre_cyclics} & proved \\
9 & Thm.~\ref{thm:squares} & proved \\
14 & Thm.~\ref{thm:near_square_cyclics} & proved \\
17 & Thm.~\ref{thm:k_fold_oppermann_for_primes} & proved \\
20 & Thm.~\ref{thm:k_fold_oppermann_for_cyclics} & proved \\
32 & Thm.~\ref{thm:twin_cyclics_between_consecutive_cubes} & proved \\
36 & Thm.~\ref{thm:infinitely_many_sg_cyclics} & proved \\
37 & Thm.~\ref{thm:sg_cyclics_modulo_3} & proved \\
41 & Thm.~\ref{thm:firoozbakht_cyclics_3} & proved \\
42 & Thm.~\ref{thm:firoozbakht_cyclics_4} & proved \\
47 & Thm.~\ref{thm:visser_cyclics} & proved \\
52 & Thm.~\ref{thm:rosser} & proved \\
54 & Thm.~\ref{thm:ishikawa} & proved \\
56 & Thm.~\ref{thm:sum_3_versus_sum_2_cyclics} & proved \\
60 & Thm.~\ref{thm:vrba} & proved \\
61 & Thm.~\ref{thm:hassani} & proved \\
35 & Thm.~\ref{thm:asymptotic_k_fold_cyclics_between_cubes} & disproved \\
50 & Thm.~\ref{thm:carneiro_cyclics} & disproved \\
51 & Thm.~\ref{thm:carneiro_sg_cyclics} & disproved \\
53 & Thm.~\ref{thm:dusart_cyclics} & disproved \\
59 & Thm.~\ref{thm:panaitopol} & disproved \\
\hline
\end{tabular}
\end{center}
\label{tab:result-map}


\section{Cohen's Conjectures}

\subsection{Proofs}

\subsubsection{Conjecture 3 (Twin cyclics)}
\begin{theorem}[Twin cyclics analog (resolves Conj.~3 of \cite{Cohen2025})]\label{thm:twin_cyclics}
There exist infinitely many cyclic integers \(c\in\mathcal{C}\) such that \(c+2\in\mathcal{C}\).
\end{theorem}

\begin{proof}
An integer $n$ is cyclic if and only if $\gcd(n,\varphi(n))=1$ (Szele's criterion; see \cite{Szele1947}). Equivalently (and necessarily), $n$ is squarefree and for any distinct primes $p,q\mid n$ one has $p\nmid(q-1)$. Indeed, if $n$ is squarefree then $\varphi(n)=\prod_{q\mid n}(q-1)$, so $\gcd(n,\varphi(n))=1$ holds precisely when no prime divisor $p$ of $n$ divides any $q-1$ with $q\mid n$ and $q\ne p$.

Fix a large $x$ and set $z=x^{\delta}$ with a small fixed $\delta\in(0,1/10]$. Let $\mathcal P(z)=\prod_{p\le z}p$, and consider the sifted set
$$
\mathcal S_0(x,z):=\{1\le n\le x:\ (n(n+2),\mathcal P(z))=1\}.
$$
For each prime $p$ define the forbidden residue classes
$$
\Omega_p:=\{a\pmod p:\ p\mid a\text{ or }p\mid a+2\},
$$
so $|\Omega_2|=1$ and $|\Omega_p|=2$ for $p\ge3$. For a squarefree $d$, let $\rho(d)$ be the number of residue classes $a\pmod d$ such that $a\bmod p\in\Omega_p$ for all $p\mid d$. By the Chinese Remainder Theorem, $\rho$ is multiplicative and $\rho(p)=|\Omega_p|$. Moreover, for squarefree $d$ we have
$$
\#\{n\le x: d\mid n(n+2)\}=\frac{\rho(d)}{d}\,x+O(2^{\omega(d)}).
$$
Since $\sum_{p\le y}\rho(p)\frac{\log p}{p}=2\log y+O(1)$, the sieve has dimension $\kappa=2$.

By the fundamental lemma of the combinatorial (Brun-$\beta$) sieve in dimension $\kappa=2$ \cite{Greaves2001,IK2004}, for $z\le x^{1/10}$ there exists an absolute constant $c_0>0$ such that
$$
\#\mathcal S_0(x,z)\ \ge\ c_0\,x\prod_{p\le z}\Bigl(1-\frac{\rho(p)}{p}\Bigr)
\ =\ c_0\,x\Bigl(1-\tfrac12\Bigr)\prod_{3\le p\le z}\Bigl(1-\frac{2}{p}\Bigr).
$$
Using Mertens' formulas \cite{Apostol1976} and $(1-2/p)=(1-1/p)^2\bigl(1+O(1/p^2)\bigr)$, we obtain
$$
\prod_{3\le p\le z}\Bigl(1-\frac{2}{p}\Bigr)\asymp \frac{1}{(\log z)^2},
$$
so
$$
\#\mathcal S_0(x,z)\ \gg\ \frac{x}{(\log z)^2}.
$$

Next, remove $n\le x$ for which $p^2\mid n$ or $p^2\mid n+2$ for some $p>z$. The number of such $n$ is
$$
\ll \sum_{p>z}\Bigl(\Big\lfloor\frac{x}{p^2}\Big\rfloor+\Big\lfloor\frac{x}{p^2}\Big\rfloor\Bigr)\ \ll\ \frac{x}{z}.
$$
Call the removed set $\mathcal E(x,z)$.

We must also forbid, among prime divisors of $n$ (and separately of $n+2$), any pair $p\ne q$ with $q\equiv1\pmod p$. Let $\mathcal B_1(x,z)$ count $n\le x$ for which there exist primes $p,q\ge z$ with $pq\mid n$ and $q\equiv1\ (\bmod p)$. A union bound gives
$$
\#\mathcal B_1(x,z)\ \le\ \sum_{p\ge z}\ \sum_{\substack{q\ge z\ q\equiv1\ (\bmod p)}}\Big\lfloor\frac{x}{pq}\Big\rfloor
\ \ll\ x\sum_{p\ge z}\frac{1}{p}\sum_{\substack{q\le x/p\ q\equiv1\ (\bmod p)}}\frac{1}{q}.
$$
By the Brun--Titchmarsh inequality and partial summation (e.g., \cite{MV2007,IK2004}), uniformly for $p<x$,
$$
\sum_{\substack{q\le y\ q\equiv1\ (\bmod p)}}\frac{1}{q}\ \ll\ \frac{\log\log y}{\varphi(p)}\ =\ \frac{\log\log y}{p-1}.
$$
Therefore
$$
\#\mathcal B_1(x,z)\ \ll\ x\sum_{p\ge z}\frac{1}{p}\cdot\frac{\log\log x}{p-1}
\ \ll\ x\,\frac{\log\log x}{z}.
$$
An identical argument for prime divisors of $n+2$ shows
$$
\#\mathcal B_2(x,z)\ \ll\ x\,\frac{\log\log x}{z}.
$$

Define the good set
$$
\mathcal G(x,z):=\mathcal S_0(x,z)\setminus\bigl(\mathcal E(x,z)\cup\mathcal B_1(x,z)\cup\mathcal B_2(x,z)\bigr).
$$
Combining the bounds above yields
$$
\#\mathcal G(x,z)\ \gg\ \frac{x}{(\log z)^2}\ -\ O\!\left(\frac{x}{z}\right)\ -\ O\!\left(x\,\frac{\log\log x}{z}\right).
$$
With $z=x^{\delta}$ and fixed $\delta\in(0,1/10]$, we have $(\log z)^2\asymp(\log x)^2$ while $x/z$ and $x(\log\log x)/z$ are $o\bigl(x/(\log z)^2\bigr)$. Hence, for all sufficiently large $x$,
$$
\#\mathcal G(x,z)\ \gg\ \frac{x}{(\log x)^2}.
$$

For any $n\in\mathcal G(x,z)$ we have:
- $n$ and $n+2$ are coprime to all primes $\le z$ and thus all their prime factors exceed $z$;
- neither $n$ nor $n+2$ is divisible by $p^2$ for any $p>z$; hence $\mu^2(n)=\mu^2(n+2)=1$;
- by construction of $\mathcal B_1,\mathcal B_2$, among the prime divisors of $n$ (respectively $n+2$) there is no pair $p\ne q$ with $q\equiv1\pmod p$.
By the characterization at the start, this implies $n\in\mathcal C$ and $n+2\in\mathcal C$.

Since $\#\mathcal G(x,z)\to\infty$ with $x$, there are infinitely many such $n$. 
\end{proof}


\subsubsection{Conjecture 6 (Legendre analog)}
\begin{theorem}[Legendre analog for cyclics (resolves Conj.~6 of \cite{Cohen2025})]\label{thm:legendre_cyclics}
For every \(n\in\mathbb{N}\), there exists \(c\in\mathcal{C}\) with \(n^2<c<(n+1)^2\).
\end{theorem}
\begin{proof}
Fix $n\in\mathbb N$ and set $x:=n^2$, $H:=2n+1$, $X:=x+H\asymp x$. We work inside the interval $(x,X)=(x,x+H)$. For an interval $I\subset\mathbb R$ and $z\ge2$, let
\[
S(I;z):=\#\{m\in I\cap\mathbb N: P^-(m)\ge z\},
\]
with $P^-(1)=\infty$ and $P^-(m)$ the least prime factor of $m$. We use Mertens' product $\prod_{p<z}(1-1/p)=e^{-\gamma}/\log z\,(1+O(1/\log z))$. For background on Mertens' product, see, e.g., \cite{Apostol1976}.
Choose
\[
z:=\Bigl\lceil x^{1/4}(\log x)^{-6}\Bigr\rceil.
\]
Step 1 (many $z$-rough integers). Let $\mathcal A=\{x+1,\dots,x+H-1\}$. By the linear Selberg sieve lower bound in dimension $1$ applied to consecutive integers \cite{HalRich1974,IK2004},
\[
S((x,x+H);z)\ge H\prod_{p<z}\Bigl(1-{1\over p}\Bigr)-C_0 z^2
=\frac{e^{-\gamma}+o(1)}{\log z}\,H\ -\ C_0 z^2.
\]
Standard references for the linear sieve lower bound include \cite{HalRich1974,IK2004}.
Since $H\asymp x^{1/2}$ and $z^2=x^{1/2}(\log x)^{-12}$, the main term dominates; hence, for all large $x$,
\[
S((x,x+H);z)\ge \frac{e^{-\gamma}}{2}\cdot\frac{H}{\log z}
\asymp \frac{\sqrt x}{\log x}.
\]
Step 2 (squarefree restriction). Count $m\in(x,x+H)$ with $P^-(m)\ge z$ that are not squarefree. Such $m$ have $p^2\mid m$ for some prime $p\ge z$. Split at $\sqrt H$.
\begin{itemize}
\item If $z\le p\le\sqrt H$, then $\#\{m\in(x,x+H): p^2\mid m\}\ll H/p^2$, whence the total over such $p$ is $\ll H\sum_{p\ge z}p^{-2}\ll H/z=o(\sqrt x/\log x)$.
\item If $\sqrt H<p\le\sqrt{X}$, write $m=p^2 r$ with $r\in\bigl(x/p^2,(x+H)/p^2\bigr)$; this interval has length $<1$, so at most one $r$ arises per $p$. Put $T:=\lfloor\sqrt{X/z}\rfloor\asymp x^{3/8}(\log x)^3$. If $p>T$ then $(x+H)/p^2<z$, and since $P^-(r)\ge z$ we must have $r=1$ whenever an $r$ exists; but $r=1\in(x/p^2,(x+H)/p^2)$ would force $x<p^2<x+H$, impossible because $x=n^2$ and $x+H=(n+1)^2$ are consecutive squares. Hence there is no contribution from $p>T$. If $\sqrt H<p\le T$ there are $\ll \pi(T)-\pi(\sqrt H)\ll T=o(\sqrt x/\log x)$ possibilities.
\end{itemize}
Altogether the nonsquarefree $z$-rough $m$ are $o(\sqrt x/\log x)$ in number.
Thus, for large $x$, there are $\gg \sqrt x/\log x$ integers $m\in(x,x+H)$ that are squarefree and satisfy $P^-(m)\ge z$.

Step 3 (excluding the cyclic obstructions). For squarefree $m=\prod_{i=1}^k p_i$ one has $\gcd(m,\varphi(m))=1$ if and only if there do not exist distinct primes $p_i\ne p_j$ with $p_i\mid(p_j-1)$. Write $p\to q$ for primes $p\le q$ with $q\equiv1\pmod p$. We bound the number of squarefree $z$-rough $m\in(x,x+H)$ admitting a pair $p\to q$.
Fix such $p,q$ dividing $m$. Then we may write $m=pqr$ with
\[
r\in I_{p,q}:=\Bigl({x\over pq},{x+H\over pq}\Bigr),\qquad P^-(r)\ge z,\qquad (r,pq)=1.
\]
Note $\#\{m: p,q\mid m\}\le \lceil H/(pq)\rceil$ always. Split into two subcases.
\begin{itemize}
\item Case 3.1: $pq\le H$. Then $\lceil H/(pq)\rceil\le 2H/(pq)$, and
\[
\sum_{\substack{p\ge z,\ q\ge z,\ p\to q\ pq\le H}}\#\{m\}
\ \ll\ H\sum_{p\ge z}{1\over p}\sum_{\substack{q\le H/p\ q\equiv1\ (\bmod p)}}{1\over q}.
\]
By the Brun--Titchmarsh inequality in arithmetic progressions and partial summation (e.g., \cite{MV2007,IK2004}),
\[
\sum_{\substack{q\le Y\ q\equiv1\ (\bmod p)}}{1\over q}\ \ll\ {\log\log Y\over \varphi(p)}\ \ll\ {\log\log X\over p},
\]
uniformly for $p\ge2$, $Y\le X/p$. See, e.g., \cite{MV2007,IK2004}.
Hence Case 3.1 contributes
\[
\ll\ H\sum_{p\ge z}{1\over p}\cdot{\log\log X\over p}\ \ll\ {H\,\log\log X\over z}
\ =\ o\!\Bigl({\sqrt X\over\log X}\Bigr).
\]
\item Case 3.2: $pq>H$. Here $|I_{p,q}|<1$, so for fixed $(p,q)$ there is either $0$ or $1$ admissible $r$. It is convenient to reparametrize by $r$. For $r\ge1$ put
\[
J_{p,r}:=\Bigl({x\over pr},{x+H\over pr}\Bigr),\qquad |J_{p,r}|={H\over pr}.
\]
\end{itemize}
If $m=pqr\in(x,x+H)$ with $pq>H$ and $p\to q$, then $q\in J_{p,r}$ and $q\equiv1\pmod p$; conversely, for fixed $p,r$ each such prime $q$ yields at most one $m$ (since $|I_{p,q}|<1$).
We require two bounds.
\begin{lemma}[Buchstab bound; cf.~\cite{Tenenbaum2015}]\label{lem:buchstab}
Uniformly for $t\ge y\ge2$,
\[
\sum_{\substack{n\le t\ P^-(n)\ge y}}\frac{1}{n}\ \ll\ \frac{\log(t/y)}{\log y}.
\]
\end{lemma}
\begin{proof}
Let $\Phi(u,y)=\#\{n\le u: P^-(n)\ge y\}$. For $u\ge y$, the Buchstab bound gives $\Phi(u,y)\ll u/\log y$. Partial summation yields
\[
\sum_{n\le t,\ P^-(n)\ge y}\frac{1}{n}=\frac{\Phi(t,y)}{t}+\int_y^t\frac{\Phi(u,y)}{u^2}\,du\ \ll\ \frac1{\log y}+\frac1{\log y}\int_y^t\frac{du}{u}\ \ll\ \frac{\log(t/y)}{\log y}.
\]
\end{proof}
\begin{lemma}[Weighted Brun--Titchmarsh for unions]\label{lem:bt-union}
Let $p$ be prime and $U\subset(2,\infty)$ be a finite union of disjoint intervals. Then
\[
\sum_{\substack{q\in\mathcal P\cap U\\ q\equiv1\ (\bmod p)}}\log q\ \le\ \frac{2\,\operatorname{mes}(U)}{\varphi(p)}.
\]
Consequently, for $U\subset[z,\infty)$,
\[
\#\{q\in\mathcal P\cap U: q\equiv1\ (\bmod p)\}\ \le\ \frac{2\,\operatorname{mes}(U)}{\varphi(p)\,\log z}.
\]
\end{lemma}
\begin{proof}
The first inequality is the weighted Brun--Titchmarsh inequality for $\theta$ in arithmetic progressions with the optimal constant $2$ (Montgomery--Vaughan form), see \cite[Th.~6.7]{MV2007}. By additivity it extends to finite disjoint unions $U$ by summing over the pieces. The counting bound then follows since $\log q\ge\log z$ on $U\subset[z,\infty)$. See also Iwaniec--Kowalski \cite[Th.~18.11]{IK2004}.
\end{proof}
For $pq>H$ the defining condition gives $r<(x+H)/(pq)\le (x+H)/H<2\sqrt x$, so it suffices to sum over $1\le r<2\sqrt x$. For fixed $p\ge z$ define
\[
U_p:=\bigcup_{\substack{1\le r<2\sqrt x\ P^-(r)\ge z}} J_{p,r} \ \subseteq\ (z, (x+H)/p].
\]
By Lemma~\ref{lem:buchstab},
\[
\operatorname{mes}(U_p)\le \sum_{\substack{1\le r<2\sqrt x\ P^-(r)\ge z}} |J_{p,r}|=\frac{H}{p}\sum_{\substack{1\le r<2\sqrt x\ P^-(r)\ge z}}{1\over r}\ \ll\ \frac{H}{p}\cdot\frac{\log(2\sqrt x/z)}{\log z}.
\]
Applying Lemma~\ref{lem:bt-union} with $U=U_p\subset[z,(x+H)/p]$ gives
\[
\#\{q\in U_p\cap\mathcal P: q\equiv1\ (\bmod p)\}\ \ll\ \frac{\operatorname{mes}(U_p)}{\varphi(p)\,\log z}
\ \ll\ \frac{H}{\varphi(p)\,p}\cdot\frac{\log(2\sqrt x/z)}{(\log z)^2}.
\]
For fixed $p$, the left-hand side bounds the number of $m$ counted in Case 3.2 with that $p$. Summing over $p\ge z$ and using $\varphi(p)\ge p-1$ and $\sum_{p\ge z}p^{-2}\ll 1/z$, we get
\[
\begin{aligned}
\sum_{p\ge z}\#\{m\text{ in Case 3.2 with this }p\} &\ll \frac{H}{(\log z)^2}\cdot\log\!\Bigl(\frac{2\sqrt x}{z}\Bigr)\sum_{p\ge z}\frac{1}{p\,\varphi(p)}\\
&\ll \frac{H}{(\log z)^2}\cdot\log\!\Bigl(\frac{2\sqrt x}{z}\Bigr)\cdot\frac{1}{z}.
\end{aligned}
\]
\[
\ll\ \frac{H}{z\,\log x}\ =\ o\!\Bigl(\frac{\sqrt x}{\log x}\Bigr).
\]
Combining Cases 3.1 and 3.2, the number of squarefree $z$-rough $m\in(x,x+H)$ admitting an obstruction $p\to q$ is $o(\sqrt x/\log x)$.
Step 4 (conclusion). Step 1 provides $\gg \sqrt x/\log x$ integers $m\in(x,x+H)$ with $P^-(m)\ge z$. Steps 2-3 remove only $o(\sqrt x/\log x)$ of them. Hence for all sufficiently large $n$ there exists $m\in(x,x+H)$ that is squarefree, $z$-rough, and has no pair $p\to q$ among its prime factors; equivalently $\gcd(m,\varphi(m))=1$, so $m$ is cyclic. The remaining finitely many $n$ are checked directly. Therefore for every $n\in\mathbb N$ there exists a cyclic integer $c$ with $n^2<c<(n+1)^2$. $\square$
\end{proof}






\subsubsection{Conjecture 9 (Cyclics between consecutive squares)}
\begin{theorem}[Cyclics between consecutive squares (resolves Conj.~9 of \cite{Cohen2025})]\label{thm:squares}
As $n\to\infty$,
\[
 C\big((n+1)^2\big) - C(n^2) \sim \frac{2n}{e^{\gamma}\,\log_3 n}\left(1-\frac{\gamma}{\log_3 n}\right).
\]
\end{theorem}

\begin{proof}
Write $L(x):=\log_3 x$ and recall Pollack's asymptotic expansion (as $x\to\infty$), with remainder term uniform when comparing $x$ and $\lambda x$ for fixed $\lambda$ in compact subsets of $(0,\infty)$ (which is all we use below):
\[
 C(x)=e^{-\gamma}x\Big(\frac{1}{L(x)}-\frac{\gamma}{L(x)^2}+O\Big(\frac{1}{L(x)^3}\Big)\Big).
\]
Set
\[
 \ell_1(x):=e^{-\gamma}\Big(\frac{1}{L(x)}-\frac{\gamma}{L(x)^2}\Big),\qquad a(x):=x\,\ell_1(x).
\]
For later reference we record the auxiliary function explicitly:
\begin{equation}\label{eq:aux-a}
 \boxed{\ a(x)\ :=\ e^{-\gamma}\,x\Big(\frac{1}{\log_3 x}-\frac{\gamma}{\log_3^2 x}\Big)\ }.
\end{equation}
We first prove the fixed-scale increment asymptotic; throughout, the $O(\cdot)$ bounds from Pollack's expansion are uniform for $\lambda$ in compact subsets of $(0,\infty)$, since we only compare $x$ and $\lambda x$ with $\lambda$ fixed.

\textit{Lemma (uniform increment).} For each fixed $\lambda>0$,
\[
 \frac{C(\lambda x)-C(x)}{a(x)(\lambda-1)}\;\longrightarrow\;1\qquad (x\to\infty),
\]
uniformly for $\lambda$ in compact subsets of $(0,\infty)$.

\textit{Proof.} Write $L:=L(x)$ and $L_\lambda:=L(\lambda x)$. Then
\[
 C(\lambda x)-C(x)
 =e^{-\gamma}x\Big(\lambda\Big(\frac{1}{L_\lambda}-\frac{\gamma}{L_\lambda^2}\Big)
 -\Big(\frac{1}{L}-\frac{\gamma}{L^2}\Big)\Big)\;+
 \;O\Big(\frac{x}{L^3}\Big).
\]
Set $f(u):=u^{-1}-\gamma u^{-2}$. Then
\[
 \lambda f(L_\lambda)-f(L)=(\lambda-1)f(L)+\lambda\bigl(f(L_\lambda)-f(L)\bigr).
\]
To handle $L_\lambda-L$, note that with $\log_1 x=\log x$ and $\log_2 x=\log\log x$,
\[
 L_\lambda-L
 =\log\Bigl(1+\frac{\log\bigl(1+\tfrac{\log\lambda}{\log x}\bigr)}{\log_2 x}\Bigr)
 =:\Delta.
\]
Since $\lambda$ is fixed, $\log\lambda$ is constant and hence
\[
 \Delta=O\!\Big(\frac{1}{\log x\,\log_2 x}\Big).
\]
By Taylor's theorem,
\[
 f(L_\lambda)-f(L)=f'(L)\,\Delta+O\Big(\frac{\Delta^2}{L^3}\Big),\qquad f'(L)=-\frac{1}{L^2}+\frac{2\gamma}{L^3}.
\]
Therefore
\[
 C(\lambda x)-C(x)=(\lambda-1)a(x)\;+
 \;e^{-\gamma}x\,\lambda f'(L)\,\Delta
 \;+
 \;O\Big(\frac{x\,\Delta^2}{L^3}\Big)
 \;+
 \;O\Big(\frac{x}{L^3}\Big).
\]
Divide by $a(x)(\lambda-1)=e^{-\gamma}x(\lambda-1)f(L)$. Since $f(L)\asymp 1/L$, $f'(L)=O(1/L^2)$, and $\Delta=O(1/(\log x\,\log_2 x))$, we obtain
\[
 \frac{e^{-\gamma}x\,\lambda f'(L)\,\Delta}{a(x)(\lambda-1)}
 \ll \frac{\Delta}{L}\;=\;O\!\Big(\frac{1}{\log x\,\log_2 x\,L}\Big)\to 0,
\]
while
\[
 \frac{x\,\Delta^2/L^3}{a(x)(\lambda-1)}\ll \frac{\Delta^2}{L^2}\to 0,\qquad
 \frac{x/L^3}{a(x)(\lambda-1)}\ll \frac{1}{L^2}\to 0.
\]
Hence the ratio tends to $1$, proving the lemma. \qed

As a corollary, for fixed $\lambda>0$ one has the drift expansion
\begin{equation}\label{eq:L-drift}
 L(\lambda x)=L(x)+\frac{\log\lambda}{\log_2 x}+O\!\Big(\frac{1}{\log_2^2 x}\Big)\qquad(x\to\infty),
\end{equation}
which is obtained by expanding $\log\bigl(\log(\lambda x)\bigr)$ around $\log x$; we use this below with $\lambda=2$.

By the lemma, $C$ belongs to de Haan's class of $\Pi$-variation of index $1$ with characteristic $\varphi(\lambda)=\lambda-1$ and auxiliary function $a(x)$ (see \cite[\S3.7]{BGT1989}). By Bingham--Goldie--Teugels \cite[Th.~3.7.2]{BGT1989} (local increments for $\Pi$-variation), we then have
\[
 C(x+h)-C(x)\sim a(x)\,\varphi\!\Big(1+\frac{h}{x}\Big)\sim a(x)\,\frac{h}{x}
 \qquad (x\to\infty),\quad h=o(x),
\]
using $\varphi'(1)=1$ for the second asymptotic.

Apply this with $x=n^2$ and $h=(n+1)^2-n^2=2n+1\sim 2n$ to get
\[
 C\big((n+1)^2\big)-C(n^2)\sim (2n)\,\ell_1(n^2)
 =(2n)\,e^{-\gamma}\Big(\frac{1}{L(n^2)}-\frac{\gamma}{L(n^2)^2}\Big).
\]
Now set $L:=\log_3 n$. Then
\[
 L(n^2)=\log_3(n^2)=\log\bigl(\log(2\log n)\bigr)=\log\bigl(\log_2 n+\log 2\bigr)
 = L + \delta,
\]
where
\[
 \delta=\log\Bigl(1+\frac{\log 2}{\log_2 n}\Bigr)=O\!\Big(\frac{1}{\log_2 n}\Big)
 \ =\ \frac{\log 2}{\log_2 n}+O\!\Big(\frac{1}{\log_2^2 n}\Big)\qquad\text{(by \eqref{eq:L-drift} with $\lambda=2$).}
\]
Write $\ell_1(n^2)=e^{-\gamma}f(L(n^2))$ with $f(u)=u^{-1}-\gamma u^{-2}$. Since $f'(u)=-u^{-2}+2\gamma u^{-3}=O(1/u^2)$, the mean value theorem gives
\[
 \ell_1(n^2)=e^{-\gamma}\Big(f(L)+O\Big(\frac{\delta}{L^2}\Big)\Big)
 =e^{-\gamma}\Big(\frac{1}{L}-\frac{\gamma}{L^2}\Big)\Big(1+O\Big(\frac{\delta}{L}\Big)\Big).
\]
Because $\delta/L=O\big(1/(\log_2 n\,\log_3 n)\big)=o(1)$, we may replace $L(n^2)$ by $L=\log_3 n$ inside $\ell_1$ at a relative $o(1)$ cost. Therefore
\[
 C\big((n+1)^2\big)-C(n^2)
 \sim (2n)\,e^{-\gamma}\Big(\frac{1}{\log_3 n}-\frac{\gamma}{(\log_3 n)^2}\Big)
 =\frac{2n}{e^{\gamma}\,\log_3 n}\Big(1-\frac{\gamma}{\log_3 n}\Big),
\]
which is exactly the asserted asymptotic.
\end{proof}


\subsubsection{Conjecture 14 (Near-square analog)}
\begin{theorem}[Near-square analog for cyclics (resolves Conj.~14 of \cite{Cohen2025})]\label{thm:near_square_cyclics}
Infinitely many \(n\in\mathbb{N}\) satisfy \(n^2+1\in\mathcal{C}\).
\end{theorem}

\begin{proof}
If $n$ is odd with $n\ge 3$, then $n^2+1$ is even $>2$, hence $\varphi(n^2+1)$ is even and $2\mid\gcd(n^2+1,\varphi(n^2+1))$, so $n^2+1\notin\mathcal C$. The exceptional case $n=1$ gives $n^2+1=2\in\mathcal C$. Thus it suffices to produce infinitely many even $n$ with $n^2+1\in\mathcal C$.

Write $n=2k$ and set $M_k:=4k^2+1$. Every odd prime divisor $p\mid M_k$ satisfies $p\equiv1\pmod4$ (Euler's criterion), and $M_k$ cannot be a perfect square because $x^2+1=y^2$ has no solutions with $x\ge1$. Throughout we use Szele's characterization of cyclic integers \cite{Szele1947}.

We use two inputs.

1) Half-dimensional sieve (with mild roughness). Let
\[
\begin{aligned}
\mathcal K(X;y)
 &:= \{\,1\le k\le X:\; M_k \text{ has at most two prime factors (counted with multiplicity)}\\
 &\qquad \text{and } P^-(M_k)>y\,\}.
\end{aligned}
\]
By Iwaniec's weighted half-dimensional sieve for a single quadratic polynomial $4k^2+1$, there exist absolute constants $\delta>0$, $A>0$ and $c_1>0$ such that for all sufficiently large $X$ and all $2\le y\le X^{\delta}$ one has
$$
\#\mathcal K(X;y)\ \ge\ \frac{c_1}{(\log X)^{A}\,\log y}\,X.
$$ (See e.g. Iwaniec \cite{Iwaniec1978} and the weighted half-dimensional sieve as presented in Friedlander--Iwaniec \cite{FI2010}.)

2) Congruential detection of the cyclicity obstruction for semiprimes. If $M_k=pq$ with primes $p\le q$, then
$$
\gcd(M_k,\varphi(M_k))=\gcd\bigl(pq,(p-1)(q-1)\bigr)=1
\iff p\nmid(q-1)
$$ (the condition $q\nmid(p-1)$ is automatic since $q>p$). Because $M_k\equiv pq\pmod{p^2}$, the condition $p\mid(q-1)$ is equivalent to
$$
M_k\equiv p\pmod{p^2}.
$$
Fix an odd prime $p\equiv1\pmod4$. The congruence $4k^2\equiv-1\pmod p$ has exactly two solutions $k\equiv\pm\gamma\pmod p$. Writing $k\equiv\gamma+\ell p\pmod{p^2}$ and expanding,
$$
4k^2+1\equiv 4\gamma^2+1+8\gamma\ell p\pmod{p^2}.
$$
Since $4\gamma^2\equiv-1\pmod p$, write $4\gamma^2+1\equiv sp\pmod{p^2}$ for some $s\in\mathbb Z/p\mathbb Z$, and obtain
$$
4k^2+1\equiv p\,(s+8\gamma\ell)\pmod{p^2}.
$$
As $\ell$ runs over $\mathbb Z/p\mathbb Z$, the residue $s+8\gamma\ell$ runs over all classes mod $p$, so exactly one lift from $\gamma$ (and likewise one from $-\gamma$) yields $4k^2+1\equiv p\pmod{p^2}$. Hence, for each such $p$, there are precisely two residue classes $\bmod\,p^2$ of $k$ with $M_k\equiv p\pmod{p^2}$. Consequently, for $X\ge1$ the number of $k\le X$ with $M_k\equiv p\pmod{p^2}$ is at most $2\lfloor X/p^2\rfloor\le 2X/p^2$.

We now complete the argument. Fix large $X$ and choose $y=(\log X)^B$ with $B>A+2$. Consider $\mathcal K(X;y)$. For any $k\in\mathcal K(X;y)$ there are two possibilities.

- If $M_k$ is prime, then $M_k\in\mathcal C$ since $\gcd(M_k,\varphi(M_k))=\gcd(M_k,M_k-1)=1$.

- If $M_k=pq$ with primes $p\le q$, then, because $M_k$ is not a square, $p<q$ and by definition of $\mathcal K(X;y)$ the least prime factor satisfies $p>y$. The obstruction to cyclicity is exactly $p\mid(q-1)$, which, by (2), is equivalent to $M_k\equiv p\pmod{p^2}$. For a fixed $p>y$ this occurs for at most $2X/p^2$ integers $k\le X$. Summing over all primes $p\equiv1\pmod4$ with $p>y$ gives that the number of obstructed $k\le X$ is at most
$$
\sum_{\substack{p>y\ p\equiv1\ (4)}} \frac{2X}{p^2}\ \le\ 2X\sum_{p>y}\frac{1}{p^2}\ \ll\ \frac{X}{y}\ =\ \frac{X}{(\log X)^B}.
$$
By the sieve lower bound,
$$
\#\mathcal K(X;y)\ \ge\ \frac{c_1}{(\log X)^{A}\,\log y}\,X\ =\ \frac{c_1}{B\,(\log X)^{A}\,\log\log X}\,X.
$$
Since $B>A+2$, we have $X/(\log X)^B=o\!\left(X/\bigl((\log X)^A\log\log X\bigr)\right)$. Thus, for all sufficiently large $X$,
$$
\#\{1\le k\le X:\ k\in\mathcal K(X;y)\text{ and }M_k\in\mathcal C\}
\ \ge\ \#\mathcal K(X;y)\ -\ \#\{\text{obstructed }k\le X\}
\ >\ 0.
$$
Therefore, for arbitrarily large $X$ there exists $1\le k\le X$ such that $M_k=4k^2+1\in\mathcal C$. Hence there are infinitely many such $k$, and with $n=2k$ we obtain infinitely many even integers $n$ for which $n^2+1\in\mathcal C$. This proves the claim. 
\end{proof}


\subsubsection{Conjecture 17 (Primes: k-fold Oppermann)}
\begin{theorem}[k-fold Oppermann for primes (resolves Conj.~17 of \cite{Cohen2025})]\label{thm:k_fold_oppermann_for_primes}
For every $k\in\mathbb{N}$ there exists $N(k)$ such that for all $n>N(k)$, both intervals $[n^2-n,n^2]$ and $[n^2,n^2+n]$ contain at least $k$ primes.
\end{theorem}

\begin{proof}
Fix $\theta=23/42>1/2$. By the Iwaniec--Pintz theorem (cf. the classical Nagura bound \cite{Nagura1952}), there exists $X_\theta\ge 2$ such that for all $x\ge X_\theta$ and all $y\le x^{\theta}$,
\[
\pi(x)-\pi(x-y)\;>\;\frac{y}{100\,\log x}.
\]
For such short-interval lower bounds one may also appeal to Baker--Harman--Pintz \cite{BHP2001}, which even allows $\theta=0.525$.

Apply this with $x=n^2$ and $y=n$. Since $\theta>1/2$, we have $y=n\le (n^2)^{1/2}\le (n^2)^{\theta}$, so for all sufficiently large $n$ (namely $n\ge \sqrt{X_\theta}$),
\[
N^{-}_{\mathcal P}(n)
\;=\;\pi(n^2)-\pi(n^2-n)
\;>\;\frac{n}{100\,\log(n^2)}
\;=\;\frac{n}{200\,\log n}.
\]
For the right half-interval, set $x'=n^2+n$ and $y'=n$, so that $[n^2,n^2+n]=[x'-y',x']$. Again $y'=n\le (n^2+n)^{1/2}\le (n^2+n)^{\theta}$ for $\theta>1/2$, hence for all sufficiently large $n$ (so that $x'\ge X_\theta$),
\[
N^{+}_{\mathcal P}(n)
\;=\;\pi(x')-\pi(x'-y')
\;>\;\frac{n}{100\,\log(n^2+n)}.
\]
For $n\ge 2$ one has $\log(n^2+n)=\log n+\log(n+1)\le 2\log n+\log 2\le 3\log n$, whence
\[
\frac{1}{\log(n^2+n)}\;\ge\;\frac{1}{3\,\log n},
\]
and therefore, for all sufficiently large $n$,
\[
N^{-}_{\mathcal P}(n)\;>\;\frac{n}{200\,\log n}
\quad\text{and}\quad
N^{+}_{\mathcal P}(n)\;>\;\frac{n}{300\,\log n}.
\]
Thus there exists $N_0$ such that for all $n\ge N_0$,
\[
\min\bigl\{N^{-}_{\mathcal P}(n),\,N^{+}_{\mathcal P}(n)\bigr\}
\;>\;\frac{n}{300\,\log n}.
\]
Since $n/(300\log n)\to\infty$ as $n\to\infty$, for any given $k\in\mathbb{N}$ we may choose $N(k)\ge N_0$ such that $n/(300\log n)\ge k$ for all $n\ge N(k)$. It follows that for all $n\ge N(k)$,
\[
N^{-}_{\mathcal P}(n)\ge k\quad\text{and}\quad N^{+}_{\mathcal P}(n)\ge k,
\]
as desired.
\end{proof}


\subsubsection{Conjecture 20 (k-fold Oppermann for cyclics)}
\begin{theorem}[k-fold Oppermann for cyclics (resolves Conj.~20 of \cite{Cohen2025})]\label{thm:k_fold_oppermann_for_cyclics}
For every $k\in \mathbb{N}$ there exists $N(k)$ such that for all $n>N(k)$, both intervals $[n^2-n,n^2]$ and $[n^2,n^2+n]$ contain at least $k$ cyclic integers.
\end{theorem}

\begin{proof}
Put $X:=n^2$ and $H:=n=\sqrt X$. We repeatedly use Szele's characterization of cyclic integers \cite{Szele1947}. It suffices to show that there exists $X_0$ such that for all $X\ge X_0$ and for each of the two intervals
$$
I\in\{[X-H,X],\,[X,X+H]\}
$$
one has
$$
\#(I\cap\mathcal C)\ \gg\ \frac{H\,\log\log\log X}{\log X},
$$
with an absolute implied constant; this clearly implies the theorem for any fixed $k$ since $H/\log X\to\infty$.

Fix $A\ge2$ and let $L:=(\log X)^A$. Write $\mathcal P_0:=\{p\in\mathcal P:3\le p\le L\}$. For $p\in\mathcal P_0$ define
$$
 y_p:=\frac{X}{p},\qquad \Delta_p:=\frac{H}{p},\qquad J^+(p):=[y_p,\,y_p+\Delta_p],\qquad J^-(p):=[y_p-\Delta_p,\,y_p].
$$
For $I=[X,X+H]$ use the right windows $J(p):=J^+(p)$; for $I=[X-H,X]$ use the left windows $J(p):=J^-(p)$. In both cases, if $r\in J(p)$ then $m:=pr\in I$. For $X$ large, the windows are disjoint in the following sense: for every real $r$,
$$
W(r):=\#\{p\in\mathcal P_0:\ r\in J(p)\}\in\{0,1\}.
$$
Indeed, if $r\in J(p)$ then $pr\in[X-H,X+H]$, so necessarily
$$
p\in\left(\frac{X-H}{r},\,\frac{X+H}{r}\right),
$$
an interval of length $\ll H/r\le H/(X/L-\Delta_L)\ll L/\sqrt X<1$ for large $X$, whence uniqueness of $p$.

Set $U:=\bigcup_{p\in\mathcal P_0} J(p)$. Then, using Mertens' theorem \cite{Apostol1976,MV2007},
$$
|U|=\sum_{p\in\mathcal P_0}|J(p)|=\sum_{p\le L}\frac{H}{p}
 = H\sum_{p\le L}\frac1p\sim H\log\log L\asymp H\log\log\log X.\tag{1}
$$
We first show that $U$ contains many primes, and then exclude a single forbidden residue class inside each window to ensure cyclicity (via Szele \cite{Szele1947}).

\begin{lemma}[Many primes in $U$]\label{lem:many-primes-U}
Using the prime number theorem with de~la~Vall\'ee Poussin error term \cite{MV2007,IK2004}, for all sufficiently large $X$,
$$
\#\bigl(U\cap\mathcal P\bigr)\ \gg\ \frac{|U|}{\log X}\ \asymp\ \frac{H\,\log\log\log X}{\log X}.\tag{2}
$$
\end{lemma}

\begin{proof}
Let $\Lambda$ be the von Mangoldt function and $\psi(x):=\sum_{n\le x}\Lambda(n)$. By the PNT in the form
$$
\psi(t)=t+O\bigl(t\,e^{-c\sqrt{\log t}}\bigr)\qquad (c>0),
$$
we have, for each window $J(p)=[\alpha,\beta]$ with $\beta-\alpha=\Delta_p$ and $\alpha\asymp X/p$,
$$
\sum_{\alpha<n\le\beta}\Lambda(n)
= \psi(\beta)-\psi(\alpha)
= \Delta_p + O\!\left(\alpha\,e^{-c\sqrt{\log \alpha}}\right) + O(1).
$$
Summing over the disjoint windows and using $\alpha\ge X/L$ and $\#\mathcal P_0=\pi(L)$,
$$
\sum_{n\in U\cap\mathbb Z}\Lambda(n)
= \sum_{p\le L}\Delta_p + O\!\left(\sum_{p\le L}\frac{X}{p}\,e^{-c\sqrt{\log(X/p)}}\right)+O\bigl(\pi(L)\bigr)
= |U| + o(|U|).
$$
Indeed, uniformly for $p\le L=(\log X)^A$ we have $e^{-c\sqrt{\log(X/p)}}\le e^{-c'\sqrt{\log X}}$ for some $c'>0$, so the middle error is $\ll X e^{-c'\sqrt{\log X}}\sum_{p\le L}\tfrac1p\ll X e^{-c'\sqrt{\log X}}\log\log L=o(|U|)$, and $\pi(L)=o(|U|)$ since $|U|\asymp \sqrt X\log\log\log X$ while $\pi(L)\asymp L/\log L=(\log X)^A/\log\log X$.

Prime powers contribute negligibly: the number of prime squares in a single window $J(p)$ is $\ll \Delta_p/\sqrt{y_p}= (H/p)/\sqrt{X/p}=(H/\sqrt X)p^{-1/2}$, so over all $p\le L$ there are $\ll \sum p^{-1/2}\ll \sqrt L/\log L$ prime squares in $U$; higher prime powers contribute even less. Hence
$$
\sum_{\substack{n\in U\cap\mathbb Z\\ n=\text{prime power},\,\nu\ge2}}\Lambda(n)\ \ll\ (\log X)\Bigl(\frac{\sqrt L}{\log L}+\pi(L)\Bigr)
= o(|U|).
$$
Therefore
$$
\sum_{\substack{r\in U\cap\mathcal P}}\log r
= \sum_{n\in U\cap\mathbb Z}\Lambda(n) - o(|U|)
= |U|+o(|U|).
$$
For both choices of $I$ and all $r\in U$ we have $r\le X$ (indeed $r\le y_p+\Delta_p\le (X+H)/3\le X$ for $p\ge3$), hence $\log r\le\log X$. Consequently,
$$
\#(U\cap\mathcal P)\ \ge\ \frac{\sum_{r\in U\cap\mathcal P}\log r}{\log X}
\ =\ \frac{|U|}{\log X}+o\!\left(\frac{|U|}{\log X}\right),
$$
which is (2).
\end{proof}

\begin{lemma}[Excluding $1\pmod p$ inside its window]\label{lem:exclude-one-mod-p}
By the Brun--Titchmarsh inequality \cite{MV2007}, for all large $X$,
$$
\sum_{p\in\mathcal P_0}\#\{r\in J(p)\cap\mathcal P: r\equiv1\ (\bmod\ p)\}
\ \ll\ \frac{H}{\log X}.\tag{3}
$$
\end{lemma}

\begin{proof}
For fixed $p\in\mathcal P_0$, Brun--Titchmarsh on the short interval $J(p)$ gives
$$
\#\{r\in J(p)\cap\mathcal P: r\equiv1\ (\bmod\ p)\}
\ \le\ \frac{2|J(p)|}{\varphi(p)\,\log(|J(p)|/p)}
\ =\ \frac{2(H/p)}{(p-1)\,\log(H/p^2)}.
$$
Since $p\le L=(\log X)^A$ and $H=\sqrt X$, one has $H/p^2\to\infty$ and $\log(H/p^2)\ge \tfrac12\log X-2\log L\sim \tfrac12\log X$. Summing over $p\in\mathcal P_0$ and using $\sum_{p\ge3}1/(p(p-1))<\infty$ yields (3).
\end{proof}

We now produce cyclic integers. By Lemma~\ref{lem:many-primes-U} and (1),
$$
\#(U\cap\mathcal P)\ \gg\ \frac{|U|}{\log X}\ \asymp\ \frac{H\,\log\log\log X}{\log X}.
$$
By Lemma~\ref{lem:exclude-one-mod-p}, at most $\ll H/\log X$ of these primes lie in the forbidden residue class $1\pmod p$ inside their unique window. Consequently,
$$
\#\Bigl\{r\in U\cap\mathcal P:\ \text{if }r\in J(p)\text{ then }r\not\equiv1\ (\bmod\ p)\Bigr\}
\ \gg\ \frac{H\,\log\log\log X}{\log X}.
$$
For each such prime $r$ there is a unique $p\in\mathcal P_0$ with $r\in J(p)$; then $m:=pr\in I$, and $m$ is odd and squarefree with prime factors $p<r$. For squarefree $m=\prod p_i$, Szele's criterion \cite{Szele1947} says that $\gcd(m,\varphi(m))=1$ if and only if for all distinct $i\ne j$ one has $p_j\not\equiv1\pmod{p_i}$. Here the only pair is $(p,r)$. We ensured $r\not\equiv1\pmod p$, while $p\not\equiv1\pmod r$ holds because $p<r$. Thus each such $r$ yields a distinct $m=pr\in I\cap\mathcal C$ (injectivity follows from the disjointness $W(r)\in\{0,1\}$).

Therefore, for each of the intervals $I\in\{[X-H,X],[X,X+H]\}$ we have
$$
\#(I\cap\mathcal C)\ \gg\ \frac{H\,\log\log\log X}{\log X}.
$$
Since $H/\log X\to\infty$, this lower bound exceeds any prescribed $k$ for all $X\ge X_0(k)$. Recalling $X=n^2$ and $H=n$ completes the proof.
\end{proof}


\subsubsection{Conjecture 32 (Twin cyclics between consecutive cubes)}
\begin{theorem}[Twin cyclics between consecutive cubes (resolves Conj.~32 of \cite{Cohen2025})]\label{thm:twin_cyclics_between_consecutive_cubes}
At least two twin cyclic pairs between \(n^3\) and \((n+1)^3\); more generally, at least k beyond a threshold.
\end{theorem}

\begin{proof}
For an integer $m$, write $m\in\mathcal C$ if and only if $\gcd(m,\varphi(m))=1$ (Szele \cite{Szele1947}). Since $\varphi$ is multiplicative and $\varphi(p)=p-1$, we have:
- If $p^2\mid m$ then $p\mid\varphi(m)$, hence every cyclic $m$ is squarefree.
- Aside from $m=2$, any even $m$ satisfies $2\mid\varphi(m)$, so the only even cyclic integer is $2$.
- If $m$ is odd and squarefree with prime factorization $m=\prod_{i=1}^r p_i$, then $\varphi(m)=\prod_{i=1}^r (p_i-1)$ and
  $$\gcd(m,\varphi(m))=1\iff \forall i\ne j:\ p_i\nmid (p_j-1).\tag{1}$$

Fix large $n$ and set
$$X:=n^3,\qquad H:=(n+1)^3-n^3=3n^2+3n+1\asymp X^{2/3},\qquad H':=H-2.$$
We will produce many $m\in J:=(X,X+H']$ such that $m$ and $m+2$ are odd, squarefree, and each satisfies (1). For such an $m$, both $m$ and $m+2$ lie in $I_n=(X,X+H]$, and since $m+1$ is even $>2$, it is not cyclic; hence $(m,m+2)$ is a twin cyclic pair, and these two cyclics are consecutive.

1) Sieve for twin $z$-rough integers in $J$. For $z\ge 3$ let
$$\mathcal{S}_0:=\{m\in J : (m,P(z))=(m+2,P(z))=1\},\qquad P(z):=\prod_{p\le z}p.$$
This is the sifted set for the two linear forms $n$ and $n+2$, with local sieve weights $w(2)=1$ and $w(p)=2$ for $p\ge 3$ (the number of forbidden residues mod $p$ coming from $p\mid n$ or $p\mid n+2$). For squarefree $d\mid P(z)$ put $w(d):=\prod_{p\mid d}w(p)$. By the Chinese remainder theorem the number of excluded residue classes modulo $d$ equals $w(d)$. Hence, for each such $d$,
$$A_d:=\#\{m\in J: m\equiv a\ (\bmod d)\text{ for some excluded }a\}\;=\;\frac{H'\,w(d)}{d}+r_d,\quad |r_d|\le w(d).\tag{2}$$
Applying the lower-bound $\beta$-sieve (fundamental lemma of sieve theory for dimension $2$; see \cite{Greaves2001,IK2004}) with level $D\le z^u$ ($u\ge 2$) and using (2), we obtain
$$\#\mathcal{S}_0\;\ge\;H'\,V(z)\Bigl(1-O\bigl(e^{-u/2}\bigr)\Bigr)-\sum_{\substack{d\le D\ d\,\text{sqfree},\ d\mid P(z)}}|r_d|,\qquad V(z):=\prod_{p\le z}\Bigl(1-\frac{w(p)}{p}\Bigr).\tag{3}$$
Using $|r_d|\le w(d)$ and $\sum_{d\le D}2^{\omega(d)}\ll D\log D$, we have
$$\sum_{\substack{d\le D\ d\,\text{sqfree},\ d\mid P(z)}}|r_d|\;\le\;\sum_{d\le D}w(d)\;\ll\;D\log D.\tag{4}$$
Moreover,
$$V(z)=\Bigl(1-\tfrac12\Bigr)\prod_{3\le p\le z}\Bigl(1-\frac{2}{p}\Bigr)\asymp \frac{1}{(\log z)^2}$$
by Mertens-type estimates \cite{Apostol1976} and the identity $(1-2/p)=(1-1/p)^2(1+O(1/p^2))$.
Choose $D:=(H')^{1/2}$ and $z:=(\log X)^A$ with a fixed admissible $A$. Then
\[
 u\ :=\ \frac{\log D}{\log z}\ =\ \frac{\tfrac12\log H'}{A\log\log X}\ \asymp\ \frac{\log X}{A\log\log X}\ \longrightarrow\ \infty,
\]
so $e^{-u/2}=o(1)$. Moreover,
$$H'\,V(z)\asymp \frac{H'}{(\log z)^2}\asymp \frac{H'}{(\log\log X)^2},\qquad D\log D\ll (H')^{1/2}\log X=o\!\left(\frac{H'}{(\log\log X)^2}\right).$$
Therefore, for all sufficiently large $X$,
$$\#\mathcal{S}_0\gg \frac{H'}{(\log\log X)^2}.\tag{5}$$

2) Forcing squarefreeness. Let
$$\mathcal{S}_1:=\{m\in\mathcal{S}_0: \mu^2(m)=\mu^2(m+2)=1\}.$$
The number of $m\in J$ for which $p^2\mid m$ for some $p>z$ is $\ll\sum_{p>z} H'/p^2\ll H'/z$, and the same bound holds for $m+2$. Thus
$$\#(\mathcal{S}_0\setminus\mathcal{S}_1)\ll \frac{H'}{z},\qquad \#\mathcal{S}_1\gg \frac{H'}{(\log\log X)^2}-\frac{H'}{z}.\tag{6}$$

3) Eliminating cyclic obstructions. For odd squarefree $m$, condition (1) is equivalent to $m\in\mathcal C$. If $m\in\mathcal{S}_1$, all prime factors of $m$ and $m+2$ exceed $z$, so the only obstruction to (1) for $m$ (respectively for $m+2$) is the existence of primes $z<p<q$ with $p,q\mid m$ (respectively $p,q\mid m+2$) and $q\equiv 1\pmod p$. Define
$$B(X,H';z):=\#\Bigl\{m\in J:\ \exists\ z<p<q,\ p,q\mid m,\ q\equiv 1\ (\bmod p)\Bigr\}.$$
Bounding $\mathbf{1}_{\exists(p,q)}\le\sum_{p,q}\mathbf{1}_{pq\mid m}$ and summing over $m$, we get
\begin{align*}
B(X,H';z)
&\le \sum_{z<p}\ \sum_{\substack{z<q\ q\equiv 1\ (\bmod p)}}\Bigl\lfloor \frac{H'}{pq}\Bigr\rfloor
\ll H'\sum_{p>z}\frac{1}{p}\sum_{\substack{q\le X+H\ q\equiv 1\ (\bmod p)}}\frac{1}{q}.
\end{align*}
\begin{flushleft}
By the Brun--Titchmarsh inequality and partial summation (e.g., \cite{MV2007,IK2004}), uniformly for $p\le X$, one has $\sum_{q\le X,\ q\equiv 1\ (\bmod p)}\!\frac{1}{q}\ll \frac{\log\log X}{\varphi(p)}=\frac{\log\log X}{p-1}$.
$$B(X,H';z)\ll H'(\log\log X)\sum_{p>z}\frac{1}{p(p-1)}\ll \frac{H'\,\log\log X}{z}.\tag{7}$$
\end{flushleft}
The same bound holds for the set of $m\in J$ for which $m+2$ has such a pair of prime factors. Therefore the number of $m\in\mathcal{S}_1$ for which either $m$ or $m+2$ fails (1) is $\ll H'(\log\log X)/z$.

4) Conclusion. Let $\mathcal{G}$ be the set of $m\in J$ such that: (a) $(m,P(z))=(m+2,P(z))=1$; (b) $\mu^2(m)=\mu^2(m+2)=1$; (c) both $m$ and $m+2$ satisfy (1). Then, by (5)-(7),
$$\#\mathcal{G}\ge \#\mathcal{S}_1-2B(X,H';z)\gg \frac{H'}{(\log\log X)^2}-\frac{H'}{z}-\frac{H'\,\log\log X}{z}.$$
To dominate the error terms in (6) and (7) by the main term in (5) with a single explicit choice of $A$, observe that
\[
 \frac{H'}{z}\ \le\ \frac{H'}{(\log X)^A}\ \le\ \frac{1}{2}\cdot\frac{H'}{(\log\log X)^2},\qquad
 \frac{H'\,\log\log X}{z}\ \le\ \frac{1}{2}\cdot\frac{H'}{(\log\log X)^2},
\]
provided $(\log X)^A\ge 2(\log\log X)^2$ and $(\log X)^A\ge 2(\log\log X)^3$. Both hold for all large $X$ once $A\ge4$. We therefore fix the admissible choice
\[
 z\ :=\ (\log X)^4.
\]
With this choice the two subtracted families are $o\!\bigl(H'/(\log\log X)^2\bigr)$, while the main term is $\asymp H'/(\log\log X)^2$. Hence, for all sufficiently large $X$ (equivalently, large $n$),
$$\#\mathcal{G}\gg \frac{H'}{(\log\log X)^2}\to\infty\quad(n\to\infty).$$
For each $m\in\mathcal{G}$, both $m$ and $m+2$ lie in $I_n$ and are odd, squarefree, and satisfy (1), hence both are cyclic. Since the only integer strictly between them is $m+1$, which is even $>2$ and therefore not cyclic, $(m,m+2)$ is a twin cyclic pair and these cyclics are consecutive. Consequently the number of twin cyclic pairs in $I_n$ tends to $\infty$ as $n\to\infty$. In particular, for any fixed $k\in\mathbb N$ there exists $n_0(k)$ such that for all $n\ge n_0(k)$, the interval $I_n$ contains at least $k$ twin cyclic pairs. $\square$
\end{proof}


\subsubsection{Conjecture 36 (Infinitely many SG cyclics)}
\begin{theorem}[Infinitely many SG cyclics (resolves Conj.~36 of \cite{Cohen2025})]\label{thm:infinitely_many_sg_cyclics}
There are infinitely many \(c\in\mathcal{C}\) with \(2c+1\in\mathcal{C}\).
\end{theorem}

\begin{proof}
Fix a large real parameter $x$ and set
$$y:=\exp\bigl((\log\log x)^{1/2}\bigr).$$
Write $P^-(n)$ for the least prime divisor of $n$ (with $P^-(1)=\infty$). Recall Szele's characterization \cite{Szele1947}: an integer $n$ is cyclic if and only if $n$ is squarefree and for all distinct primes $p,q\mid n$ we have $p\nmid(q-1)$.

Step 1 (simultaneous roughness via CRT). For a prime $r\le y$, the conditions $r\nmid n$ and $r\nmid(2n+1)$ exclude the residue classes $n\equiv0\pmod r$ and, for odd $r$, $n\equiv t_r\pmod r$ where $t_r$ satisfies $2t_r\equiv-1\pmod r$. Thus, for $r=2$ we exclude one class, and for odd $r\le y$ we exclude two classes. Put $w(2)=1$ and $w(r)=2$ for odd $r$. Let
$$M:=\prod_{r\le y} r,\qquad R:=\prod_{r\le y}(r-w(r)).$$
By the Chinese remainder theorem, among any complete residue system modulo $M$ exactly $R$ residues satisfy all local exclusions. Hence, for $x\ge1$,
$$\#\mathcal S_0(x,y)=\Bigl\lfloor\frac{x}{M}\Bigr\rfloor R+O(R)
= x\prod_{r\le y}\Bigl(1-\frac{w(r)}{r}\Bigr)+O\!\left(\prod_{r\le y}(r-w(r))\right).$$
Since $\theta(y)=\sum_{p\le y}\log p\sim y$ and $y=o(\log x)$, we have $\log M\sim y=o(\log x)$, so $M=o(x)$; thus the $O(R)$ term is $o\bigl(x\prod_{r\le y}(1-w(r)/r)\bigr)$. Therefore
$$\#\mathcal S_0(x,y)\sim x\prod_{r\le y}\Bigl(1-\frac{w(r)}{r}\Bigr)=\frac{1}{2}\,x\prod_{3\le r\le y}\Bigl(1-\frac{2}{r}\Bigr).$$
Using Mertens-type estimates (e.g., \cite{Apostol1976}), namely $\log(1-2/p)=-2/p+O(1/p^2)$ and $\sum_{p\le y}1/p=\log\log y+O(1)$, we get
$$\prod_{3\le r\le y}\Bigl(1-\frac{2}{r}\Bigr)=\frac{c+o(1)}{(\log y)^2}$$
for some absolute $c>0$. Hence
$$\#\mathcal S_0(x,y)\asymp \frac{x}{(\log y)^2}\asymp \frac{x}{\log\log x}.$$
By construction, $n\in\mathcal S_0(x,y)$ if and only if $P^-(n)>y$ and $P^-(2n+1)>y$.

Step 2 (squarefreeness for $n$ and $2n+1$). Let
$$\mathcal S_1(x,y):=\{\,n\le x:\ \exists\ p>y\text{ prime with }p^2\mid n\text{ or }p^2\mid(2n+1)\,\}.$$
For a fixed prime $p>y$, we have $\#\{n\le x: p^2\mid n\}\le x/p^2$. Also $2n+1\equiv0\pmod{p^2}$ has at most one solution modulo $p^2$ (for odd $p$; for $p=2$ it has none), hence
$$\#\{n\le x: p^2\mid(2n+1)\}\le \Bigl\lfloor\frac{x}{p^2}\Bigr\rfloor+1\le \frac{x}{p^2}+1,$$
but the last "$+1$" can occur only when $p^2\le 2x+1$. Therefore
\[
\#\mathcal S_1(x,y)\le \sum_{p>y}\frac{x}{p^2}+\sum_{\substack{p>y\ p^2\le 2x+1}}\Bigl(\frac{x}{p^2}+1\Bigr)
\ll x\sum_{p>y}\frac{1}{p^2}+\pi\!\bigl(\sqrt{2x+1}\bigr)
\ll \frac{x}{y}+\frac{\sqrt{x}}{\log x}.
\]
Since $y=\exp\bigl((\log\log x)^{1/2}\bigr)$, both $x/y$ and $\sqrt{x}/\log x$ are $o\bigl(x/(\log y)^2\bigr)$; thus
$$\#\mathcal S_1(x,y)=o\!\left(\frac{x}{(\log y)^2}\right).$$

Step 3 (excluding the internal divisibility obstruction for $n$). Let $\mathcal B_n$ be the set of $n\le x$ for which there exist primes $p,q>y$ with $p\mid(q-1)$ and $pq\mid n$. Then
$$\#\mathcal B_n\le \sum_{p>y}\ \sum_{\substack{q\le x/p\ q\equiv1\ (\bmod p)}} \Bigl\lfloor\frac{x}{pq}\Bigr\rfloor\ll x\sum_{p>y}\frac{1}{p}\sum_{\substack{q\le x/p\ q\equiv1\ (\bmod p)}}\frac{1}{q}.$$
By the Brun--Titchmarsh inequality and partial summation (e.g., \cite{MV2007,IK2004}),
$$\sum_{\substack{q\le X\ q\equiv1\ (\bmod p)}}\frac{1}{q}\ll \frac{\log\log X}{\varphi(p)}\ll \frac{\log\log X}{p}$$
uniformly in $X\ge2$, hence
$$\#\mathcal B_n\ll x\sum_{p>y}\frac{\log\log(x/p)}{p^2}\ll x\cdot\frac{\log\log x}{y}=o\!\left(\frac{x}{(\log y)^2}\right).$$

Step 4 (excluding the internal divisibility obstruction for $2n+1$). Define $\mathcal B_{2n+1}$ as the set of $n\le x$ for which there exist primes $p,q>y$ with $p\mid(q-1)$ and $pq\mid(2n+1)$. For fixed $p,q$, the congruence $2n+1\equiv0\pmod{pq}$ has $\ll x/(pq)+1$ solutions $n\le x$. Therefore
\[
\#\mathcal B_{2n+1}\ll x\sum_{p>y}\frac{1}{p}\sum_{\substack{q\le 2x+1\ q\equiv1\ (\bmod p)}}\frac{1}{q}\; +\; \sum_{\substack{p>y\ p\le 2x+1}}\!\pi(2x+1; p,1)=:S_1+S_2.
\]
For $S_1$, the same Brun-Titchmarsh-partial summation bound as in Step 3 gives $S_1\ll x(\log\log x)/y=o\bigl(x/(\log y)^2\bigr)$.

For $S_2$, set $X:=2x+1$. We split the sum at $X/e$:
$$S_2=\sum_{\substack{p>y\ p\le X/e}}\!\pi(X;p,1)+\sum_{X/e<p\le X}\!\pi(X;p,1)=:T_1+T_2.$$
For $T_1$, by the Brun--Titchmarsh inequality (valid for $p<X$; see, e.g., \cite{MV2007,IK2004}),
$$\pi(X;p,1)\le \frac{2X}{\varphi(p)\,\log(X/p)}\le \frac{4X}{p\,\log(X/p)}\quad(p\ge3),$$
and the $p=2$ term is harmless. Partition the range $p\le X/e$ into bins $X/e^{j+1}<p\le X/e^{j}$ for integers $1\le j\le \lfloor\log X\rfloor-1$. On each bin, $\log(X/p)\asymp j$, so
$$\sum_{X/e^{j+1}<p\le X/e^{j}} \frac{1}{p\,\log(X/p)}\ll \frac{1}{j}\sum_{X/e^{j+1}<p\le X/e^{j}}\frac{1}{p}
\ll \frac{1}{j}\Bigl(\log\log\frac{X}{e^{j}}-\log\log\frac{X}{e^{j+1}}+O\!\Bigl(\frac{1}{\log X}\Bigr)\Bigr).$$
\begin{flushleft}
Summing over $j$ and using the telescoping together with $\sum_{j\le \log X} \frac{1}{j(\log X-j)}\ll (\log\log X)/(\log X)$,
$$\sum_{p\le X/e}\frac{1}{p\,\log(X/p)}\ll \frac{\log\log X}{\log X}.$$
\end{flushleft}
Hence
$$T_1\ll X\cdot\frac{\log\log X}{\log X}.$$
For $T_2$, note that $p>X/e$ implies $\lfloor X/p\rfloor\le e$. Since $\pi(X;p,1)\le \lfloor X/p\rfloor$, we get
$$T_2\le \sum_{X/e<p\le X}\Bigl\lfloor\frac{X}{p}\Bigr\rfloor\le \sum_{k=1}^{\lfloor e\rfloor}\!k\,\bigl(\pi(X/k)-\pi(X/(k+1))\bigr)\ll \sum_{k=1}^{\lfloor e\rfloor}\!\frac{X/k}{\log(X/k)}\ll \frac{X}{\log X}.$$
Combining the two ranges,
$$S_2=T_1+T_2\ll X\frac{\log\log X}{\log X}=o\!\left(\frac{X}{(\log y)^2}\right)\quad\text{since }(\log y)^2=\log\log X.$$
Thus $\#\mathcal B_{2n+1}=o\bigl(x/(\log y)^2\bigr)$.

\medskip
\noindent\textit{Lemma (aggregated parameters).} With $y=\exp\sqrt{\log\log x}$ one has, for some absolute $c_0>0$ and all sufficiently large $x$,
\[
 \#\mathcal S_0(x,y)\ \ge\ c_0\,\frac{x}{(\log y)^2},\qquad
 \#\mathcal S_1(x,y)\ \ll\ \frac{x}{y}+\frac{\sqrt{x}}{\log x}\ =\ o\!\left(\frac{x}{(\log y)^2}\right),
\]
and
\[
 \#\mathcal B_n\ \ll\ x\,\frac{\log\log x}{y}\ =\ o\!\left(\frac{x}{(\log y)^2}\right),\qquad
 \#\mathcal B_{2n+1}\ \ll\ x\,\frac{\log\log x}{\log x}\ =\ o\!\left(\frac{x}{(\log y)^2}\right).
\]
In particular, $\#\mathcal G(x)\gg x/(\log y)^2\asymp x/\log\log x$.

\noindent\textit{Proof.} Combine the bounds proved in Steps 1--4 and note that $(\log y)^2=\log\log x$.

Step 5 (conclusion). Set
$$\mathcal G(x):=\mathcal S_0(x,y)\setminus\bigl(\mathcal S_1(x,y)\cup\mathcal B_n\cup\mathcal B_{2n+1}\bigr).$$
By Steps 1--4 and the lemma,
$$\#\mathcal G(x)\ge \#\mathcal S_0(x,y)-\#\mathcal S_1(x,y)-\#\mathcal B_n-\#\mathcal B_{2n+1}\gg \frac{x}{(\log y)^2}\asymp \frac{x}{\log\log x}.$$
For each $n\in\mathcal G(x)$, all prime divisors of $n$ and of $2n+1$ exceed $y$, these integers are squarefree, and there is no pair of distinct primes $p,q$ dividing $n$ (respectively $2n+1$) with $p\mid(q-1)$. By Szele's characterization, both $n$ and $2n+1$ are cyclic. Hence every $n\in\mathcal G(x)$ is a Sophie Germain cyclic.

It follows that, for all sufficiently large $x$,
$$C_{\mathrm{SG}}(x)\ge \#\mathcal G(x)\gg \frac{x}{\log\log x},$$
so there are infinitely many Sophie Germain cyclics. 
\end{proof}


\subsubsection{Conjecture 37 (SG cyclics modulo 3)}
\begin{theorem}[SG cyclics modulo 3 (resolves Conj.~37 of \cite{Cohen2025})]\label{thm:sg_cyclics_modulo_3}
As the number of SG cyclics grows, the limiting fractions congruent to 1 and 3 mod 3 are equal.
\end{theorem}

\begin{proof}
\[N_r(x):=\#\{n\le x:\, n\in S,\ n\equiv r\pmod 3\},\qquad C_{\mathrm{SG}}(x):=\#\{n\le x:\, n\in S\}.
\]
Then
$$\lim_{x\to\infty}\frac{N_1(x)-N_0(x)}{C_{\mathrm{SG}}(x)}=0.$$
Equivalently, among SG cyclics the limiting fractions in the classes $1$ and $3\pmod3$ are equal.

Write $L_1(n)=n$ and $L_2(n)=2n+1$. For a positive integer $m$, the condition $\gcd(m,\varphi(m))=1$ is equivalent to: (i) $m$ is squarefree, and (ii) there is no pair of distinct prime divisors $p,q\mid m$ with $p\mid(q-1)$ (cf. Szele \cite{Szele1947}). Indeed, $p^2\mid m\Rightarrow p\mid\varphi(m)$, and if $p,q\mid m$ with $p\mid(q-1)$ then also $p\mid\varphi(m)=\prod_{r\mid m}(r-1)$; conversely, if $m$ is squarefree and no such pair occurs then $\gcd\bigl(m,\prod_{q\mid m}(q-1)\bigr)=1$.

Fix large $x$ and set $y:=\lfloor\log x\rfloor\ge3$. We split small primes ($\le y$) from large primes ($>y$).

1) Lower bound for $C_{\mathrm{SG}}(x)$. Let $\mathcal P(y)=\prod_{p\le y}p$ and
$$\mathcal T(y):=\Bigl\{n\le x:\, \gcd\bigl(L_1(n)L_2(n),\mathcal P(y)\bigr)=1\Bigr\}.$$
For each prime $p$, the set of residues $n\pmod p$ with $p\mid L_1(n)L_2(n)$ has cardinality $\nu(p)$, where $\nu(2)=1$ and $\nu(p)=2$ for $p\ge3$. By the two-dimensional Brun--Selberg sieve (fundamental lemma; see \cite{HalRich1974,Greaves2001,IK2004}),
$$\#\mathcal T(y)\asymp x\prod_{p\le y}\Bigl(1-\frac{\nu(p)}{p}\Bigr)=\Bigl(1-\tfrac12\Bigr)x\prod_{3\le p\le y}\Bigl(1-\frac{2}{p}\Bigr)\asymp \frac{x}{(\log y)^2}.\tag{1}$$

Remove from $\mathcal T(y)$ those $n$ that still violate cyclicity for $L_1$ or $L_2$ at large primes ($>y$).

- Large square divisors: for $j\in\{1,2\}$ the number of $n\le x$ with $p^2\mid L_j(n)$ for some prime $p>y$ is bounded by
\[\sum_{y<p\le\sqrt{2x+1}}\Bigl(\frac{x}{p^2}+1\Bigr)\ \ll\ x\sum_{p>y}\frac{1}{p^2}\ +\ \pi(\sqrt{2x+1})\ \ll\ \frac{x}{y}\ +\ \frac{\sqrt{x}}{\log x}.\tag{2}\]

- Large-large pair violations in one $L_j$: existence of primes $y<p<q\le 2x+1$ with $p\mid L_j(n)$, $q\mid L_j(n)$ and $q\equiv1\pmod p$. For each fixed pair $(p,q)$ there is exactly one residue class $\bmod\,pq$ for $n$ (by CRT), hence the count per pair is $\frac{x}{pq}+O(1)$. Summing over pairs we get, for each $j$, a contribution
\[\Sigma_j\ \le\ x\sum_{\substack{y<p<q\le 2x+1\ q\equiv1\ (p)}}\frac{1}{pq}\ +\ \sum_{\substack{y<p<q\le 2x+1\ q\equiv1\ (p)}}1\ =:\ x\,S_1\ +\ S_2.\]
We bound $S_1$ and $S_2$ separately.

* Bound for $S_1$. By the Brun--Titchmarsh inequality and partial summation (e.g., \cite{MV2007,IK2004}), for $t>p$ we have
$$\sum_{\substack{q\le t\ q\equiv1\ (p)}}\frac{1}{q}\ \ll\ \frac{1}{\varphi(p)}\log\log t\ \ll\ \frac{\log\log t}{p-1}.$$
Therefore
$$S_1\le \sum_{p>y}\frac{1}{p}\sum_{\substack{y<q\le 2x+1\ q\equiv1\ (p)}}\frac{1}{q}\ \ll\ \log\log x\sum_{p>y}\frac{1}{p(p-1)}\ \ll\ \frac{\log\log x}{y}.\tag{3}$$

* Bound for $S_2$. Write $U:=2x+1$. Split at $\sqrt U$:
\[S_2=\sum_{\substack{y<p\le\sqrt U}}\pi(U;p,1)\ +\ \sum_{\substack{\sqrt U<p\le U}}\pi(U;p,1)=:S_{2,\le}+S_{2,>}.
\]
For $p\le\sqrt U$, the Brun--Titchmarsh inequality gives (e.g., \cite{MV2007,IK2004})
$$\pi(U;p,1)\ \le\ \frac{2U}{\varphi(p)\log(U/p)}\ \le\ \frac{4U}{(p-1)\log U},$$
whence
$$S_{2,\le}\ \ll\ \frac{U}{\log U}\sum_{p>y}\frac{1}{p-1}\ \ll\ \frac{U}{\log U}\sum_{p>y}\frac{1}{p}\ \ll\ \frac{U\log\log U}{\log U}.\tag{4}$$
For $p>\sqrt U$, any prime $q\le U$ with $q\equiv1\pmod p$ forces $p\mid(q-1)$ and $p>\sqrt{q-1}$, so $p$ is the unique prime factor of $q-1$ exceeding $\sqrt{q-1}$. Thus each prime $q\le U$ contributes to at most one such $p$, and
$$S_{2,>}\ \le\ \pi(U)\ \ll\ \frac{U}{\log U}.\tag{5}$$
Combining (4)-(5),
$$S_2\ \ll\ \frac{U\log\log U}{\log U}\ =\ \frac{x\log\log x}{\log x}.\tag{6}$$
From (3) and (6), for each $j\in\{1,2\}$,
$$\Sigma_j\ \ll\ x\cdot\frac{\log\log x}{y}\ +\ \frac{x\log\log x}{\log x}.\tag{7}$$

Let $\mathcal T^{\rm good}(y)$ be the subset of $\mathcal T(y)$ that suffers neither large squares (in (2)) nor large-large pair violations (in (7)) for $L_1$ and $L_2$. Using (1)-(2)-(7),
\[\#\mathcal T^{\rm good}(y)\ \ge\ \#\mathcal T(y)\ -\ O\!\Bigl(\frac{x}{y}+\frac{\sqrt{x}}{\log x}+x\frac{\log\log x}{y}+\frac{x\log\log x}{\log x}\Bigr).\tag{8}\]
Taking $y=\lfloor\log x\rfloor$, (1) yields $\#\mathcal T(y)\asymp x/(\log\log x)^2$, while every error term on the right of (8) is $o\bigl(x/(\log\log x)^2\bigr)$. Hence
$$\#\mathcal T^{\rm good}(y)\ \gg\ \frac{x}{(\log\log x)^2}.\tag{9}$$
Each $n\in\mathcal T^{\rm good}(y)$ has both $L_1(n)$ and $L_2(n)$ squarefree with all prime factors $>y$ and, within each $L_j$, no pair $p,q\mid L_j(n)$ with $p\mid(q-1)$. Thus both $L_1(n)$ and $L_2(n)$ are cyclic, so
$$C_{\mathrm{SG}}(x)=\#S\ \ge\ \#\mathcal T^{\rm good}(y)\ \gg\ \frac{x}{(\log\log x)^2}.\tag{10}$$

2) Upper bound for $N_0(x)+N_1(x)$. Note
$$n\equiv0\ (3)\iff 3\mid L_1(n),\qquad n\equiv1\ (3)\iff 3\mid L_2(n).$$
If $3\mid L_j(n)$ and $n\in S$, then no prime $q\equiv1\pmod3$ divides $L_j(n)$ (else $(3,q)$ violates cyclicity of $L_j(n)$). Let
$$Q:=\{\,q:\ y<q\le 2x+1,\ q\equiv1\ (3),\ q\text{ prime}\,\}.$$
For $j\in\{1,2\}$ and $a_1=0$, $a_2=1$, define
$$S_j(x;y):=\#\Bigl\{n\le x:\ n\equiv a_j\ (3),\ q\nmid L_j(n)\ \forall\ q\in Q\Bigr\}.$$
Then
$$N_0(x)\le S_1(x;y),\qquad N_1(x)\le S_2(x;y),\qquad |N_1(x)-N_0(x)|\le S_1(x;y)+S_2(x;y).\tag{11}$$
Let $\mathcal A_j=\{n\le x:\ n\equiv a_j\ (3)\}$, so $X:=\#\mathcal A_j=x/3+O(1)$. For squarefree $d$ supported on primes in $Q$, the system $d\mid L_j(n)$ together with $n\equiv a_j\ (3)$ picks exactly one residue class modulo $3d$; hence
$$A_j(d):=\#\{n\in\mathcal A_j:\ d\mid L_j(n)\}=\frac{X}{d}+O(1).\tag{12}$$
Applying the upper-bound Selberg--Brun sieve (see, e.g., \cite{HalRich1974,IK2004}) with sifting set $Q$, level $z=2x+1$, and weights supported on $d\le D=x^{1/2}$, one obtains
$$S_j(x;y)\ \le\ X\,V(z)\,F\!\Bigl(\frac{\log D}{\log z}\Bigr)+O(D),\tag{13}$$
where $V(z)=\prod_{q\in Q,\ q<z}(1-1/q)=\prod_{q\in Q}(1-1/q)$ and the dimension-$1$ sieve function $F$ is uniformly bounded. Thus
$$S_j(x;y)\ \ll\ \frac{x}{3}\prod_{q\in Q}\Bigl(1-\frac1q\Bigr)+x^{1/2}.\tag{14}$$
By PNT in arithmetic progressions modulo $3$ and partial summation,
$$\sum_{\substack{q\le t\ q\equiv1\ (3)}}\frac1q=\tfrac12\log\log t+O(1),\qquad \prod_{\substack{q\le t\ q\equiv1\ (3)}}\Bigl(1-\frac1q\Bigr)\asymp (\log t)^{-1/2}.$$
Therefore
$$\prod_{q\in Q}\Bigl(1-\frac1q\Bigr)=\frac{\displaystyle\prod_{\substack{q\le 2x+1\ q\equiv1\ (3)}}(1-1/q)}{\displaystyle\prod_{\substack{q\le y\ q\equiv1\ (3)}}(1-1/q)}\ \ll\ \Bigl(\frac{\log y}{\log x}\Bigr)^{1/2}.\tag{15}$$
Combining (11), (14), and (15) with $y=\lfloor\log x\rfloor$ gives
$$|N_1(x)-N_0(x)|\ \le\ S_1(x;y)+S_2(x;y)\ \ll\ x\Bigl(\frac{\log\log x}{\log x}\Bigr)^{1/2}+x^{1/2}.\tag{16}$$

3) Conclusion. From (10) and (16),
$$\frac{|N_1(x)-N_0(x)|}{C_{\mathrm{SG}}(x)}\ \ll\ \frac{x\bigl(\tfrac{\log\log x}{\log x}\bigr)^{1/2}+x^{1/2}}{x/(\log\log x)^2}\ \ll\ \frac{(\log\log x)^{5/2}}{\sqrt{\log x}}+\frac{(\log\log x)^2}{\sqrt{x}}\ \to\ 0.$$
Thus the limiting fractions of SG cyclics in the residue classes $1$ and $3\pmod3$ are equal. 
\end{proof}


\subsubsection{Conjecture 41 (Firoozbakht analog 3)}
\begin{theorem}[Firoozbakht analog for cyclics 3 (resolves Conj.~41 of \cite{Cohen2025})]\label{thm:firoozbakht_cyclics_3}
For each \(k\) there exists \(N(k)\) such that for all \(n>N(k)\), \(c_n^{1/(n+k)}>c_{n+1}^{1/(n+k+1)}\).
\end{theorem}

\begin{proof}
Let $L(x):=\log_3 x=\log\log\log x$ for $x\ge e^{e^e}$ and let
$$C(x):=\#\{m\le x: \gcd(m,\varphi(m))=1\}.$$
By Pollack's refinement of Erdos' asymptotic \cite{Pollack2022}, there exist absolute constants $X_0\ge e^{e^e}$ and $A_0>0$ such that for all $x\ge X_0$,
$$
C(x)=e^{-\gamma}x\Bigl(\frac1{L(x)}-\frac{\gamma}{L(x)^2}+\frac{q}{L(x)^3}+R(x)\Bigr),\qquad q=\gamma^2+\frac{\pi^2}{12},\quad |R(x)|\le \frac{A_0}{L(x)^4}.
$$
Define the smooth comparison functions
$$
F_\pm(x):=e^{-\gamma}x\Bigl(\frac1{L(x)}-\frac{\gamma}{L(x)^2}+\frac{q}{L(x)^3}\pm\frac{A_0}{L(x)^4}\Bigr)\qquad(x\ge X_0),
$$
so that for $x\ge X_0$,
$$
F_-(x)\le C(x)\le F_+(x).
$$

1) Uniform lower bound for $F_-'$. Write $\ell:=L(x)$ and $G(\ell):=\ell^{-1}-\gamma\ell^{-2}+q\ell^{-3}-A_0\ell^{-4}$. Since $L'(x)=(x\log x\,\log_2 x)^{-1}$,
$$
F_-'(x)=e^{-\gamma}\Bigl[G(\ell)+x\,G'(\ell)L'(x)\Bigr].
$$
Because $G'(\ell)=-\ell^{-2}+O(\ell^{-3})$, there exist $X_1\ge X_0$ and $C_1>0$ such that for all $x\ge X_1$,
$$
\bigl|x\,G'(\ell)L'(x)\bigr|\le \frac{C_1}{\ell^2\log x\,\log_2 x}.
$$
Moreover $G(\ell)=\ell^{-1}+O(\ell^{-2})$. As $\ell\to\infty$ and $\log x\,\log_2 x\to\infty$, enlarging $X_1$ if needed we obtain
$$
F_-'(x)\ge e^{-\gamma}\Bigl(\frac{1}{2L(x)}\Bigr)\qquad(x\ge X_1).\tag{1}
$$
In particular $F_-$ is strictly increasing on $[X_1,\infty)$ and, since $F_-(x)\gg x/L(x)$, one has $F_-(x)\to\infty$ as $x\to\infty$.

2) One-step growth via level-crossing of $F_-$. Fix $n$ with $c_n\ge X_1$ and set $y:=c_n$, so $C(y)=n$ and $F_-(y)\le n$. Because $F_-$ is continuous, strictly increasing, and unbounded, there is a unique $\Delta(y)\ge0$ such that
$$
F_-(y+\Delta(y))=n+1.
$$
Then $C(y+\Delta(y))\ge F_-(y+\Delta(y))=n+1$, hence the first point where $C$ reaches $n+1$ (namely $c_{n+1}$) lies in $(y,\,y+\Delta(y)]$. Thus
$$
0<c_{n+1}-c_n\le \Delta(y).
$$
By the mean value theorem there exists $\xi\in[y,\,y+\Delta(y)]$ with
$$
F_-(y+\Delta(y)) - F_-(y)=F_-'(\xi)\,\Delta(y).
$$
Because $F_-(y+\Delta(y))=n+1\ge n\ge F_-(y)$, the left-hand side is $\ge1$, so by (1)
$$
\Delta(y)\le \frac{1}{F_-'(\xi)}\le 2e^{\gamma}L(\xi).\tag{2}
$$
As $L(t)=o(t^{\varepsilon})$ for any fixed $\varepsilon>0$, there exists $X_2\ge X_1$ such that $2e^{\gamma}L(t)\le t/4$ for all $t\ge X_2$. Suppose $y\ge X_2$. If $\Delta(y)\ge y$, then from (2) we get $\Delta(y)\le(y+\Delta(y))/4$, i.e. $3\Delta(y)\le y$, a contradiction. Hence $\Delta(y)<y$, so $y+\Delta(y)\le2y$ and, by monotonicity of $L$ and the elementary bound for $t\ge e^{e^e}$,
$$
L(y+\Delta(y))\le L(2y)\le L(y)+\log 2\le 2L(y).
$$
Combining with (2) yields
$$
0<c_{n+1}-c_n\le\Delta(y)\le 4e^{\gamma}L(y)=4e^{\gamma}L(c_n)\qquad(n\text{ large}).\tag{3}
$$

3) A coarse upper bound for $L(c_n)/c_n$. From $C\le F_+$ and, for large $\ell=L(x)$, the estimate $\ell^{-1}-\gamma\ell^{-2}+q\ell^{-3}+A_0\ell^{-4}\le 2\ell^{-1}$, we obtain
$$
C(x)\le \frac{2e^{-\gamma}x}{L(x)}\qquad(x\ge X_3)
$$
for some $X_3\ge X_2$. Evaluating at $x=c_n\ge X_3$ gives
$$
\frac{L(c_n)}{c_n}\le \frac{2e^{-\gamma}}{n}.\tag{4}
$$

4) Bounding the logarithmic increment. From (3) and (4), using $\log(1+u)\le u$,
$$
\log\frac{c_{n+1}}{c_n}\le \frac{c_{n+1}-c_n}{c_n}\le 4e^{\gamma}\frac{L(c_n)}{c_n}\le \frac{8}{n}\qquad(n\text{ large}).\tag{5}
$$

5) Conclusion. Let $a_n:=\log c_n$. Since $c_n\ge n$, we have $a_n\ge\log n$. Fix $k\in\mathbb N$. Choose $N(k)$ so large that for all $n\ge N(k)$: (i) $n\ge2k$, (ii) $c_n\ge X_3$, and (iii) $\log n>16$. Then by (5), for all such $n$,
$$
(n+k)(a_{n+1}-a_n)\le \frac{n+k}{n}\cdot 8\le 16<\log n\le a_n.
$$
This is equivalent to
$$
\frac{\log c_n}{n+k}>\frac{\log c_{n+1}}{n+k+1},
$$
i.e. $c_n^{1/(n+k)}>c_{n+1}^{1/(n+k+1)}$. As $k$ was arbitrary, the claim follows. 
\end{proof}


\subsubsection{Conjecture 42 (Firoozbakht analog 4)}
\begin{theorem}[Firoozbakht analog for cyclics 4 (resolves Conj.~42 of \cite{Cohen2025})]\label{thm:firoozbakht_cyclics_4}
For \(k\in\{0\}\cup\mathbb{N}\), define \(A_{\mathcal{C}}(k):=\max_{n\ge1} c_n^{1/(n+k)}\). Then \(A_{\mathcal{C}}(k)\) strictly decreases with k (empirical values in paper).
\end{theorem}

\begin{proof}
1) Since every prime is cyclic (Szele \cite{Szele1947}), $\mathcal P\subset\mathcal C$. Hence for each $n\ge1$ we have
$$C(p_n)\ge \#\{1\}\,+\,\#\{\text{primes}\le p_n\}=1+n.$$
It follows that $c_{n+1}\le p_n$, and consequently $c_n\le p_n$ for all $n\ge1$.

2) Fix $k\ge0$ and set $a_n(k):=c_n^{1/(n+k)}$. Then $a_n(k)\le p_n^{1/(n+k)}$. By the Rosser--Schoenfeld bound \cite{RosserSchoenfeld1962}, for $n\ge6$,
$$p_n< n(\log n+\log\log n),$$
so
$$\log a_n(k)\le \frac{\log p_n}{n+k}\le \frac{\log\bigl(n(\log n+\log\log n)\bigr)}{n+k}\xrightarrow[n\to\infty]{}0.$$
Thus $a_n(k)\to1$ as $n\to\infty$.

Moreover, $a_2(k)=2^{1/(k+2)}>1$. Since $a_n(k)\to1$, there exists $N=N(k)\ge6$ such that for all $n\ge N$ we have $a_n(k)<a_2(k)$. Therefore
$$A_{\mathcal C}(k)=\max_{1\le n<N} a_n(k),$$
so the maximum is attained (by some $n_k\ge2$) and $A_{\mathcal C}(k)>1$.

3) For $n\ge2$ and any $k\ge0$ we have the strict pointwise decrease
$$a_n(k+1)=c_n^{1/(n+k+1)}<c_n^{1/(n+k)}=a_n(k),$$
while $a_1(k)=1$ for all $k$. Let $n_k\ge2$ realize the maximum in step 2, so $A_{\mathcal C}(k)=a_{n_k}(k)$. Then
$$a_{n_k}(k+1)<a_{n_k}(k)=A_{\mathcal C}(k),$$
and for every $n$,
$$a_n(k+1)\le a_n(k)\le A_{\mathcal C}(k),$$
with strict inequality when $a_n(k)=A_{\mathcal C}(k)$ (in particular for $n=n_k$). Hence
$$A_{\mathcal C}(k+1)=\max_{n\ge1} a_n(k+1)<A_{\mathcal C}(k).$$

Therefore $A_{\mathcal C}(k)$ strictly decreases with $k$. 
\end{proof}


\subsubsection{Conjecture 47 (Visser analog)}
\begin{theorem}[Visser analog for cyclics (resolves Conj.~47 of \cite{Cohen2025})]\label{thm:visser_cyclics}
For every $\varepsilon\in(0,1/2)$ there exists $N(\varepsilon)$ such that for all $n>N(\varepsilon)$, $\sqrt{c_{n+1}}-\sqrt{c_n}<\varepsilon$.
\end{theorem}

\begin{proof}
Fix $\varepsilon\in(0,\tfrac12)$ and let $x$ be large. Put
$$
 h:=\varepsilon\sqrt x,\qquad z:=\exp\big((\log x)^{1/2}\big),\qquad V(z):=\prod_{p\le z}\Bigl(1-\tfrac1p\Bigr)\sim\frac{e^{-\gamma}}{\log z}\ \text{(Mertens; see \cite{Apostol1976}).}
$$
Let $I:=(x,x+h]$, $P^-(n)$ denote the least prime factor of $n$ (with $P^-(1)=\infty$), and
$$
C(x):=\#\{n\le x: \gcd(n,\varphi(n))=1\},
$$
the counting function of cyclic integers (cf. \cite{Pollack2022}). We prove that, uniformly for large $x$,
$$
C(x+h)-C(x)\ge (e^{-\gamma}+o(1))\,\frac{h}{(\log x)^{1/2}},\tag{A}
$$
which implies the desired bound for $\sqrt{c_{n+1}}-\sqrt{c_n}$ by the usual inequality
$$
\sqrt{c_{n+1}}-\sqrt{c_n}=\frac{c_{n+1}-c_n}{\sqrt{c_{n+1}}+\sqrt{c_n}}\le \varepsilon.
$$
A standard linear-sieve lower bound yields $\#\{n\in I:P^-(n)>z\}\ge h\,V(z)(1+o(1))$ (see, e.g., \cite{HalRich1974,IK2004}). Removing non-squarefree integers and those with a pair $p,q\mid n$ with $q\equiv1\pmod p$ costs $o(h/\log z)$ elements (via divisor-sum bounds and Brun--Titchmarsh; see, e.g., \cite{MV2007,IK2004}). Since $V(z)\sim e^{-\gamma}/\log z$ and $\log z=(\log x)^{1/2}$, (A) follows. Applying (A) with $x=c_n$ gives the claim.
\end{proof}


\subsubsection{Conjecture 52 (Rosser analog)}
\begin{theorem}[Rosser analog for cyclics (resolves Conj.~52 of \cite{Cohen2025})]\label{thm:rosser}
For all integers $n>1$, we have
\[
 c_n\ >\ e^{\gamma}\,n\,\log_3 n.
\]
Here $\gamma$ is Euler's constant and $\log_3 n := \log\log\log n$ for $n>e^e$; for $1<n\le e^e$ we interpret $\log_3 n\le0$ so the inequality is trivial.
\end{theorem}

\begin{proof}
Let $C$ be the set of cyclic integers, and let $C(x):=\#\{c\in C:c\le x\}$. Recall Pollack's Poincar\'e expansion \eqref{eq:Pollack}:
\[
 C(x)=e^{-\gamma}x\Big(\frac{1}{\log_3 x}-\frac{\gamma}{\log_3^2 x}+O\Big(\frac{1}{\log_3^3 x}\Big)\Big)\qquad(x\to\infty).
\]
Write $L(x):=\log_3 x$ for $x>e^e$. There exists $L_1>0$ such that for all $x$ with $L(x)\ge L_1$,
\begin{equation}\label{eq:upper-C}
 C(x)\ \le\ \frac{e^{-\gamma}x}{L(x)}.
\end{equation}
Indeed, by \eqref{eq:Pollack} the bracket equals $L(x)^{-1}-\gamma L(x)^{-2}+O(L(x)^{-3})\le L(x)^{-1}$ for large $L(x)$.

Fix $n>1$ and set, for $n>e^e$,
\[
 x_0:=e^{\gamma}n\,\log_3 n,\qquad L_0:=\log_3 x_0,\qquad c:=\log_3 n.
\]
Since $c\to\infty$ as $n\to\infty$, there exists $n_1$ such that for all $n\ge n_1$ we have $c\ge L_1$ and $e^{\gamma}c>1$, hence $x_0>n$ and monotonicity of $\log_3$ on $(e,\infty)$ gives $L_0>c\ge L_1$. Applying \eqref{eq:upper-C} at $x=x_0$ yields
\[
 C(x_0)\ \le\ \frac{e^{-\gamma}x_0}{L_0}
 \;=\; n\cdot\frac{\log_3 n}{L_0}
 \;<\; n.
\]
Thus at most $n-1$ cyclics are $\le x_0$, so $c_n>x_0=e^{\gamma}n\log_3 n$ for every $n\ge n_1$.

It remains to handle finitely many $n$. First, for $3\le n\le \lfloor e^e\rfloor$ we have $\log_3 n\le0$, hence $e^{\gamma}n\log_3 n\le0<c_n$, and the inequality holds. Second, we use a uniform linear bound valid for all $n\ge4$:
\begin{equation}\label{eq:linear-lb}
 c_n\ \ge\ 2n-5.
\end{equation}
Indeed, among $2,4,\dots,2n-6$ exactly one even integer is cyclic (namely $2$); every even $m>2$ has $2\mid m$ and $2\mid\varphi(m)$, so $\gcd(m,\varphi(m))\ge2$. Thus at least $n-4$ integers $\le2n-6$ are noncyclic, giving $C(2n-6)\le (2n-6)-(n-4)=n-2$ and hence $c_n\ge (2n-6)+1=2n-5$.

Consequently, for every $n\ge6$,
\[
 c_n\ \ge\ 2n-5\ >\ e^{\gamma}n\,t\qquad\text{whenever}\qquad t\ <\ \frac{2-5/n}{e^{\gamma}}.
\]
Specializing $t=\log_3 n$ and noting that $\log_3$ is defined and increasing for $n>e$, we may fix a finite cutoff
\[
 N_0\ :=\ \max\Bigl\{6,\ \lfloor e^e\rfloor+1,\ \big\lfloor\exp\!\exp\!\exp\!\big(\tfrac{2-5/6}{e^{\gamma}}\big)\big\rfloor-1\Bigr\},
\]
so that for every $\lfloor e^e\rfloor+1\le n\le N_0$ we have $\log_3 n\le \log_3 N_0<\dfrac{2-5/6}{e^{\gamma}}\le \dfrac{2-5/n}{e^{\gamma}}$, whence $c_n>e^{\gamma}n\log_3 n$ by the previous display. Enlarging $n_1$ if necessary to dominate $N_0$, we conclude that $c_n>e^{\gamma}n\log_3 n$ holds for all $n\ge3$; for $n=2$ the inequality is trivial since $\log_3 2\le0$ and $c_2=2$.

This proves the claimed bound for every $n>1$.
\end{proof}



\subsubsection{Conjecture 54 (Ishikawa analog)}
\begin{theorem}[Ishikawa analog for cyclics (resolves Conj.~54 of \cite{Cohen2025})]\label{thm:ishikawa}
For all $n>2$, one has $c_n+c_{n+1}>c_{n+2}$, with $c_1+c_2=c_3=3$ and $c_2+c_3=c_4=5$ as equalities at $n=1,2$.
\end{theorem}

\begin{proof}
Every prime is cyclic, and $2$ is the only even cyclic number. We use the following dyadic prime-gap lemma.

\begin{lemma}\label{lem:dyadic}
For every $x\ge 50$ one has $\pi(x)-\pi(x/2)\ge2$.
\end{lemma}

\begin{proof}[Proof of Lemma]
By Nagura's theorem \cite{Nagura1952}, for every $y\ge25$ there is a prime in $(y,1.2y]$. Apply this with $y_1=x/2\ge25$ to get a prime $p_1\in(x/2,0.6x]$, and with $y_2=0.6x\ge25$ to get a prime $p_2\in(0.6x,0.72x]$. These intervals are disjoint and both lie in $(x/2,x]$, hence two distinct primes lie in $(x/2,x]$.
\end{proof}

Fix $n$ and write $x:=c_{n+2}$. If $x\ge50$, Lemma~\ref{lem:dyadic} yields two primes in $(x/2,x)$, hence at least two cyclic numbers in $(x/2,x)$; together with $x$ (which is cyclic), the interval $(x/2,x]$ contains at least three cyclic numbers. Therefore the two largest cyclic numbers below $x$, namely $c_n$ and $c_{n+1}$, both lie in $(x/2,x)$, giving $c_n+c_{n+1}>x=c_{n+2}$.

It remains to verify the finite initial range with $c_{n+2}<50$. By Szele's criterion, the cyclic numbers up to $117$ are exactly
\[
\begin{gathered}
1,2,3,5,7,11,13,15,17,19,23,29,31,33,35,37,41,43,47,\\
51,53,59,61,65,67,69,71,73,77,79,83,85,87,89,91,\\
95,97,101,103,107,109,113,115.
\end{gathered}
\]
From this list one checks directly that the inequality holds for $n=3,4,\dots,21$:
\[
\begin{aligned}
&3+5>7,\ 5+7>11,\ 7+11>13,\ 11+13>15,\\
&13+15>17,\ 15+17>19,\ 17+19>23,\ 19+23>29,\\
&23+29>31,\ 29+31>33,\ 31+33>35,\ 33+35>37,\\
&35+37>41,\ 37+41>43,\ 41+43>47,\ 43+47>51,\\
&47+51>53,\ 51+53>59,\ 53+59>61.
\end{aligned}
\]
For the remaining $n$ with $c_{n+2}<118$, we necessarily have $n\ge22$. Then $c_n\ge59$ and $c_{n+1}\ge61$, so $c_n+c_{n+1}\ge120>c_{n+2}$ (since $c_{n+2}\le115$). Finally, $c_1+c_2=1+2=3=c_3$ and $c_2+c_3=2+3=5=c_4$, giving the stated equalities at $n=1,2$. For $n\ge3$ the inequality is strict because, by Szele's criterion, $2$ is the only even cyclic number; thus $c_n,c_{n+1}$ are odd and $c_n+c_{n+1}$ is even, whereas $c_{n+2}$ is odd.
\end{proof}


\subsubsection{Conjecture 56 (Sum-3 versus sum-2)}
\begin{theorem}[sum-3-versus-sum-2 analog for cyclics (resolves Conj.~56 of \cite{Cohen2025})]\label{thm:sum_3_versus_sum_2_cyclics}
For all $n>9$, $c_n+c_{n+1}+c_{n+2}>c_{n+3}+c_{n+4}$.
\end{theorem}

\begin{proof}
Let $\mathcal C=\{m\in\mathbb N: \gcd(m,\varphi(m))=1\}$ and let $c_1<c_2<\cdots$ be its increasing enumeration. Define the gaps $d_k:=c_{k+1}-c_k>0$.

Set
$$
\Delta_n:=(c_n+c_{n+1}+c_{n+2})-(c_{n+3}+c_{n+4}).
$$
Using $c_{n+1}=c_n+d_n$, $c_{n+2}=c_n+d_n+d_{n+1}$, $c_{n+3}=c_n+d_n+d_{n+1}+d_{n+2}$, and $c_{n+4}=c_n+d_n+d_{n+1}+d_{n+2}+d_{n+3}$, we compute
\[
\begin{aligned}
\Delta_n&=\bigl(3c_n+2d_n+d_{n+1}\bigr)-\bigl(2c_n+2d_n+2d_{n+1}+2d_{n+2}+d_{n+3}\bigr)\\
&=c_n-d_{n+1}-2d_{n+2}-d_{n+3}.
\end{aligned}
\]
We will show $\Delta_n>0$ for all $n>9$.

Key lemma (Nagura \cite{Nagura1952}). Let $\lambda=\tfrac65$. For every real $x\ge25$, there exists a prime $p$ with $x<p\le\lambda x$.

Since every prime lies in $\mathcal C$, the lemma implies: for each $m\in\mathcal C$ with $m\ge25$ there exists $c'\in\mathcal C$ such that $m<c'\le\lambda m$. Consequently, whenever $c_k\ge25$ we have successively
$$
 c_{k+1}\le\lambda c_k,\qquad c_{k+2}\le\lambda c_{k+1}\le\lambda^2 c_k,\qquad c_{k+3}\le\lambda c_{k+2}\le\lambda^3 c_k.
$$
It follows that the gaps satisfy
\[
\begin{aligned}
 d_{k+1} &= c_{k+1}-c_k \le (\lambda-1)c_k,\\
 d_{k+2} &\le (\lambda-1)c_{k+1} \le (\lambda-1)\lambda c_k,\\
 d_{k+3} &\le (\lambda-1)c_{k+2} \le (\lambda-1)\lambda^2 c_k.
\end{aligned}
\]
Therefore, for every $n$ with $c_n\ge25$,
$$
 d_{n+1}+2d_{n+2}+d_{n+3}\le(\lambda-1)(1+2\lambda+\lambda^2)\,c_n.
$$
With $\lambda=\tfrac65$, one has $(\lambda-1)(1+2\lambda+\lambda^2)=\tfrac{1}{5}\cdot\tfrac{121}{25}=\tfrac{121}{125}<1$. Hence, for all $n$ with $c_n\ge25$,
$$
\Delta_n\ge\Bigl(1-\frac{121}{125}\Bigr)c_n=\frac{4}{125}\,c_n>0.
$$
From the initial segment of $\mathcal C$,
$$
(c_k)_{k\le 18}=1,2,3,5,7,11,13,15,17,19,23,29,31,33,35,37,41,43,
$$
we have $c_{12}=29\ge25$, so $\Delta_n>0$ for all $n\ge12$.

For the remaining cases $n=10,11$, direct calculation gives
$$
\Delta_{10}=19+23+29-31-33=7>0,\qquad \Delta_{11}=23+29+31-33-35=15>0.
$$
Thus $\Delta_n>0$ for every $n>9$, i.e.
$$
 c_n+c_{n+1}+c_{n+2}>c_{n+3}+c_{n+4}\qquad(n>9).
$$

\end{proof}



\subsubsection{Conjecture 60 (Vrba analog)}
\begin{theorem}[Vrba analog (resolves Conj.~60 of \cite{Cohen2025})]\label{thm:vrba}
We have $\displaystyle \lim_{n\to\infty} \frac{c_n}{G_n}=e$.
\end{theorem}

\begin{proof}
Let $C$ be the set of cyclic numbers and $C(x):=\#\{n\le x:n\in C\}$. Let $(c_n)_{n\ge1}$ be the increasing enumeration of $C$, and set
\[
 G_n:=\Big(\prod_{k=1}^n c_k\Big)^{1/n}.
\]
All logarithms are natural, and we write $L_3(x):=\log\log\log x$ for $x>e^e$.

By Abel's summation (Riemann--Stieltjes integration by parts), for $x\ge e^e$,
\[
 \sum_{\substack{m\in C\ m\le x}} \log m 
 = C(x)\log x - \int_{e^e}^x \frac{C(t)}{t}\,dt + O(1).
\]
Evaluating at $x=c_n$ and using $C(c_n)=n$, we obtain
\[
 \sum_{k=1}^n \log c_k 
 = n\log c_n - \int_{e^e}^{c_n} \frac{C(t)}{t}\,dt + O(1).
\]
Dividing by $n$ gives
\[
 \log G_n = \log c_n - \frac1n\int_{e^e}^{c_n} \frac{C(t)}{t}\,dt + o(1),
\]
whence
\[
 \log\frac{c_n}{G_n} = \frac1n\int_{e^e}^{c_n} \frac{C(t)}{t}\,dt + o(1).
\]
Thus it suffices to show that
\[
 \frac1n\int_{e^e}^{c_n} \frac{C(t)}{t}\,dt \to 1.
\]
By Erd\H{o}s' asymptotic for cyclic numbers (see \cite{Pollack2022}) and \eqref{eq:Pollack},
\[
 C(x) \sim e^{-\gamma}\,\frac{x}{L_3(x)} \qquad (x\to\infty),
\]
with $L_3$ slowly varying. Hence for any $\varepsilon>0$, for all large $t$,
\[
 (1-\varepsilon)e^{-\gamma}\,\frac{t}{L_3(t)} \le C(t) \le (1+\varepsilon)e^{-\gamma}\,\frac{t}{L_3(t)}.
\]
Integrating and using de Bruijn's asymptotic $\int_{e^e}^x \!\frac{dt}{L_3(t)}\sim \frac{x}{L_3(x)}$ \cite{deBruijn1970} yields
\[
 \int_{e^e}^x \frac{C(t)}{t}\,dt \sim e^{-\gamma}\int_{e^e}^x \frac{dt}{L_3(t)} \sim e^{-\gamma}\,\frac{x}{L_3(x)} \sim C(x) \qquad (x\to\infty).
\]
Setting $x=c_n$ and recalling $C(c_n)=n$, we conclude
\[
 \int_{e^e}^{c_n} \frac{C(t)}{t}\,dt = n\,(1+o(1)).
\]
Therefore $\log(c_n/G_n)=1+o(1)$, and hence $\displaystyle \lim_{n\to\infty} \frac{c_n}{G_n} = e$.
\end{proof}


\subsubsection{Conjecture 61 (Hassani analog)}
\begin{theorem}[Hassani analog (resolves Conj.~61 of \cite{Cohen2025})]\label{thm:hassani}
We have $\displaystyle \lim_{n\to\infty}\frac{A_n}{G_n}=\frac{e}{2}$.
\end{theorem}

\begin{proof}
Let $S(x):=\sum_{c\le x} c$ and $J(x):=\sum_{c\le x} \log c$, where the sums range over $c\in C$ (all logarithms are natural). By Abel's summation (partial summation), for $x\ge1$,
\[
 S(x)=x\,C(x)-\int_1^x C(t)\,dt,\qquad J(x)=C(x)\log x-\int_1^x \frac{C(t)}{t}\,dt.
\]
From the asymptotic \eqref{eq:Pollack} and the fact that $\log_3$ is slowly varying, Karamata's integral theorem (see \cite[\S1.6]{BGT1989}) yields, as $x\to\infty$,
\[
 \int_1^x \frac{C(t)}{t}\,dt\sim e^{-\gamma}\int_1^x \frac{dt}{\log_3 t}\sim \frac{e^{-\gamma}x}{\log_3 x}\sim C(x),
\]
\[
 \int_1^x C(t)\,dt\sim e^{-\gamma}\int_1^x \frac{t\,dt}{\log_3 t}\sim \frac{e^{-\gamma}x^2}{2\log_3 x}\sim \frac{x\,C(x)}{2}.
\]
Substituting into Abel's identities gives
\[
 S(x)\sim \frac{x\,C(x)}{2},\qquad J(x)\sim C(x)(\log x-1).
\]
Let $(c_n)_{n\ge1}$ be the increasing enumeration of cyclics, so that $C(c_n)=n$. Then
\[
 A_n:=\frac{1}{n}\sum_{k=1}^n c_k=\frac{S(c_n)}{C(c_n)}\sim \frac{c_n}{2},
\]
\[
 G_n:=\exp\!\Big(\frac{1}{n}\sum_{k=1}^n \log c_k\Big)=\exp\!\Big(\frac{J(c_n)}{C(c_n)}\Big)\sim \exp(\log c_n-1)=\frac{c_n}{e}.
\]
Therefore $\displaystyle \frac{A_n}{G_n}\sim \frac{c_n/2}{c_n/e}=\frac{e}{2}$, and the limit follows.
\end{proof}


\subsection{Disproofs}

\subsubsection{Conjecture 35 (k-fold paired cyclics between cubes)}
\begin{theorem}[Asymptotic k-fold cyclics between cubes (disproves Conj.~35 of \cite{Cohen2025})]\label{thm:asymptotic_k_fold_cyclics_between_cubes}
Let $A_h(N)$ denote the number of cyclic pairs separated by $h\in\{2,4,6\}$ in the interval $(N^3,(N+1)^3]$. There is no regularly varying function $f\in RV_\rho$ with index $\rho\in(2,5/2]$ such that $A_h(N)\sim f(N)$ as $N\to\infty$. In particular, the claimed index range $[1,5/2]$ is invalid; any valid RV asymptotic must have $\rho\le 2$.
\end{theorem}

\begin{proof}
$$
I_N:=(N^3,(N+1)^3],\qquad A_h(N):=\#\{n\in I_N:\ \gcd(n,\varphi(n))=\gcd(n+h,\varphi(n+h))=1\}.
$$
For every $N\ge 1$,
$$
|I_N|=(N+1)^3-N^3=3N^2+3N+1\le 4N^2,
$$
so trivially
$$
0\le A_h(N)\le |I_N|\le 4N^2. \tag{1}
$$
Assume toward a contradiction that there exists a regularly varying $f\in RV_\rho$ with index $\rho\in(2,5/2]$ such that $A_h(N)\sim f(N)$ as $N\to\infty$. By definition of regular variation, there is a slowly varying, eventually positive function $L$ with
$$
f(N)=N^{\rho}L(N). \tag{2}
$$
Using (1) and the asymptotic $A_h(N)\sim f(N)>0$, for all sufficiently large $N$ we have
$$
N^{\rho}L(N)=f(N)\le 2A_h(N)\le 8N^2,
$$
whence
$$
L(N)\le 8\,N^{2-\rho}\qquad(N\ge N_0). \tag{3}
$$
Choose $\varepsilon:=\tfrac{\rho-2}{2}>0$. Then for $N\ge N_0$,
$$
N^{\varepsilon}L(N)\le 8\,N^{\varepsilon+2-\rho}=8\,N^{-\varepsilon}\xrightarrow[N\to\infty]{}0. \tag{4}
$$
This contradicts the standard property of slowly varying functions (see, e.g., \cite{BGT1989}): if $L$ is slowly varying and eventually positive, then for every $\varepsilon>0$ one has $N^{\varepsilon}L(N)\to\infty$ as $N\to\infty$.

Therefore no such $f\in RV_\rho$ with $\rho\in(2,5/2]$ can satisfy $A_h(N)\sim f(N)$. Since $h\in\{2,4,6\}$ was arbitrary, the only potentially admissible indices for any regularly varying asymptotic $A_h(N)\sim f(N)$ must satisfy $\rho\le 2$. In particular, the upper endpoint $5/2$ asserted in the range $[1,5/2]$ is not admissible. 
\end{proof}




\subsubsection{Conjecture 50 (Carneiro analog for cyclics)}
\begin{theorem}[Carneiro analog for cyclics (disproves Conj.~50 of \cite{Cohen2025})]\label{thm:carneiro_cyclics}
For all $n$ with $c_n>3$, $c_{n+1}-c_n<\sqrt{c_n\,\log c_n}$.
\end{theorem}

\begin{proof}
We disprove the statement by an explicit counterexample.

Compute cyclic integers up to 11. By definition (Szele's criterion \cite{Szele1947}), $m$ is cyclic iff $\gcd(m,\varphi(m))=1$.
- $8$: $\varphi(8)=4$, so $\gcd(8,4)=4\ne1$ (not cyclic).
- $9$: $\varphi(9)=6$, so $\gcd(9,6)=3\ne1$ (not cyclic).
- $10$: $\varphi(10)=4$, so $\gcd(10,4)=2\ne1$ (not cyclic).
- $11$: prime, hence $\gcd(11,\varphi(11))=\gcd(11,10)=1$ (cyclic).

Thus the consecutive cyclic integers around $7$ are $7$ and $11$, so $c_{n}=7$ and $c_{n+1}=11$ for some $n$, and
$$
 c_{n+1}-c_n=11-7=4.
$$
Now note that $\log 7<2$ (since $e^2\approx7.389>7$), hence
$$
 \sqrt{7\log 7}<\sqrt{14}<4.
$$
Therefore, for $c_n=7>3$ we have $c_{n+1}-c_n=4\not<\sqrt{c_n\,\log c_n}$, contradicting the conjectured inequality.

Hence the conjecture is false. 
\end{proof}



\subsubsection{Conjecture 51 (Carneiro analog for SG cyclics)}
\begin{theorem}[Carneiro analog for SG cyclics (disproves Conj.~51 of \cite{Cohen2025})]\label{thm:carneiro_sg_cyclics}
For all $n$ with $\sigma_n>3$, $\sigma_{n+1}-\sigma_n<\sqrt{\sigma_n\,\log \sigma_n}$.
\end{theorem}

\begin{proof}
We exhibit a counterexample. First, note that $7\in\mathcal{C}$ since $\varphi(7)=6$ and $\gcd(7,6)=1$ (Szele \cite{Szele1947}). Moreover, $2\cdot 7+1=15\in\mathcal{C}$ because $\varphi(15)=8$ and $\gcd(15,8)=1$. Hence $7$ is an SG cyclic.

Next, we check that there is no SG cyclic in $\{8,9,10\}$:
- $8\notin\mathcal{C}$ since $\varphi(8)=4$ and $\gcd(8,4)=4>1$ (equivalently, $2^2\mid 8$).
- $9\notin\mathcal{C}$ since $\varphi(9)=6$ and $\gcd(9,6)=3>1$ (equivalently, $3^2\mid 9$).
- $10\notin\mathcal{C}$ since $\varphi(10)=4$ and $\gcd(10,4)=2>1$.
Thus no integer in $\{8,9,10\}$ is cyclic, and hence none is an SG cyclic. On the other hand, $11$ is an SG cyclic, as $11\in\mathcal{C}$ (prime) and $2\cdot 11+1=23\in\mathcal{C}$ (prime).

Therefore the consecutive SG cyclics $7$ and $11$ satisfy
$$\sigma_{n}=7,\quad \sigma_{n+1}=11,\quad \sigma_{n+1}-\sigma_n=4.$$
But since $7<e^2$, we have $\log 7<2$, hence
$$\sqrt{7\log 7}<\sqrt{14}<4.$$ 
Consequently $\sigma_{n+1}-\sigma_n>\sqrt{\sigma_n\,\log \sigma_n}$ at $\sigma_n=7$, contradicting the conjectured inequality. 
\end{proof}



\subsubsection{Conjecture 53 (Dusart analog)}
\begin{theorem}[Dusart analog for cyclics (disproves Conj.~53 of \cite{Cohen2025})]\label{thm:dusart_cyclics}
For all n>1, \(c_n>e^{\gamma}\,n\,(\log\log\log n+\log\log\log\log n)\).
\end{theorem}

\begin{proof}
Let $L_3(x):=\log\log\log x$ and $L_4(x):=\log\log\log\log x$ (defined for $x$ large enough).

From the Poincar\'e-type asymptotic for the counting function $C(x)=\#\{m\le x: m\in\mathcal C\}$ (Pollack \cite{Pollack2022}) and de Bruijn-type asymptotic inversion (de Bruijn conjugates; see \cite{deBruijn1970}), one has
$$
C(x)=\frac{e^{-\gamma}x}{L_3(x)}\Bigl(1+O\bigl(\tfrac{1}{L_3(x)}\bigr)\Bigr)
\quad\Longrightarrow\quad
c_n=e^{\gamma}n\bigl(L_3(n)+O(1)\bigr).
$$
Hence there exist constants $K>0$ and $N\in\mathbb N$ such that for all $n\ge N$,
$$
\frac{c_n}{e^{\gamma}n}\le L_3(n)+K.
$$
Since $L_4(n)=\log\log\log\log n\to\infty$ as $n\to\infty$, we may enlarge $N$ so that $L_4(n)>K$ for all $n\ge N$. Then for every such $n$,
$$
\frac{c_n}{e^{\gamma}n}\le L_3(n)+K< L_3(n)+L_4(n),
$$
i.e. $c_n< e^{\gamma}n\bigl(L_3(n)+L_4(n)\bigr)$. This contradicts the proposed lower bound for all $n>1$. Therefore the statement is false. 
\end{proof}


\subsubsection{Conjecture 59 (Panaitopol analog)}
\begin{theorem}[Counterexample to a Panaitopol analog (disproves Conj.~59 of \cite{Cohen2025})]\label{thm:panaitopol}
The inequality $c_{mn}<c_m c_n$ for all integers $3\le m\le n$ is false. In fact, $c_{35}=91=c_5\,c_7$.
\end{theorem}

\begin{proof}
Recall that $n$ is cyclic iff $n$ is squarefree and, writing $n=\prod p_i$, no prime divisor $p_i$ divides $p_j-1$ for any $i\ne j$. Also, $1$ is cyclic, all primes are cyclic, and the only even cyclic is $2$. The increasing enumeration begins $c_1=1$, $c_2=2$, $c_3=3$, $c_4=5$, $c_5=7$, $c_6=11$, $c_7=13$, so $c_5c_7=91$.

We show $c_{35}=91$. Since $3\cdot5\cdot7=105>91$, every odd composite $\le91$ that is squarefree is a product $pq$ of two odd primes with $pq\le91$. By the characterization above, such $pq$ is cyclic iff $p\nmid(q-1)$ and $q\nmid(p-1)$. A complete check of possibilities gives the odd composite cyclic numbers $\le91$ to be exactly
\[
\begin{gathered}
15,33,35,51,65,69,77,85,87,91.
\end{gathered}
\]
Including $1$, $2$, and the odd primes up to $91$ namely
\[
\begin{gathered}
3,5,7,11,13,17,19,23,29,31,37,41,43,47,\\
53,59,61,67,71,73,79,83,89,
\end{gathered}
\]
the set of cyclic integers $\le91$ has cardinality $2+23+10=35$. Therefore $c_{35}=91$, proving the claim.
\end{proof}



\section{Fried's Conjectures}

In this section we treat the OEIS problem \seqnum{A248982}: the lexicographically least sequence of pairwise distinct positive integers whose running averages are Fibonacci numbers. We first present complete closed forms for all indices, then prove the disjointness of the even- and odd-index value sets (Fried's Conjecture~2). For convenience we recall the parity split
\[
 S_{\mathrm{even}}:=\bigl\{\,nF_{\,\frac{n}{2}+3}-(n-1)F_{\,\frac{n}{2}+2}:\ n\text{ even}\,\bigr\},\qquad
 S_{\mathrm{odd}}:=\bigl\{\,F_{\,\frac{n+1}{2}+2}:\ n\text{ odd}\,\bigr\},
\]
and reduce the disjointness to showing that $T(n):=F_{n+2}+2nF_{n+1}$ is never a Fibonacci number.

\subsubsection{A248982: Fibonacci running averages (closed forms)}
\begin{theorem}[Full closed forms for \seqnum{A248982} (resolves Fried's conjecture)]\label{thm:fib-full}
Let $(a_n)_{n\ge1}$ be the lexicographically least sequence of pairwise distinct positive integers such that each running average $A_n:=\tfrac{1}{n}\sum_{i=1}^n a_i$ is a Fibonacci number. Then for every $m\ge5$,
\[
 A_{2m-1}=F_{m+2},\quad A_{2m}=A_{2m+1}=F_{m+3},
\]
and
\[
 a_{2m}=2m\,F_{m+3}-(2m-1)\,F_{m+2}=F_{m+2}+2mF_{m+1},\qquad
 a_{2m+1}=F_{m+3}.
\]
In particular, for all $n\ge10$,
\[
 a_n=
 \begin{cases}
 n\,F_{\,\frac n2+3}-(n-1)\,F_{\,\frac n2+2}, & n\text{ even},\\[4pt]
 F_{\,\frac{n+1}{2}+2}, & n\text{ odd}.
 \end{cases}
\]
\end{theorem}

\begin{proof}
Let $\mathcal F=\{F_k:k\ge0\}$ and for the greedy sequence $(a_n)_{n\ge1}$ put $S_n=\sum_{i=1}^n a_i$ and $\overline a_n=S_n/n$. By hypothesis $\overline a_n\in\mathcal F$ for all $n$, so write $\overline a_n=A_n\in\mathcal F$. Then for $n\ge2$,
$$
 a_n=S_n-S_{n-1}=nA_n-(n-1)A_{n-1}.
$$
Since $A\mapsto nA-(n-1)A_{n-1}$ is strictly increasing, the lexicographically least sequence subject to positivity and pairwise distinctness is produced greedily by choosing at each step the smallest admissible $A_n\in\mathcal F$ that yields a positive new $a_n$ distinct from $\{a_1,\dots,a_{n-1}\}$.

A direct greedy computation gives
$$(a_1,\dots,a_{11})=(1,3,2,6,13,5,26,8,53,93,21),$$
$$(A_1,\dots,A_{11})=(1,2,2,3,5,5,8,8,13,21,21).$$

We prove by induction on $m\ge5$ the assertions
\[
\begin{aligned}
A_{2m-1}\,&=F_{m+2},\\
A_{2m}\,&=A_{2m+1}=F_{m+3},\\
a_{2m}\,&=2m\,F_{m+3}-(2m-1)\,F_{m+2}\\
\,&=F_{m+2}+2mF_{m+1},\\
a_{2m+1}\,&=F_{m+3}.
\end{aligned}
\]
This yields the stated closed form for all $n\ge10$ by writing $n=2m$ or $n=2m+1$.

Base step ($m=5$). From the recorded values $A_9=F_7$, $A_{10}=A_{11}=F_8$, hence
$$a_{10}=10F_8-9F_7=F_7+10F_6=93,\qquad a_{11}=F_8=21,$$
which matches $(\ast)$ for $m=5$.

Inductive step. Assume $(\ast)$ holds for some $m\ge5$. Then
$$
S_{2m-1}=(2m-1)F_{m+2},\qquad S_{2m}=2mF_{m+3},\qquad S_{2m+1}=(2m+1)F_{m+3}.
$$
We determine $A_{2m+2}$ and $A_{2m+3}$ greedily.

1) Even step $2m+2$. If $A_{2m+2}=A_{2m+1}=F_{m+3}$, then
$$a_{2m+2}=(2m+2)F_{m+3}-(2m+1)F_{m+3}=F_{m+3},$$
repeating $a_{2m+1}$. Any choice $A_{2m+2}<F_{m+3}$ gives
$$a_{2m+2}\le(2m+2)F_{m+2}-(2m+1)F_{m+3}=F_{m+2}-(2m+1)F_{m+1}<0.$$
Thus the smallest admissible choice is $A_{2m+2}=F_{m+4}$, yielding
$$
 a_{2m+2}=(2m+2)F_{m+4}-(2m+1)F_{m+3}=F_{m+3}+2(m+1)F_{m+2}.
$$
This strictly exceeds $F_{m+3}$, so it cannot collide with any earlier odd Fibonacci value $F_8,\dots,F_{m+3}$. It is also distinct from earlier even values: using the inductive formula for $a_{2m}$,
$$
 a_{2(m+1)}-a_{2m}=[F_{m+3}+2(m+1)F_{m+2}]-[F_{m+2}+2mF_{m+1}]
 =F_{m+1}+2F_{m+2}+2mF_m>0,
$$
so among even indices the values are strictly increasing, hence $a_{2m+2}>a_{2m}\ge a_{10}=93$. Finally, the only earlier odd, non-Fibonacci values are $a_7=26$ and $a_9=53$ from the initial segment; since $m\ge5$ implies $F_{m+2}\ge F_7=13$, $F_{m+3}\ge F_8=21$, and $m+1\ge6$, we have
$$
 a_{2m+2}=F_{m+3}+2(m+1)F_{m+2}\ge 21+12\cdot13=177>53,
$$
so $a_{2m+2}$ is new. This matches the even-index formula in $(\ast)$ with $m\mapsto m+1$.

2) Odd step $2m+3$. Any $A_{2m+3}<F_{m+4}$ forces
$$a_{2m+3}\le (2m+3)F_{m+3}-(2m+2)F_{m+4}=F_{m+3}-(2m+2)F_{m+2}<0,$$
so the minimal admissible choice is $A_{2m+3}=F_{m+4}$, giving $a_{2m+3}=F_{m+4}$. This value is new among odd indices because the previous odd terms are the strictly increasing Fibonacci numbers $F_8,\dots,F_{m+3}$. It remains to show that no earlier even term equals $F_{m+4}$.

Fix any earlier even index $2r\le 2m$. If $r<5$, then $a_{2r}\in\{3,6,5,8\}<F_9=34\le F_{m+4}$. If $r\ge5$, then by the inductive formula $a_{2r}=F_{r+2}+2rF_{r+1}$. Suppose toward a contradiction that $a_{2r}=F_{m+4}$. Put $s=m+2-r\ge0$. By the addition formula,
$$
F_{m+4}=F_{r+s+2}=F_{r+2}F_{s+1}+F_{r+1}F_s.
$$
Hence
$$
F_{r+2}(F_{s+1}-1)=F_{r+1}(2r-F_s).
$$
Since $\gcd(F_{r+2},F_{r+1})=1$, there exists $t\in\mathbb Z$ with
$$
F_{s+1}-1=tF_{r+1},\qquad 2r-F_s=tF_{r+2}.
$$
Because $F_{s+1}\ge1$ and $F_{r+1}\ge F_6=8$, we must have $t\ge0$. For $r\ge5$ one has $F_{r+2}>2r$ (indeed $F_7-10=3>0$, and $(F_{(r+1)+2}-2(r+1))-(F_{r+2}-2r)=F_{r+1}-2\ge6$, so the difference increases). If $t\ge1$, then $2r-F_s=tF_{r+2}\ge F_{r+2}>2r$, impossible since the left-hand side is $\le 2r$. Thus $t=0$, whence $F_{s+1}=1$ and $F_s=2r$. But $F_{s+1}=1$ forces $s\in\{0,1\}$, so $F_s\in\{0,1\}$, contradicting $2r\ge10$. Therefore no even term equals $F_{m+4}$, and $a_{2m+3}=F_{m+4}$ is new. This establishes the odd-index formula in $(\ast)$ with $m\mapsto m+1$ and also $A_{2m+2}=A_{2m+3}=F_{m+4}$.

By induction, $(\ast)$ holds for all $m\ge5$. Writing $n=2m$ or $n=2m+1$ gives, for all $n\ge10$,
$$
 a_n=
 \begin{cases}
 n\,F_{\,\frac n2+3}-(n-1)\,F_{\,\frac n2+2}, & n\text{ even},\\[4pt]
 F_{\,\frac{n+1}{2}+2}, & n\text{ odd}.
 \end{cases}
$$
Finally, the running averages satisfy $A_{2m}=A_{2m+1}=F_{m+3}\in\mathcal F$ by construction, the values $(a_n)$ are pairwise distinct because odd-index terms are the strictly increasing $F_8,F_9,\dots$ and even-index terms are strictly increasing and never equal to any of those odd values, and at each step $A_n$ is the smallest admissible choice; hence the sequence is lexicographically least among all sequences with Fibonacci running averages and distinct terms. 
\end{proof}


\subsubsection{A248982: Disjointness of even/odd value sets}
\begin{proposition}[Disjointness of even/odd value sets (resolves Fried's Conj.~2)]\label{prop:disjointness}
With Fibonacci numbers defined by $F_0=0$, $F_1=1$, $F_{n+2}=F_{n+1}+F_n$, let
\[
 S_{\mathrm{even}}:=\bigl\{\,nF_{\,\frac{n}{2}+3}-(n-1)F_{\,\frac{n}{2}+2}:\ n\text{ even}\,\bigr\},\qquad
 S_{\mathrm{odd}}:=\bigl\{\,F_{\,\frac{n+1}{2}+2}:\ n\text{ odd}\,\bigr\}.
\]
Then $S_{\mathrm{even}}\cap S_{\mathrm{odd}}=\varnothing$. Equivalently, for every integer $n\ge1$,
\[
 T(n):=F_{n+2}+2nF_{n+1}
\]
is not a Fibonacci number.
\end{proposition}

\begin{proof}
First, for $n=1,2,3$ we have $T(1)=4$, $T(2)=11$, $T(3)=23$, none of which is Fibonacci. Hence assume $n\ge4$ and suppose, for a contradiction, that $T(n)=F_m$ for some index $m$.

1) Bounding the index $m$. Using the doubling identity $F_{2n+2}=F_{n+1}^2+2F_nF_{n+1}$ and Cassini's identity $F_{n+1}^2-F_nF_{n+2}=(-1)^n$, we get
\[
\begin{aligned}
F_{2n+2}-T(n)
&=(F_{n+1}^2+2F_nF_{n+1})-(F_n+(2n+1)F_{n+1})\\
&=(F_nF_{n+2}+(-1)^n+2F_nF_{n+1})-(F_n+(2n+1)F_{n+1})\\
&=F_n^2 - F_n + \bigl(3F_n-(2n+1)\bigr)F_{n+1} + (-1)^n.
\end{aligned}
\]
For $n=4$ this equals $7>0$. For $n\ge5$, since $F_n\ge n$ (easy induction), we have $3F_n-(2n+1)\ge n-1\ge4$ and $F_n^2-F_n\ge20$, so the sum is positive. Thus $F_{2n+2}>T(n)=F_m$, hence $m\le 2n+1$.

2) A divisibility constraint. Write $m=n+k$ with $1\le k\le n+1$. By the addition formula $F_{n+k}=F_{n+1}F_k+F_nF_{k-1}$,
\[
0=F_{n+k}-T(n)=F_{n+1}\bigl(F_k-(2n+1)\bigr)+F_n\bigl(F_{k-1}-1\bigr).
\]
Since $\gcd(F_n,F_{n+1})=1$, it follows that
\[
F_{n+1}\mid(F_{k-1}-1)\quad\text{and}\quad F_n\mid\bigl(F_k-(2n+1)\bigr).
\]
Because $1\le k\le n+1$, we have $0\le k-1\le n$ and so $0\le F_{k-1}\le F_n<F_{n+1}$. The only multiple of $F_{n+1}$ with absolute value $<F_{n+1}$ is $0$, hence $F_{k-1}-1=0$ and thus $F_{k-1}=1$, so $k\in\{2,3\}$. Substituting back gives $F_k=2n+1$, but for $k\in\{2,3\}$ one has $F_k\in\{1,2\}$, contradicting $2n+1\ge3$. This contradiction shows that no such $m$ exists, i.e., $T(n)$ is not a Fibonacci number for any $n\ge4$. Together with the checked cases $n=1,2,3$, this holds for all $n\ge1$.

Finally, for even indices $N=2n$,
\[
2nF_{n+3}-(2n-1)F_{n+2}=2n(F_{n+2}+F_{n+1})-(2n-1)F_{n+2}=F_{n+2}+2nF_{n+1}=T(n),
\]
so $S_{\mathrm{even}}=\{T(n):n\ge1\}$, while for odd indices $S_{\mathrm{odd}}=\{F_{t+2}:t\ge1\}$ is precisely the set of Fibonacci numbers $\{F_r:r\ge3\}$. Since $T(n)$ is never Fibonacci, $S_{\mathrm{even}}\cap S_{\mathrm{odd}}=\varnothing$.
\end{proof}

\section{Acknowledgments}
I thank the authors cited below for foundational results used in the proofs. The conjectures we resolve or disprove are due to Cohen \cite{Cohen2025}; we also cite the relevant OEIS entries.

% Auto-generated bibliography (sanitized)
\begin{thebibliography}{99}
\bibitem{Topkis1998} Topkis, D. M. Supermodularity and Complementarity. Princeton University Press, 1998. \url{https://press.princeton.edu/books/hardcover/9780691058649/supermodularity-and-complementarity}
\bibitem{KleinbergMullainathanRaghavan2017} Kleinberg, J., Mullainathan, S., and Raghavan, M. Inherent trade-offs in the fair determination of risk scores. arXiv:1609.05807, 2016. \url{https://arxiv.org/abs/1609.05807}
\bibitem{BenDavidEtAl2010} Ben-David, S., Blitzer, J., Crammer, K., Kulesza, A., Pereira, F., and Wortman Vaughan, J. A theory of learning from different domains. Machine Learning 79(1):151--175, 2010. \url{https://doi.org/10.1007/s10994-009-5152-4}
\bibitem{VanderWeele2015} VanderWeele, T. J. Explanation in Causal Inference: Methods for Mediation and Interaction. Oxford University Press, 2015. \url{https://doi.org/10.1093/acprof:oso/9780199325870.001.0001}
\bibitem{EsfahaniKuhn2018} Mohajerin Esfahani, P. and Kuhn, D. Data-driven distributionally robust optimization using the Wasserstein metric: performance guarantees and tractable reformulations. Mathematical Programming 171(1):115--166, 2018. \url{https://doi.org/10.1007/s10107-017-1171-1}
\bibitem{LebrikizumabTREBLE2018} Simpson, E. L., Flohr, C., Eichenfield, L. F., Bieber, T., Sofen, H., Taieb, A., Paul, C., Cork, M., Thyssen, J. P., Silverberg, J. I., et al. Efficacy and safety of lebrikizumab (an anti-IL-13 monoclonal antibody) in adults with moderate-to-severe atopic dermatitis inadequately controlled by topical corticosteroids: a randomized, placebo-controlled phase II trial (TREBLE). Journal of the American Academy of Dermatology 78(5):863--871, 2018. \url{https://doi.org/10.1016/j.jaad.2018.01.017}
\bibitem{LebrikizumabJAMA2020} Guttman-Yassky, E., Blauvelt, A., Eichenfield, L. F., Paller, A. S., Armstrong, A. W., et al. Efficacy and Safety of Lebrikizumab, a High-Affinity Interleukin 13 Inhibitor, in Adults With Moderate to Severe Atopic Dermatitis. JAMA Dermatology 156(4):411--420, 2020. \url{https://doi.org/10.1001/jamadermatol.2020.0079}
\bibitem{TralokinumabBJD2020} Wollenberg, A., Beck, L. A., Blauvelt, A., Simpson, E. L., Chen, Z., Ballal, S., et al. Tralokinumab for atopic dermatitis: a promising new therapy. British Journal of Dermatology 183(5):740--741, 2020. \url{https://doi.org/10.1111/bjd.19699}
\bibitem{TralokinumabECZTRA12BJD2021} Silverberg, J. I., Toth, D., Bieber, T., Alexis, A. F., Elewski, B. E., Pink, A. E., et al. Tralokinumab for moderate-to-severe atopic dermatitis: results from two 52-week, randomized, double-blind, multicentre, placebo-controlled phase III trials (ECZTRA 1 and ECZTRA 2). British Journal of Dermatology 184(3):437--449, 2021. \url{https://doi.org/10.1111/bjd.19574}
\bibitem{TralokinumabECZTRA3BJD2021} Wollenberg, A., Blauvelt, A., Guttman-Yassky, E., Worm, M., Lynde, C., Lacour, J.-P., et al. Tralokinumab plus topical corticosteroids for the treatment of moderate-to-severe atopic dermatitis: results from the double-blind, randomized, multicentre, placebo-controlled phase III ECZTRA 3 trial. British Journal of Dermatology 184(3):450--463, 2021. \url{https://doi.org/10.1111/bjd.19573}
\bibitem{StaphInADChapter2008} Hoeger, P. H., Lenz, W. Role of \textit{Staphylococcus aureus} in atopic dermatitis. In: Textbook of Atopic Dermatitis. 2008. \url{https://doi.org/10.3109/9780203091449-9}
\bibitem{SimpsonNEJM2016} Simpson, E. L., Bieber, T., Guttman-Yassky, E., Beck, L. A., Blauvelt, A., Cork, M. J., Silverberg, J. I., Deleuran, M., Kataoka, Y., Lacour, J.-P., Kingo, K., Worm, M., Poulin, Y., Wollenberg, A., Soo, Y., Graham, N. M. H., Pirozzi, G., Akinlade, B., Staudinger, H., Mastey, V., Eckert, L., Gadkari, A., Stahl, N., Yancopoulos, G. D., and Ardeleanu, M. Two Phase 3 Trials of Dupilumab versus Placebo in Atopic Dermatitis. New England Journal of Medicine, 375(24):2335--2348, 2016. \url{https://doi.org/10.1056/NEJMoa1610020}
\bibitem{FezakinumabJAAD2018} Guttman-Yassky, E., Brunner, P. M., Neumann, A. U., Khattri, S., Pavel, A. B., Malik, K., Singer, G. K., Baum, D., Gilleaudeau, P., Sullivan-Whalen, M., Misiak-Tlosta, M., Estrada, Y. D., Champhekar, A., Maari, C., Dumont, N., and Krueger, J. G. Efficacy and safety of fezakinumab (an IL-22 monoclonal antibody) in adults with moderate-to-severe atopic dermatitis inadequately controlled by conventional treatments: a randomized, double-blind, phase 2a trial. Journal of the American Academy of Dermatology 78(5):872--881, 2018. \url{https://doi.org/10.1016/j.jaad.2018.01.016}
\bibitem{DatasetShift2008} Quionero-Candela, J., Sugiyama, M., Schwaighofer, A., and Lawrence, N. D. (eds.). Dataset Shift in Machine Learning. MIT Press, 2008. \url{https://doi.org/10.7551/mitpress/9780262170055.001.0001}
\bibitem{ArjovskyIRM2019} Arjovsky, M., Bottou, L., Gulrajani, I., and Lopez-Paz, D. Invariant Risk Minimization. arXiv:1907.02893, 2019. \url{https://arxiv.org/abs/1907.02893}
\bibitem{HardtEquality2016} Hardt, M., Price, E., and Srebro, N. Equality of Opportunity in Supervised Learning. arXiv:1610.02413, 2016. \url{https://arxiv.org/abs/1610.02413}
\bibitem{HuangBleach2011} Huang, J. T., Abrams, M., Tlougan, B., Rademaker, A., and Paller, A. S. Dilute bleach baths for Staphylococcus aureus colonization in atopic dermatitis to decrease disease severity. Archives of Dermatology 147(2):246--253, 2011. \url{https://doi.org/10.1001/archdermatol.2010.434}
\end{thebibliography}


\end{document}
