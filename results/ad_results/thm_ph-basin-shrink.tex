\begin{theorem}\label{thm:ph-basin-shrink}
In a microbiome system with pH-dependent AMP potency \(\alpha(\mathrm{pH})\), there exist admissible parameters and \(\Delta\mathrm{pH}>0\) such that increasing skin pH by \(\Delta\mathrm{pH}\) (so that \(\alpha\) decreases) strictly shrinks the basin of attraction of the healthy equilibrium:
\[
\mathrm{basin}\big(E_{\mathrm{h}};\alpha(\mathrm{pH}{+}\Delta\mathrm{pH})\big)\;\subsetneq\;\mathrm{basin}\big(E_{\mathrm{h}};\alpha(\mathrm{pH})\big).
\]
\end{theorem}
\begin{proof}
Assume, for contradiction, that in the concrete system below an admissible increase in pH (hence a decrease in \(\alpha\)) enlarges the healthy basin.

Let the physiological pH interval be \(I=[\mathrm{pH}_0,\mathrm{pH}_0+H]\) and set
\[
\alpha(p):=\alpha_0-\zeta\,(p-\mathrm{pH}_0),\qquad 0<\alpha_0<1,\;0<\zeta<\alpha_0/H,
\]
so \(\alpha(I)=[\alpha_{\min},\alpha_0]\) with \(\alpha_{\min}:=\alpha_0-\zeta H\in(0,\alpha_0)\). Consider the planar system for commensals \(C\ge 0\) and \emph{S.\!aureus} \(S\ge 0\):
\[
\dot C=f(C,S):=C(1-C-bS),\qquad \dot S=g(C,S;\alpha):=S(1-\alpha-S-cC),
\]
with constants \(b,c>0\) chosen to satisfy
\[
\text{(A1)}\; b>\frac{1}{1-\alpha_0},\qquad \text{(A2)}\; c>1-\alpha_{\min}.
\]
\emph{1) Forward invariance and boundedness.} The axes \(\{C=0\}\), \(\{S=0\}\) are invariant and
\[
\dot C\le C(1-C),\qquad \dot S\le S\big((1-\alpha_{\min})-S\big),
\]
so every trajectory with \(C,S\ge 0\) is bounded and the rectangle \(Q:=[0,1]\times[0,1-\alpha_{\min}]\) is forward invariant for all \(\alpha\in\alpha(I)\).

\emph{2) Equilibria and their type.} The Jacobian is
\[
J(C,S;\alpha)=\begin{pmatrix}1-2C-bS & -bC\\ -cS & 1-\alpha-2S-cC\end{pmatrix}.
\]
The equilibria in the closed first quadrant are
\[
E_0=(0,0),\quad E_{\mathrm{h}}=(1,0),\quad E_{\mathrm{d}}=(0,1-\alpha),\quad P(\alpha)=(C^*(\alpha),S^*(\alpha))\in(0,\infty)^2,
\]
where
\[
S^*(\alpha)=\frac{c-1+\alpha}{cb-1},\qquad C^*(\alpha)=\frac{b(1-\alpha)-1}{cb-1}.
\]
Under (A1)--(A2), \(cb>\dfrac{c}{1-\alpha_0}>1\), hence \(P(\alpha)\in(0,\infty)^2\). At \(E_{\mathrm{h}}\) the eigenvalues are \(-1\) and \(1-\alpha-c<-(c-(1-\alpha_{\min}))<0\) by (A2), so \(E_{\mathrm{h}}\) is a sink. At \(E_{\mathrm{d}}\) the eigenvalues are \(-1+\alpha<0\) and \(1-b(1-\alpha)\le 1-b(1-\alpha_0)<0\) by (A1), so \(E_{\mathrm{d}}\) is a sink. At \(E_0\) both eigenvalues are positive, hence a source. At \(P(\alpha)\) one has \(1-2C^*-bS^*=-C^*\) and \(1-\alpha-2S^*-cC^*=-S^*\), so
\[
\operatorname{tr}J(P)=-(C^*+S^*)<0,\qquad \det J(P)=C^*S^*(1-cb)<0,
\]
whence \(P(\alpha)\) is a saddle.

\emph{3) No periodic orbits.} In \((0,\infty)^2\), take the Bendixson--Dulac function \(B(C,S)=\dfrac{1}{CS}\). Then
\[
\frac{\partial}{\partial C}\big(Bf\big)+\frac{\partial}{\partial S}\big(Bg\big)
=\frac{\partial}{\partial C}\Big(\frac{1}{S}(1-C-bS)\Big)+\frac{\partial}{\partial S}\Big(\frac{1}{C}(1-\alpha-S-cC)\Big)
=-\frac{1}{S}-\frac{1}{C}<0
\]
on \((0,\infty)^2\). Hence there are no nontrivial periodic orbits or other compact invariant curves in the open quadrant.

Consequently, for each \(\alpha\in\alpha(I)\) the only attractors in the first quadrant are the sinks \(E_{\mathrm{h}}\) and \(E_{\mathrm{d}}\); the unique interior saddle \(P(\alpha)\) has a one-dimensional \(C^1\) stable manifold \(\Sigma(\alpha):=W^s(P(\alpha))\) that lies in \((0,\infty)^2\) and separates the two basins \(\mathcal B(\alpha):=\mathrm{basin}(E_{\mathrm{h}};\alpha)\) and \(\mathcal B^{\mathrm{d}}(\alpha):=\mathrm{basin}(E_{\mathrm{d}};\alpha)\) within \(Q\cap(0,\infty)^2\).

\emph{Monotone dependence on \(\alpha\) and basin inclusion.} Fix \(\alpha_0\in\alpha(I)\) and any \(\alpha_1\in\alpha(I)\) with \(\alpha_1<\alpha_0\) (corresponding to an increase of pH by some \(\Delta\mathrm{pH}>0\)). Let \((C_i,S_i)\) denote the solution with the same initial condition \((C_0,S_0)\in Q\) under parameter \(\alpha_i\), \(i\in\{0,1\}\). Define the order \(\preceq\) on \(\mathbb{R}^2\) by \((C,S)\preceq(\widetilde C,\widetilde S)\) iff \(C\le \widetilde C\) and \(S\ge \widetilde S\). A first-violation argument yields, for all \(t\ge 0\),
\[
C_1(t)\le C_0(t),\qquad S_1(t)\ge S_0(t).
\]
Indeed, suppose \(\tau\) is the first time at which either \(C_1>C_0\) or \(S_1<S_0\). If \(C_1=C_0\) and \(S_1\ge S_0\) at \(t=\tau\), then
\[
\dot C_1-\dot C_0=f(C_1,S_1)-f(C_0,S_0)=-b\,C_0(\tau)\,(S_1-S_0)\le 0,
\]
so \(C_1\) cannot cross above \(C_0\). If \(S_1=S_0\) and \(C_1\le C_0\) at \(t=\tau\), then
\[
\dot S_1-\dot S_0=g(C_1,S_1;\alpha_1)-g(C_0,S_0;\alpha_0)=-c\,S_0(\tau)\,(C_1-C_0)+\big(\alpha_0-\alpha_1\big)S_0(\tau)\ge 0,
\]
so \(S_1\) cannot cross below \(S_0\). Thus the order is preserved for all \(t\ge 0\).

Now take any \((C_0,S_0)\in\mathcal B(\alpha_1)\). Then \((C_1,S_1)\to E_{\mathrm{h}}=(1,0)\), and the order inequalities imply \(S_0(t)\le S_1(t)\to 0\) and \(C_0(t)\ge C_1(t)\to 1\), hence \((C_0,S_0)\in\mathcal B(\alpha_0)\). Therefore
\[
\mathcal B(\alpha_1)\subseteq\mathcal B(\alpha_0)\qquad\text{whenever }\alpha_1<\alpha_0.
\]
This already contradicts the assumed enlargement \(\mathcal B(\alpha_1)\supseteq\mathcal B(\alpha_0)\).

\emph{Strictness of the inclusion.} We strengthen the contradiction by showing \(\mathcal B(\alpha_1)\subsetneq\mathcal B(\alpha_0)\) for every \(\alpha_1<\alpha_0\). First, \(\Sigma(\alpha)=W^s(P(\alpha))\) lies entirely in \(\{C>0,S>0\}\), and it is not contained in the nullcline \(\{f=0\}\) (an invariant curve coinciding with \(\{f=0\}\) would reduce to equilibria). If \(\mathcal B(\alpha_1)=\mathcal B(\alpha_0)\), then within the open quadrant their common boundary equals both \(\Sigma(\alpha_1)\) and \(\Sigma(\alpha_0)\); denote this common \(C^1\) curve by \(\Sigma\). Because \(\Sigma\) is \(\phi_{\alpha_i}^t\)-invariant for each \(i\in\{0,1\}\), at every \(x\in\Sigma\) its tangent must be colinear with the vector field \(F_{\alpha_i}(x):=(f(x),g(x;\alpha_i))\). Choose \(x\in\Sigma\) with \(S(x)>0\) and \(f(x)\neq 0\). Then
\[
F_{\alpha_0}(x)=(f(x),g(x;\alpha_0)),\qquad F_{\alpha_1}(x)=(f(x),g(x;\alpha_0)-\big(\alpha_0-\alpha_1\big)S(x)),
\]
and since \(S(x)>0\) and \(\alpha_0>\alpha_1\), these two vectors are not colinear. Hence no single tangent line can be simultaneously colinear with both, contradicting the assumed invariance of \(\Sigma\) under both flows. Therefore \(\Sigma(\alpha_1)\ne\Sigma(\alpha_0)\), which, together with \(\mathcal B(\alpha_1)\subseteq\mathcal B(\alpha_0)\), implies
\[
\mathcal B(\alpha_1)\subsetneq\mathcal B(\alpha_0)\qquad\text{for every }\alpha_1<\alpha_0.
\]

\emph{Conclusion.} For the system constructed above, any admissible increase in pH (hence decrease in \(\alpha\)) strictly shrinks the healthy basin:
\[
\mathrm{basin}\big(E_{\mathrm{h}};\alpha(\mathrm{pH}_0{+}\Delta\mathrm{pH})\big)\;\subsetneq\;\mathrm{basin}\big(E_{\mathrm{h}};\alpha(\mathrm{pH}_0)\big),\quad\qedhere
\]
contradicting the assumed enlargement. Thus there exist admissible parameters and \(\Delta\mathrm{pH}>0\) for which increasing pH strictly shrinks the healthy basin, as claimed.
\end{proof}
