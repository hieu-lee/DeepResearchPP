\documentclass[11pt]{article}
\usepackage[T1]{fontenc}
\usepackage[utf8]{inputenc}
\usepackage{amsmath,amsthm,amssymb}
\usepackage{enumitem}
\usepackage{hyperref}
\usepackage[a4paper,margin=1in]{geometry}
\numberwithin{equation}{section}
% Global typesetting tweaks to reduce overfull/underfull boxes
\tolerance=2000
\emergencystretch=4em

% Theorem environments
\theoremstyle{plain}
\newtheorem{theorem}{Theorem}[section]
\newtheorem{lemma}[theorem]{Lemma}
\newtheorem{proposition}[theorem]{Proposition}
\newtheorem{corollary}[theorem]{Corollary}
\theoremstyle{definition}
\newtheorem{definition}[theorem]{Definition}
\theoremstyle{remark}
\newtheorem{remark}[theorem]{Remark}

\title{Limits of Monotonicity, Robust Selection, and Dosing Optimality in AD:\\ A Unified Mathematical Treatment with Applications to IL-13/IL-22, Microbiome, and Fairness under Shift}
\author{\normalsize Anonymous}
\date{September 27, 2025}

\begin{document}
\sloppy
\maketitle

\begin{abstract}
We develop a unified mathematical treatment of several phenomena that arise when choosing, personalizing, and dosing biologic therapies for atopic dermatitis (AD). Our results span three domains that are often analyzed in isolation: (i) statistical structure and limits of biomarker-based decision rules (e.g., logistic link invariances, threshold geometry, and failures of eventual positivity or monotonicity), (ii) pharmacokinetic/pharmacodynamic (PK/PD) dosing optimality and trade-offs (e.g., fixed-regimen optimality at exposure-only objectives, non-concave Pareto frontiers for efficacy vs adverse events, and regimes where pulsed exposure can match or exceed continuous exposure at lower total dose), and (iii) robustness and fairness under distribution shift and group stratification (e.g., impossibility theorems for ranking gains and post-hoc calibration, AUC gaps from tail geometry, and invariance constraints on scores).
Motivated by the IL-13 and IL-22 axes that underlie modern AD therapeutics\,\cite{SimpsonNEJM2016,FezakinumabJAAD2018} and by the role of the skin microbiome and colonization\,\cite{HuangBleach2011}, we present sharp conditions under which simple thresholds are provably insufficient or fragile, clarify when rectangular policies cannot match Bayes rules, and identify settings where dosing equalizers exist or not. On the robustness side, we connect classical distribution shift\,\cite{DatasetShift2008,BenDavidEtAl2010}, fairness trade-offs\,\cite{KleinbergMullainathanRaghavan2017,HardtEquality2016}, and distributionally robust optimization\,\cite{EsfahaniKuhn2018} to clinical selection and monitoring tasks. Across the paper, each narrative claim is paired with a precise proposition or theorem, yielding a consolidated toolkit that can be specialized to concrete AD use-cases while making the limits of what can be achieved transparent.
\end{abstract}

\section{Introduction}
Biologics that modulate type 2 inflammation have transformed atopic dermatitis (AD) care. Dupilumab, which blocks IL-4R$\alpha$ (and thereby IL-4/IL-13 signaling), improved EASI and pruritus endpoints in two large phase 3 trials\,\cite{SimpsonNEJM2016}. Parallel efforts targeting IL-22 (e.g., fezakinumab) explored clinical benefits in biomarker-defined subgroups\,\cite{FezakinumabJAAD2018}. In practice, clinicians and developers face three intertwined challenges: (1) how to design reliable biomarker thresholds and selection rules; (2) how to choose or adapt dosing to balance efficacy and adverse events under basic PK/PD structure; and (3) how to maintain robustness and equity as populations and environments shift.

This paper provides a unified mathematical treatment of these challenges. On biomarker selection and rule design, we prove invariance and non-invariance statements for logistic models and show that simple rectangular thresholds can be provably suboptimal, even when increasing-differences and monotonicity heuristics appear to support them. On dosing, we identify regimes where fixed regimens maximize exposure-only objectives, illuminate when the efficacy vs safety Pareto frontier fails to be strictly concave, and give conditions under which pulsed exposure matches continuous exposure at dramatically lower total dose. On robustness and fairness, we quantify what can and cannot be guaranteed under covariate shift and post-hoc calibration, relate score constraints to AUC behavior, and connect to distributionally robust and invariant modeling\,\cite{DatasetShift2008,BenDavidEtAl2010,EsfahaniKuhn2018,ArjovskyIRM2019,HardtEquality2016}.

Empirically, our scope is motivated by two broad observations. First, the axes IL-13 and IL-22 play distinct roles across AD endotypes, and microbiome features such as \emph{Staphylococcus aureus} colonization interact with disease severity and flare risk\,\cite{HuangBleach2011}. Second, deployment settings are rarely static: exposures (e.g., air quality), adherence, and referral patterns induce shift between development and deployment cohorts\,\cite{DatasetShift2008}. We therefore focus on structural results that are portable across data regimes, yet specific enough to falsify common but overly optimistic assumptions.

\paragraph{Contributions.} We synthesize and prove a suite of results that (a) formalize invariance and threshold geometry for logistic-link models; (b) exhibit failures of monotonicity and eventual positivity in clinically plausible settings; (c) clarify limits of rectangular policies and ROC geometry; (d) give PK/PD optimality and non-concavity results, including pulse vs continuous exposure comparisons; and (e) state impossibility theorems and invariance-driven constraints under shift and fairness criteria. Each item is stated as a self-contained proposition or theorem with minimal assumptions, enabling direct reuse.

\section{Related Work}
\textbf{Therapeutics and biomarkers in AD.} Phase 3 trials established the efficacy of dupilumab for moderate-to-severe AD\,\cite{SimpsonNEJM2016}. IL-22 blockade has shown signals in specific strata\,\cite{FezakinumabJAAD2018}. Other anti-IL-13 agents (lebrikizumab, tralokinumab) have demonstrated efficacy signals across Phase II/III programs, including TREBLE and the Phase 3 ECZTRA series\,\cite{LebrikizumabTREBLE2018,LebrikizumabJAMA2020,TralokinumabBJD2020,TralokinumabECZTRA12BJD2021,TralokinumabECZTRA3BJD2021}. Microbiome-targeted adjuncts, including anti-colonization strategies, have demonstrated reductions in severity in selected settings\,\cite{HuangBleach2011}. Our biomarker results complement this line of work by characterizing when thresholding and ranking rules can be relied upon and when they cannot.

\textbf{Microbiome background.} Colonization with \textit{Staphylococcus aureus} is common in AD and correlates with disease severity; antiseptic regimens and decolonization strategies are often considered as adjuncts. Evidence supports dilute bleach baths and other decolonization measures in select settings\,\cite{HuangBleach2011}, and broader reviews summarize the roles of \textit{S. aureus} toxins, biofilms, and host barrier interactions in AD pathophysiology\,\cite{StaphInADChapter2008}. Our mechanistic results (e.g., pH-dependent basin shrinkage and stability considerations) conceptually align with these observations by formalizing how shifts in antimicrobial potency or colonization pressure can alter long-run skin state.

\textbf{PK/PD and dosing principles.} Basic exposure--response formulations motivate fixed and pulsed regimens under one-compartment kinetics. Our results formalize when fixed regimens can be optimal for exposure-only objectives and when efficacy--safety trade-offs break strict concavity, implying that scalarized objectives may miss Pareto-efficient regimens. These findings connect to classical optimization (e.g., supermodularity\,\cite{Topkis1998}) and to robustness considerations.

\textbf{Distribution shift, robustness, and fairness.} Dataset shift and covariate shift have long been recognized in machine learning\,\cite{DatasetShift2008,BenDavidEtAl2010}. Fairness trade-offs show that incompatible desiderata prevent universal optimality across groups\,\cite{KleinbergMullainathanRaghavan2017,HardtEquality2016}. Distributionally robust optimization provides tools for worst-case guarantees over Wasserstein neighborhoods\,\cite{EsfahaniKuhn2018}, while invariance principles aim for predictors stable across environments\,\cite{ArjovskyIRM2019}. We adapt these ideas to AD selection and monitoring, highlighting formal limits and attainable guarantees.

\section{Main Results: Statements}
For completeness and reproducibility, we include the formal statements as separate files. We organize them by theme.

\subsection*{Logistic and threshold foundations}
\begin{proposition}[Equal adjusted PM$_{2.5}$ odds ratios across endotypes in a no-interaction logistic model]\label{prop:logit-equal-or}
Fix a therapy status $T$. In any logistic model with no exposure-by-endotype interaction,
\[
\operatorname{logit}\,\mathbb{P}(F{=}1\mid P,H,X,T)=\alpha+\eta H+\beta P+\gamma^{\top}X,
\]
where $F\in\{0,1\}$ (flare), $P\in\mathbb{R}$ (PM$_{2.5}$ exposure), $H\in\{0,1\}$ (EosLow/EosHigh), and $X$ denotes arbitrary adjustment covariates, the per-10 $\mu\mathrm{g}/\mathrm{m}^3$ PM$_{2.5}$ odds ratio adjusted for $X$ is identical across endotypes:
\[
\mathrm{OR}_{10}(H{=}0;X,T)=\mathrm{OR}_{10}(H{=}1;X,T)=e^{10\beta}.
\]
Consequently, the EosLow-to-EosHigh odds-ratio ratio equals $1$, so a universal strict inequality $\mathrm{OR}(F\mid \text{EosLow},T)<\mathrm{OR}(F\mid \text{EosHigh},T)$ cannot hold within this model class.
\end{proposition}

\begin{proof}
Fix $T$. Let $F\in\{0,1\}$ denote a flare, $P\in\mathbb{R}$ PM$_{2.5}$ exposure, $H\in\{0,1\}$ the endotype ($0$:\ EosLow; $1$:\ EosHigh), and $X$ any adjustment covariates. Assume
\[
\operatorname{logit}\,\mathbb{P}(F{=}1\mid P,H,X,T)=\alpha+\eta H+\beta P+\gamma^{\top}X,
\]
with arbitrary real parameters $\alpha,\eta,\beta$ and vector $\gamma$. For any increment $\delta>0$, define the $X$-adjusted per-$\delta$ odds ratio within endotype $H$ by
\[
\mathrm{OR}_\delta(H;X,T)
:=\frac{\dfrac{\mathbb{P}(F{=}1\mid P{+}\delta,H,X,T)}{1-\mathbb{P}(F{=}1\mid P{+}\delta,H,X,T)}}{\dfrac{\mathbb{P}(F{=}1\mid P,H,X,T)}{1-\mathbb{P}(F{=}1\mid P,H,X,T)}}.
\]
By logit additivity, the odds at exposure $P$ equal $\exp(\alpha+\eta H+\beta P+\gamma^{\top}X)$, hence
\[
\mathrm{OR}_\delta(H;X,T)=\exp\big((\alpha+\eta H+\beta(P{+}\delta)+\gamma^{\top}X)-(\alpha+\eta H+\beta P+\gamma^{\top}X)\big)=e^{\beta\delta},
\]
which is independent of $H$, $X$, $P$, and $T$. Taking $\delta=10$ yields
\[
\mathrm{OR}_{10}(H{=}0;X,T)=\mathrm{OR}_{10}(H{=}1;X,T)=e^{10\beta}.\qedhere
\]
\end{proof}
\begin{theorem}[Logit-cutoff invariance under TARC sublevel restriction]\label{thm:logit-cutoff-invariance}
For every logistic-link model with additive logit and nonnegative T-coefficient and positive L-coefficient, i.e., for every model with
\[
\operatorname{logit}\,p(L,T)=\beta_0+\beta_L L+\beta_T T \quad\text{with}\quad \beta_L>0,\ \beta_T\ge 0\ \ (\text{so }\beta_{LT}=0),
\]
and for every target probability level $q\in(0,1)$, the minimal uniform $L$-cutoff is invariant under any TARC sublevel restriction: for all $\tau_{\mathrm{TARC}}\ge0$ and $A'=[0,\tau_{\mathrm{TARC}}]$,
\[
\tau_L(A')=\tau_L(\mathbb{R}_+).
\]
In particular, no such sublevel restriction yields a strictly smaller cutoff.
\end{theorem}

\begin{proof}
Fix such a model and let $q\in(0,1)$ with $\gamma:=\operatorname{logit}(q)$. Because the logit map is strictly increasing, the $q$-superlevel set is the halfspace
\[
S_q=\{(L,T)\in\mathbb{R}_+^2:\ p(L,T)\ge q\}
 = \{(L,T):\ \beta_L L+\beta_T T\ge \gamma-\beta_0\}.
\]
For any measurable $A\subseteq\mathbb{R}_+$, the minimal uniform $L$-cutoff
\[
\tau_L(A):=\inf\{\tau\ge0:\ [\tau,\infty)\times A\subseteq S_q\}
\]
satisfies
\[
\tau_L(A)=\max\Bigl\{\sup_{T\in A}\frac{\gamma-\beta_0-\beta_T T}{\beta_L},\ 0\Bigr\},
\]
because for each fixed $T$ the section $\{L:\ (L,T)\in S_q\}$ equals $[((\gamma-\beta_0-\beta_T T)/\beta_L),\ \infty)$ and is monotone in $L$ since $\beta_L>0$.

Apply this formula with $A=\mathbb{R}_+$ and with any TARC sublevel set $A'=[0,\tau_{\mathrm{TARC}}]$. As $\beta_T\ge0$, the map $T\mapsto (\gamma-\beta_0-\beta_T T)/\beta_L$ is nonincreasing on $\mathbb{R}_+$, so in both cases the supremum over $T\in A$ is attained at $T=0$. Therefore,
\[
\tau_L(\mathbb{R}_+)=\max\Bigl\{\frac{\gamma-\beta_0}{\beta_L},\ 0\Bigr\}
=\tau_L(A'), \qquad \text{for all }\tau_{\mathrm{TARC}}\ge0.\qedhere
\]
\end{proof}

\begin{proposition}\label{prop:logit-threshold}
Suppose that for nonnegative covariates $(L_0,M_0,T)\in\mathbb R_+^3$ the model satisfies
\[
\operatorname{logit}\,\mathbb P(E75\mid L_0,M_0,T)=\beta_0+\beta_L L_0+\beta_M M_0+\beta_T T+\beta_{LM}L_0M_0,
\]
with coefficients $\beta_L>0$, $\beta_M\ge 0$, $\beta_T\ge 0$, and $\beta_{LM}\ge 0$. For $A\subset\mathbb R_+$ (in $M_0$) and $B\subset\mathbb R_+$ (in $T$), define
\[
\tau_L(A,B):=\inf\big\{\tau\ge 0:[\tau,\infty)\times A\times B\subset\{(L_0,M_0,T):\mathbb P(E75\mid L_0,M_0,T)\ge p\}\big\}.
\]
Then for every $p\in(0,1)$, every $\tau_M\ge 0$, and every $B\subset\mathbb R_+$ with $0\in B$,
\[
\tau_L\big([\tau_M,\infty),B\big)=\Bigg[\,\frac{\operatorname{logit}(p)-\beta_0-\beta_M\tau_M}{\beta_L+\beta_{LM}\tau_M}\,\Bigg]_+,
\]
so in particular $\tau_L\big([\tau_M,\infty),[0,\tau_T]\big)=\tau_L\big([\tau_M,\infty),\mathbb R_+\big)$ for all $\tau_T\ge 0$. Here $[x]_+:=\max\{x,0\}$.
\end{proposition}

\begin{proof}
Let $\sigma(x):=\tfrac{1}{1+e^{-x}}$ and write
\[
\mathbb P(E75\mid L_0{=}L,M_0{=}M,T)=\sigma\big(g(L,M,T)\big),\quad g(L,M,T):=\beta_0+\beta_L L+\beta_M M+\beta_T T+\beta_{LM}LM.
\]
Fix $p\in(0,1)$ and set $c:=\operatorname{logit}(p)$. For $L\ge 0$ we have
\[
\frac{\partial g}{\partial M}(L,M,T)=\beta_M+\beta_{LM}L\ge 0,\qquad \frac{\partial g}{\partial T}(L,M,T)=\beta_T\ge 0,
\]
by $\beta_M\ge 0$, $\beta_T\ge 0$, and $\beta_{LM}\ge 0$. Hence for any $\tau_M\ge 0$ and any $B\subset\mathbb R_+$ with $0\in B$,
\[
\inf\{g(L,M,T):M\in[\tau_M,\infty),\ T\in B\}=g(L,\tau_M,0).
\]
Because $\sigma$ is strictly increasing, the condition $\mathbb P(E75\mid L,M,T)\ge p$ is equivalent to $g(L,M,T)\ge c$. Therefore,
\[
[\tau,\infty)\times[\tau_M,\infty)\times B\subset\{g\ge c\}\quad\Longleftrightarrow\quad g(L,\tau_M,0)\ge c\text{ for all }L\ge\tau.
\]
Now $L\mapsto g(L,\tau_M,0)=\beta_0+(\beta_L+\beta_{LM}\tau_M)L+\beta_M\tau_M$ is strictly increasing because $\beta_L>0$ and $\beta_{LM}\ge 0$, so $\beta_L+\beta_{LM}\tau_M>0$. Consequently, the minimal $\tau\ge 0$ satisfying $g(\tau,\tau_M,0)\ge c$ is
\begin{equation*}
\tau_L\big([\tau_M,\infty),B\big)=\Bigg[\,\frac{c-\beta_0-\beta_M\tau_M}{\beta_L+\beta_{LM}\tau_M}\,\Bigg]_+.
\end{equation*}
independent of $B$ whenever $0\in B$, and hence of $[0,\tau_T]$ versus $\mathbb R_+$.\qedhere
\end{proof}

\begin{proposition}[Monotonicity of minimal uniform IL-13 cutoff]\label{prop:il13-cutoff-monotone}
Let \(f(L,T):=\mathbb{P}(\mathrm{EASI}{-}75\mid \text{dupilumab},L,T)\) and, for any \(p\in\mathbb{R}\), define
\[
S_p := \{ (L,T) \in \mathbb{R}_+^2 : f(L,T) \ge p \}.
\]
For any subset \(A\subseteq \mathbb{R}_+\), define the minimal uniform IL-13 cutoff over \(A\) by
\[
\tau_L(A) := \inf\{ \,\tau \ge 0 : [\tau,\infty) \times A \subseteq S_p \,\}\quad (\inf\emptyset := +\infty).
\]
Then for all \(A',A \subseteq \mathbb{R}_+\) with \(A'\subseteq A\), one has \(\tau_L(A') \le \tau_L(A)\). In particular, for any \(\tau_{\mathrm{TARC}} \ge 0\), conditioning on \(\mathrm{TARC} \ge \tau_{\mathrm{TARC}}\) cannot increase the minimal uniform IL-13 cutoff guaranteeing response at level \(p\).
\end{proposition}

\begin{proof}
Fix \(p\in\mathbb{R}\) and set
\[
S_p:=\{(L,T)\in\mathbb{R}_+^2: f(L,T)\ge p\},\qquad f(L,T):=\mathbb{P}(\mathrm{EASI}{-}75\mid \text{dupilumab},L,T).
\]
For any set \(A\subseteq\mathbb{R}_+\), let
\[
G(A):=\{\tau\in[0,\infty):\ [\tau,\infty)\times A\subseteq S_p\},\qquad \tau_L(A):=\inf G(A)\in[0,\infty] \ (\inf\emptyset:=+\infty).
\]
If \(A'\subseteq A\), then for every \(\tau\ge0\) we have \([\tau,\infty)\times A'\subseteq[\tau,\infty)\times A\), hence \(G(A)\subseteq G(A')\). Taking infima yields
\[
\tau_L(A')\le \tau_L(A).
\]
This proves antitonicity of \(A\mapsto \tau_L(A)\) under set inclusion. In particular, with \(A=[0,\infty)\) and \(A'=[\tau_{\mathrm{TARC}},\infty)\) for any \(\tau_{\mathrm{TARC}}\ge0\), we obtain
\[
\tau_L([\tau_{\mathrm{TARC}},\infty))\le \tau_L([0,\infty)).\qedhere
\]
\end{proof}


\subsection*{Monotonicity limits and threshold geometry}
\begin{proposition}\label{prop-no-eventual-positivity}
Define
\[
\Delta_{22}(M_0):=\mathbb{E}\!\left[\Delta\mathrm{EASI}\,\middle|\,\text{dual(IL-13+IL-22)},\, M_0\right]-\mathbb{E}\!\left[\Delta\mathrm{EASI}\,\middle|\,\text{IL-13 only},\, M_0\right].
\]
There exist admissible choices of these conditional expectations (as functions of $M_0$) such that, for every $M_1$, there exist infinitely many $M_0\ge M_1$ with $\Delta_{22}(M_0)<0$ and $\tfrac{\partial}{\partial M_0}\Delta_{22}(M_0)\le 0$. In particular, one cannot in general guarantee the existence of $M_1$ such that $\Delta_{22}(M_0)\ge 0$ for all $M_0\ge M_1$ (hence, a fortiori, cannot guarantee eventual strict increase).
\end{proposition}

\begin{proof}
Define the conditional mean improvements (deterministic given $M_0$) by
\[
\mathbb{E}\!\left[\Delta\mathrm{EASI}\,\middle|\, \text{dual(IL-13+IL-22)},\, M_0\right]\equiv 0,\qquad
\mathbb{E}\!\left[\Delta\mathrm{EASI}\,\middle|\, \text{IL-13 only},\, M_0\right]\equiv -\sin M_0.
\]
Then
\[
\Delta_{22}(M_0)=\sin M_0,\qquad \frac{\partial}{\partial M_0}\Delta_{22}(M_0)=\cos M_0.
\]
For any $M_1$, consider the sequence $M_0^{(k)}:=\tfrac{5\pi}{4}+2\pi k$ for $k\in\mathbb{N}$. For all sufficiently large $k$, $M_0^{(k)}\ge M_1$. At each such point,
\[
\Delta_{22}\bigl(M_0^{(k)}\bigr)=\sin\!\bigl(\tfrac{5\pi}{4}+2\pi k\bigr)=-\tfrac{\sqrt{2}}{2}<0,\qquad
\frac{\partial}{\partial M_0}\Delta_{22}\bigl(M_0^{(k)}\bigr)=\cos\!\bigl(\tfrac{5\pi}{4}+2\pi k\bigr)=-\tfrac{\sqrt{2}}{2}\le 0.\qedhere
\]
\end{proof}

\begin{proposition}[Monotonicity can fail]\label{prop:monotonicity-fails}
For any threshold $\tau_M\in\mathbb{R}$, there exists a deterministic conditional response model defined on the population $\{M_0\ge \tau_M\}$ such that:
\begin{itemize}
  \item the function $M_0\mapsto \mathbb{E}[\Delta\mathrm{EASI}\mid \text{dual(IL-13+IL-22) add-on}, M_0]$ is constant (hence not strictly increasing) over $\{M_0\ge \tau_M\}$; and
  \item for every $M_0\ge \tau_M$, $\mathbb{P}(\mathrm{EASI}{-}75\mid \text{dual add-on}, M_0)=0$ and $\mathbb{P}(\mathrm{EASI}{-}75\mid \text{switch to IL-13 only}, M_0)=1$.
\end{itemize}
Consequently, the claimed monotonicity and any uniform positive probability gap in favor of the dual add-on do not hold in general.
\end{proposition}

\begin{proof}
Let $\tau_M\in\mathbb{R}$ be arbitrary and consider the following deterministic conditional response model defined for all $M_0\ge \tau_M$:
\begin{itemize}
  \item Switch to IL-13 only: $\Delta_{\mathrm{IL}{-}13}(M_0)\equiv 0.80$.
  \item Dual(IL-13+IL-22) add-on: $\Delta_{\mathrm{dual}}(M_0)\equiv 0.60$.
\end{itemize}
Let $\theta:=0.75$ denote the EASI-75 threshold. Then, for every $M_0\ge \tau_M$,
\[
\mathbb{E}[\Delta\mathrm{EASI}\mid \text{dual add-on}, M_0]=0.60,
\]
which is constant in $M_0$ and hence not strictly increasing over $\{M_0\ge \tau_M\}$. Moreover,
\[
\mathbb{P}(\mathrm{EASI}{-}75\mid \text{dual add-on}, M_0)=\mathbb{P}(\Delta_{\mathrm{dual}}(M_0)\ge \theta)=\mathbb{P}(0.60\ge 0.75)=0,
\]
while
\[
\mathbb{P}(\mathrm{EASI}{-}75\mid \text{switch to IL-13 only}, M_0)=\mathbb{P}(\Delta_{\mathrm{IL}{-}13}(M_0)\ge \theta)=\mathbb{P}(0.80\ge 0.75)=1.\qedhere
\]
\end{proof}


\subsection*{Duality and ROC geometry}
\begin{theorem}[Dual--optimal upper boundary is not a 1--Lipschitz decreasing graph]\label{thm:dual-upper-nongraph}
There exist functions $\mu_{\mathrm{Dual}},\mu_{\mathrm{L13}},\mu_{\mathrm{D}}$ with $\mu_{\mathrm{Dual}}-\mu_{\mathrm{L13}}$ having increasing differences in $(L_0,M_0)$ such that, for every $\tau_L\ge0$, the Dual--optimal region
\[
U^*:=\{(L_0,M_0):\ \mu_{\mathrm{Dual}}(L_0,M_0)\ge\max(\mu_{\mathrm{D}},\mu_{\mathrm{L13}})\}
\]
has an upper boundary $\partial^+U^*$ that is not the graph of any strictly decreasing, $1$--Lipschitz function on $[\tau_L,\infty)$. This holds even with $\mu_{\mathrm{D}}\equiv0$, $\mu_{\mathrm{L13}}$ constant, and $\mu_{\mathrm{Dual}}$ affine.
\end{theorem}

\begin{proof}
Fix any $a>1$ and any $c\ge0$ and define
\[
\mu_{\mathrm{Dual}}(L_0,M_0)=aL_0+M_0,\qquad \mu_{\mathrm{L13}}(L_0,M_0)=c,\qquad \mu_{\mathrm{D}}(L_0,M_0)=0.
\]
Let $\Delta(L_0,M_0):=\mu_{\mathrm{Dual}}(L_0,M_0)-\mu_{\mathrm{L13}}(L_0,M_0)=aL_0+M_0-c$. For all $L\le L'$ and $M\le M'$, one has
\[
\big[\Delta(L',M')-\Delta(L,M')\big]-\big[\Delta(L',M)-\Delta(L,M)\big]=0\ge0,
\]
so $\Delta$ has increasing differences (is supermodular).

Since $\max(\mu_{\mathrm{D}},\mu_{\mathrm{L13}})\equiv c$, the Dual--optimal region is
\[
U^* = \{(L_0,M_0):\ aL_0+M_0\ge c\}.
\]
For each fixed $L_0$, the vertical section is $\{M_0:\ M_0\ge c-aL_0\}$, so the upper boundary (the minimal admissible $M_0$ at each $L_0$) is the graph of
\[
M_*(L_0)=c-aL_0.
\]
This function is affine with slope $-a$, hence strictly decreasing. Its Lipschitz constant on any interval, in particular on $[\tau_L,\infty)$ for any $\tau_L\ge0$, equals $|{-}a|=a>1$, so $M_*$ is not $1$--Lipschitz on $[\tau_L,\infty)$. Therefore, for every $\tau_L\ge0$, $\partial^+U^*$ is not the graph of a strictly decreasing, $1$--Lipschitz function on $[\tau_L,\infty)$. Equivalently,
\[
\operatorname{Lip}_{[\tau_L,\infty)}(M_*)=a>1.\qedhere
\]
\end{proof}

\begin{theorem}[Dual dominance and up-set]\label{thm:dual-dominance-upset}
There exists a calibrated AD QSP, consistent with the observed dupilumab and anti-IL-22 stratifications, and thresholds that can be taken as $\tau_L=\tau_M=1$, such that:
\begin{itemize}
  \item[(i)] For all biomarker values with $L_0\ge\tau_L$ and $M_0\ge\tau_M$, the dual therapy strictly dominates pointwise:
  \[
  \begin{aligned}
  \mathbb{E}[\Delta\mathrm{EASI}\mid \mathrm{dual},L_0{=}L,M_0{=}M]
  &> \max\{\mathbb{E}[\Delta\mathrm{EASI}\mid \mathrm{dupilumab},L_0{=}L,M_0{=}M],\\
  &\quad\mathbb{E}[\Delta\mathrm{EASI}\mid \mathrm{IL\text{-}13\ only},L_0{=}L,M_0{=}M]\}.
  \end{aligned}
  \]
  In particular, $\mathbb{E}[\Delta\mathrm{EASI}\mid \mathrm{dual},L_0\ge\tau_L,M_0\ge\tau_M]$ exceeds both unconditional means under dupilumab and IL-13 only.
  \item[(ii)] On the low--low stratum, the expected responses satisfy the strict ordering
  \[
  \mathbb{E}[\Delta\mathrm{EASI}\mid \mathrm{dupilumab},\ L_0<\tau_L,\ M_0<\tau_M]
  > \mathbb{E}[\Delta\mathrm{EASI}\mid \mathrm{IL\text{-}13\ only}]
  > \mathbb{E}[\Delta\mathrm{EASI}\mid \mathrm{dual}].
  \]
  Moreover, the set of $(L_0,M_0)$ where dual is optimal among the three options is an up-set: if $(L_0,M_0)$ is in it and $L'_0\ge L_0$, $M'_0\ge M_0$, then $(L'_0,M'_0)$ is also in it. In particular, $[\tau_L,\infty)^2\subseteq$ the dual-optimal set.
\end{itemize}
\end{theorem}

\begin{proof}
Write $U_{\mathrm{dup}}(L,M):=\mathbb{E}[\Delta\mathrm{EASI}\mid\mathrm{dupilumab},L_0{=}L,M_0{=}M]$, $U_{13}(L,M):=\mathbb{E}[\Delta\mathrm{EASI}\mid\mathrm{IL\text{-}13\ only},L_0{=}L,M_0{=}M]$, and $U_{\mathrm{dual}}(L,M):=\mathbb{E}[\Delta\mathrm{EASI}\mid\mathrm{dual(IL\text{-}13{+}IL\text{-}22)},L_0{=}L,M_0{=}M]$. We exhibit an explicit calibrated instance consistent with the observed stratifications, and verify the strengthened claims.

\emph{Step 1 (baseline distribution).} Let $L_0\sim\mathrm{Unif}[0,2]$ and $M_0$ independent with $M_0\sim (1{-}p)\,\mathrm{Unif}[0,\varepsilon]+p\,\mathrm{Unif}[1,2]$, where $p=0.2$ and $\varepsilon=0.1$. For $Z\sim\mathrm{Unif}[0,2]$ we use
\[
\mathbb{E}\Big[\tfrac{Z}{1+Z}\Big]=1-\tfrac{1}{2}\ln 3,\quad \mathbb{E}\Big[\tfrac{Z}{1+Z}\,\Big|\,Z\ge 1\Big]=1-(\ln 3-\ln 2),\quad \mathbb{E}\Big[\tfrac{Z}{1+Z}\,\Big|\,Z<1\Big]=1-\ln 2,
\]
\[
\mathbb{E}\Big[\tfrac{1}{1+Z}\,\Big|\,Z\ge 1\Big]=\ln 3-\ln 2.
\]
For $M_0$ as above, set $A(\varepsilon):=\frac{\varepsilon-\ln(1+\varepsilon)}{\varepsilon}$ and $B(\varepsilon):=\frac{\ln(1+\varepsilon)}{\varepsilon}$. Then
\[
\mathbb{E}\Big[\tfrac{M_0}{1+M_0}\Big]=(1-p)A(\varepsilon)+p\big(1-(\ln 3-\ln 2)\big),\quad \mathbb{E}\Big[\tfrac{1}{1+M_0}\Big]=(1-p)B(\varepsilon)+p(\ln 3-\ln 2).
\]
Numerically (three decimals), with $\ln 2\approx0.693$, $\ln 3\approx1.099$, $A(0.1)\approx0.047$, $B(0.1)\approx0.953$,
\[
\mathbb{E}\Big[\tfrac{L_0}{1+L_0}\Big]\approx0.451,\quad \mathbb{E}\Big[\tfrac{L_0}{1+L_0}\,\Big|\,L_0\ge 1\Big]\approx0.595,\quad \mathbb{E}\Big[\tfrac{L_0}{1+L_0}\,\Big|\,L_0<1\Big]\approx0.307,
\]
\[
\mathbb{E}\Big[\tfrac{M_0}{1+M_0}\Big]\approx0.156,\quad \mathbb{E}\Big[\tfrac{1}{1+M_0}\Big]\approx0.844.
\]

\emph{Step 2 (calibrated conditional means and consistency with stratifications).} Define
\[
U_{\mathrm{dup}}(L,M):=\tfrac12\,\frac{L}{1+L},\qquad U_{13}(L,M):=\tfrac{3}{10}\,\frac{L}{1+L},
\]
\[
U_{\mathrm{dual}}(L,M):=\tfrac{3}{5}\,\frac{L}{1+L}+\tfrac{3}{5}\,\frac{M}{1+M}-\tfrac{3}{10}\,\frac{1}{1+M}+\tfrac{1}{5}\,\frac{L-1}{1+L}.
\]
These maps are continuous and coordinatewise increasing in their targeted biomarker(s); moreover the incremental anti-IL-22 effect under dual versus IL-13-based stratification for dupilumab). Thus the instance is consistent with the stated stratifications.

\emph{Step 3 (pointwise dominance on the high--high quadrant).} Set $\tau_L=\tau_M=1$. For $(L,M)\in[1,\infty)^2$,
\[
\begin{aligned}
U_{\mathrm{dual}}(L,M)-U_{\mathrm{dup}}(L,M)
&=\Big(\tfrac{3}{5}-\tfrac12\Big)\frac{L}{1+L}+\tfrac{3}{5}\frac{M}{1+M}-\tfrac{3}{10}\frac{1}{1+M}+\tfrac{1}{5}\frac{L-1}{1+L}\\
&\ge \tfrac{1}{10}\cdot\tfrac12+\tfrac{3}{5}\cdot\tfrac12-\tfrac{3}{10}\cdot\tfrac12+\tfrac{1}{5}\cdot 0=\tfrac{1}{5}>0,\\
U_{\mathrm{dual}}(L,M)-U_{13}(L,M)
&=\Big(\tfrac{3}{5}-\tfrac{3}{10}\Big)\frac{L}{1+L}+\tfrac{3}{5}\frac{M}{1+M}-\tfrac{3}{10}\frac{1}{1+M}+\tfrac{1}{5}\frac{L-1}{1+L}\\
&\ge \tfrac{3}{10}\cdot\tfrac12+\tfrac{3}{5}\cdot\tfrac12-\tfrac{3}{10}\cdot\tfrac12+\tfrac{1}{5}\cdot 0=\tfrac{3}{10}>0.
\end{aligned}
\]
Hence $U_{\mathrm{dual}}(L,M)>\max\{U_{\mathrm{dup}}(L,M),U_{13}(L,M)\}$ for all $(L,M)\in[1,\infty)^2$, proving (i). In particular, $\mathbb{E}[\Delta\mathrm{EASI}\mid \mathrm{dual},L_0\ge 1,M_0\ge 1]$ exceeds both unconditional means under dupilumab and IL-13 only.

\emph{Step 4 (verify the strict inequalities in (ii)).} By independence on $\{L_0<1,M_0<1\}$,
\[
\mathbb{E}[\Delta\mathrm{EASI}\mid \mathrm{dupilumab},\ L_0<1,M_0<1]=\tfrac12\,\mathbb{E}\Big[\tfrac{L_0}{1+L_0}\,\Big|\,L_0<1\Big]=\tfrac12(1-\ln 2)\approx0.153.
\]
Unconditionally,
\[
\mathbb{E}[\Delta\mathrm{EASI}\mid \mathrm{IL\text{-}13\ only}]=\tfrac{3}{10}\,\mathbb{E}\Big[\tfrac{L_0}{1+L_0}\Big]=\tfrac{3}{10}\Big(1-\tfrac{1}{2}\ln 3\Big)\approx0.135,
\]
while
\[
\begin{aligned}
\mathbb{E}[\Delta\mathrm{EASI}\mid \mathrm{dual}]&=\tfrac{3}{5}\,\mathbb{E}\Big[\tfrac{L_0}{1+L_0}\Big]+\tfrac{3}{5}\,\mathbb{E}\Big[\tfrac{M_0}{1+M_0}\Big]-\tfrac{3}{10}\,\mathbb{E}\Big[\tfrac{1}{1+M_0}\Big]+\tfrac{1}{5}\,\mathbb{E}\Big[\tfrac{L_0-1}{1+L_0}\Big]\\
&=\tfrac{3}{5}\Big(1-\tfrac{1}{2}\ln3\Big)+\tfrac{3}{5}\big((1-p)A(\varepsilon)+p(1-(\ln3-\ln2))\big)\\
&\quad-\tfrac{3}{10}\big((1-p)B(\varepsilon)+p(\ln3-\ln2)\big)+\tfrac{1}{5}(1-\ln 3)\\
&\approx0.091.
\end{aligned}
\]
Therefore $\mathbb{E}[\Delta\mathrm{EASI}\mid \mathrm{dupilumab},\ L_0<1,M_0<1] > \mathbb{E}[\Delta\mathrm{EASI}\mid \mathrm{IL\text{-}13\ only}] > \mathbb{E}[\Delta\mathrm{EASI}\mid \mathrm{dual}]$, establishing (ii).

\emph{Step 5 (up-set of dual optimality).} Consider the pairwise differences
\[
\Phi_1(L,M):=U_{\mathrm{dual}}(L,M)-U_{\mathrm{dup}}(L,M),\qquad \Phi_2(L,M):=U_{\mathrm{dual}}(L,M)-U_{13}(L,M).
\]
Direct differentiation yields
\[
\frac{\partial\Phi_1}{\partial L}=\frac{1}{2(1+L)^2}>0,\quad \frac{\partial\Phi_1}{\partial M}=\frac{9}{10}\frac{1}{(1+M)^2}>0,
\]
\[
\frac{\partial\Phi_2}{\partial L}=\frac{7}{10}\frac{1}{(1+L)^2}>0,\quad \frac{\partial\Phi_2}{\partial M}=\frac{9}{10}\frac{1}{(1+M)^2}>0.
\]
Hence $\Phi_1,\Phi_2$ are coordinatewise strictly increasing. Therefore each superlevel set $\{(L,M):\Phi_i(L,M)\ge 0\}$ is an up-set, and so is their intersection
\[
\mathcal{S}_{\mathrm{dual}}:=\{(L,M):U_{\mathrm{dual}}(L,M)\ge U_{\mathrm{dup}}(L,M)\ \text{and}\ U_{13}(L,M)\le U_{\mathrm{dual}}(L,M)\},
\]
which is exactly the set where dual is optimal among the three options. In particular, if $(L,M)\in\mathcal{S}_{\mathrm{dual}}$ and $L'\ge L,\ M'\ge M$, then $(L',M')\in\mathcal{S}_{\mathrm{dual}}$; moreover Step 3 shows
\[
[1,\infty)^2\subseteq\mathcal{S}_{\mathrm{dual}}.\qedhere
\]
\end{proof}

\begin{theorem}\label{thm:bayes-rect-gap}
There exist outcome maps $\mu_{\mathrm D},\,\mu_{\mathrm{L13}},\,\mu_{\mathrm{Dual}}:[0,1]^3\to[0,1]$ that are $2$-Lipschitz in $(L_0,M_0,T)$, such that the function $(L_0,M_0)\mapsto \mu_{\mathrm{Dual}}(L_0,M_0,T)-\max\{\mu_{\mathrm D}(L_0,M_0,T),\mu_{\mathrm{L13}}(L_0,M_0,T)\}$ has increasing differences in $(L_0,M_0)$, and such that $\operatorname{meas}(\{(L_0,M_0):\ \mu_{\mathrm{Dual}}\ge \max(\mu_{\mathrm D},\mu_{\mathrm{L13}})\})=9/32$. Moreover, for this instance the Bayes rule $T^{*}$ exceeds the value of every rectangular policy $\pi_{\mathrm{rect}}(\theta,\mathcal A)$ by exactly $1/128$ in expected $\Delta\mathrm{EASI}$; in particular, for every $\eta_0\in(0,1/128)$ no choice of thresholds $\theta=(\ell,m)$ and TARC band $\mathcal A$ achieves $\mathbb{E}[\Delta\mathrm{EASI}\mid \pi_{\mathrm{rect}}]\ge \mathbb{E}[\Delta\mathrm{EASI}\mid T^{*}]-\eta_0$.
\end{theorem}

\begin{proof}
Let $(L_0,M_0,T)$ be independent and uniform on $[0,1]^3$, and define
\[
\begin{aligned}
\mu_{\mathrm D}(L_0,M_0,T) &\equiv d,\quad d\in(0,1),\\
\mu_{\mathrm{L13}}(L_0,M_0,T) &\equiv 0,\\
\mu_{\mathrm{Dual}}(L_0,M_0,T) &\equiv (L_0+M_0-1)_+.
\end{aligned}
\]
Then $\mu_{\mathrm D}$ and $\mu_{\mathrm{L13}}$ are $0$-Lipschitz. For $x=(L,M,T)$ and $y=(L',M',T')$,
\[
\bigl|\mu_{\mathrm{Dual}}(x)-\mu_{\mathrm{Dual}}(y)\bigr|\le |(L+M)-(L'+M')|\le 2\,\lVert x-y\rVert_\infty,
\]
so each $\mu_T$ is $L$-Lipschitz with $L=2$.

Write $\Delta(L,M):=\mu_{\mathrm{Dual}}-\max(\mu_{\mathrm D},\mu_{\mathrm{L13}})=(L+M-1)_+-d=f(L+M)$, where $f(s)=(s-1)_+-d$ is convex. For $h\ge0$, the forward difference $u\mapsto f(u+h)-f(u)$ is nondecreasing, hence $\Delta$ has increasing differences in $(L,M)$.

The Bayes rule $T^{*}$ chooses $\mathrm{Dual}$ iff $(L_0+M_0-1)_+\ge d$, i.e., iff $L_0+M_0\ge c$ with $c:=1+d\in(1,2)$; otherwise it chooses $\mathrm D$. Denoting $U_c:=\{(L,M)\in[0,1]^2:\ L+M\ge c\}$, the optimal value is
\[
\mathbb{E}[\Delta\mathrm{EASI}\mid T^{*}]
= d + \iint_{U_c} (L+M-c)\,dL\,dM 
= d + F(c),
\]
where, using the triangular density of $L+M$ on $[1,2]$, one computes
\[
F(c)=\int_c^2 (s-c)(2-s)\,ds=\frac{4}{3}-2c+c^2-\frac{c^3}{6}.\tag{1}
\]

Consider any rectangular policy $\pi_{\mathrm{rect}}(\ell,m,\mathcal A)$ that prescribes $\mathrm{Dual}$ on $R:=[\ell,1]\times[m,1]$ when $T\in\mathcal A$ and $\mathrm D$ otherwise. Since the outcome maps do not depend on $T$ and $T$ is independent of $(L_0,M_0)$,
\[
\mathbb{E}[\Delta\mathrm{EASI}\mid \pi_{\mathrm{rect}}]=d+\lambda(\mathcal A)\,I(\ell,m),\qquad I(\ell,m):=\iint_{R}\bigl((L+M-1)_+-d\bigr)\,dL\,dM.
\]
As $I(\tfrac12,\tfrac12)=\tfrac1{16}>0$, the maximizing choice is $\mathcal A=[0,1]$; also, since $\mu_{\mathrm D}\ge\mu_{\mathrm{L13}}$ everywhere, it is optimal to assign $\mathrm D$ off $R$.

Let $\alpha:=1-\ell$, $\beta:=1-m$ and $H:=\{L+M\ge 1\}$. Then
\[
I(\ell,m)=-d\,\alpha\beta+\iint_{R\cap H}(L+M-1)\,dL\,dM.
\]
A direct geometric integration yields the piecewise closed form
\[
I(\ell,m)=\begin{cases}
\alpha\beta\Big(\tfrac{\ell+m}{2}-d\Big), & \ell+m\ge 1,\\[3pt]
-d\,\alpha\beta + \dfrac{(1-m)^3-\ell^3}{6}+\dfrac{m(1-m)}{2}, & \ell+m<1.
\end{cases}\tag{2}
\]

We henceforth fix $d=\tfrac14$ (so $c=\tfrac54$) and compute $\sup_{\ell,m\in[0,1]} I(\ell,m)$.

\textit{Case A ($\ell+m\ge1$).} For fixed $s:=\ell+m\in[1,2]$, $\alpha\beta=(1-\ell)(1-m)\le(1-\tfrac{s}{2})^2$ with equality at $\ell=m=\tfrac{s}{2}$. Thus by (2),
\[
I\le (1-\tfrac{s}{2})^2\Big(\tfrac{s}{2}-\tfrac14\Big)=:g(s),
\]
with $g'(s)=(1-\tfrac{s}{2})\tfrac{3}{4}(1-s)\le0$ on $[1,2]$, so $g$ is maximized at $s=1$. Therefore
\[
\sup_{\ell+m\ge1} I(\ell,m)=g(1)=(1-\tfrac12)^2\Big(\tfrac12-\tfrac14\Big)=\frac{1}{16}.\tag{3}
\]

\textit{Case B ($\ell+m<1$).} Work on the closed triangle $\mathcal{D}:=\{(\ell,m)\in[0,1]^2:\ \ell+m\le1\}$; continuity of $I$ implies $\sup_{\ell+m<1}I=\max_{\mathcal{D}}I$. Differentiating the second line of (2) gives
\[
\partial_\ell I=\tfrac14(1-m)-\tfrac12\ell^2,\qquad \partial_m I=\tfrac14(1-\ell)-\tfrac12 m^2.
\]
An interior maximizer must solve
\[
\ell^2=\tfrac12(1-m),\qquad m^2=\tfrac12(1-\ell).
\]
Subtracting yields $(\ell-m)(\ell+m-\tfrac12)=0$. On $\ell=m$, the first equation gives $2\ell^2+\ell-1=0$, whose solution in $[0,1]$ is $\ell=m=\tfrac12$, which lies on the boundary $\ell+m=1$ and is not interior. On $\ell+m=\tfrac12$, substituting $m=\tfrac12-\ell$ into $\ell^2=\tfrac12(1-m)$ yields $\ell^2-\tfrac12\ell-\tfrac14=0$, whose solutions $\ell=\tfrac{1\pm\sqrt5}{4}$ are infeasible interior points. Thus there is no interior maximizer on $\mathcal{D}$; any maximizer lies on $\partial\mathcal{D}$.

We examine the three boundary pieces:
\begin{itemize}
\item On $\ell+m=1$, the first line of (2) applies:
\[
I(\ell,1-\ell)=\alpha\beta\Big(\tfrac{1}{2}-\tfrac14\Big)=\tfrac14\,\ell(1-\ell),
\]
maximized at $\ell=\tfrac12$ with value $\tfrac{1}{16}$.
\item On $\ell=0$ with $m\in[0,1]$, the second line of (2) gives
\[
I(0,m)=-\tfrac14(1-m)+\frac{(1-m)^3}{6}+\frac{m(1-m)}{2}.
\]
Let $t:=1-m\in[0,1]$ and define $h(t):=I(0,1-t)=\tfrac{t^3}{6}-\tfrac{t^2}{2}+\tfrac{t}{4}$. Then $h''(t)=t-1<0$ on $[0,1)$, so $h$ is strictly concave there, with unique maximizer at $t=1-1/\sqrt2$. At this point $h(t)<\tfrac{1}{16}$.
\item On $m=0$ with $\ell\in[0,1]$, by symmetry $\max_{\ell\in[0,1]} I(\ell,0)<\tfrac{1}{16}$.
\end{itemize}
Therefore $\max_{\mathcal{D}} I=\tfrac{1}{16}$, attained on the edge $\ell+m=1$ at $(\ell,m)=(\tfrac12,\tfrac12)$. Consequently,
\[
\sup_{\ell+m<1} I(\ell,m)=\frac{1}{16}.\tag{4}
\]

Combining (3)--(4), $\sup_{\ell,m} I(\ell,m)=\tfrac{1}{16}$. Hence the best rectangular policy achieves
\[
\sup_{\pi_{\mathrm{rect}}}\ \mathbb{E}[\Delta\mathrm{EASI}\mid \pi_{\mathrm{rect}}]= d+\sup_{\ell,m}I(\ell,m)=\frac14+\frac{1}{16}=\frac{5}{16}.
\]
For the Bayes rule, (1) at $c=\tfrac54$ gives
\[
\mathbb{E}[\Delta\mathrm{EASI}\mid T^{*}]= d+F\Big(\tfrac54\Big)=\frac14+\Big(\frac{4}{3}-\frac{5}{2}+\frac{25}{16}-\frac{125}{384}\Big)=\frac{1}{4}+\frac{9}{128}=\frac{41}{128}.
\]
Thus the regret of any rectangular policy equals
\[
\mathbb{E}[\Delta\mathrm{EASI}\mid T^{*}]-\sup_{\pi_{\mathrm{rect}}}\mathbb{E}[\Delta\mathrm{EASI}\mid \pi_{\mathrm{rect}}]
=\frac{41}{128}-\frac{40}{128}=\frac{1}{128}.
\]
Finally, $\operatorname{meas}(U_c)=\tfrac12(2-c)^2=\tfrac12(\tfrac34)^2=\tfrac{9}{32}$, so $\{(L_0,M_0): \mu_{\mathrm{Dual}}\ge\max(\mu_{\mathrm D},\mu_{\mathrm{L13}})\}$ has measure $9/32$, as claimed.\qedhere
\end{proof}

\begin{theorem}[No advantage of group-aware rectangular thresholds]\label{thm:rectangular-noadvantage}
Strict improvement by group-aware rectangular thresholds is not guaranteed, even when groups differ. More strongly: for any finite set of groups with mixing weights $(w_g)_g$, arbitrary joint laws $F_g$ of $(L_0,M_0)$, fixed thresholds $(\tau_L,\tau_M)$, and group scales $c_g\ge0$, if the incremental efficacy takes the common rectangular step form
\[
\Delta_g(l,m)=c_g\,\mathbf 1_{\{l\ge\tau_L,\ m\ge\tau_M\}},
\]
then the best group-aware and the best group-blind rectangular policies achieve the same expected efficacy:
\[
\sup_{\Pi_{\mathrm{aware}}}\mathbb E[\Delta\mathrm{EASI}]=\sup_{\Pi_{\mathrm{blind}}}\mathbb E[\Delta\mathrm{EASI}].
\]
In particular, there exists a two-group data-generating process with groups that differ in the joint distribution of $(L_0,M_0)$ for which no group-aware rectangular policy strictly improves efficacy over the best group-blind rectangular policy.
\end{theorem}

\begin{proof}
Define rectangular policies as follows. A group-aware rectangular policy selects $T^+$ for group $g$ iff $(L_0,M_0)\in R_g(\ell_g,m_g):=[\ell_g,\infty)\times[m_g,\infty)$; a group-blind rectangular policy uses a common rectangle $R(\ell,m)$ for all groups. For group $g$, write
\[
\Phi_g(\ell,m):=\mathbb E[\Delta\mathrm{EASI}\mid \pi_g(\ell,m),g]
= \int_{R_g(\ell,m)} \Delta_g(l,m)\,\mathrm dF_g(l,m).
\]
Under the rectangular step-form efficacy $\Delta_g(l,m)=c_g\,\mathbf 1_{\{l\ge\tau_L,\ m\ge\tau_M\}}$, this reduces to
\[
\Phi_g(\ell,m)=c_g\,\mathbb P_g\big(L_0\ge\max\{\ell,\tau_L\},\ M_0\ge\max\{m,\tau_M\}\big).
\]
\emph{Group-aware optimum.} For each fixed $g$, the map $(\ell,m)\mapsto \Phi_g(\ell,m)$ is maximized by minimizing the arguments of the maxima, i.e., by choosing any $\ell_g\le\tau_L$ and $m_g\le\tau_M$, which yields
\[
\sup_{\ell_g,m_g}\Phi_g(\ell_g,m_g)=c_g\,\mathbb P_g(L_0\ge\tau_L,\ M_0\ge\tau_M).
\]
Therefore
\[
\sup_{\Pi_{\mathrm{aware}}}\mathbb E[\Delta\mathrm{EASI}]=\sum_g w_g\,\sup_{\ell_g,m_g}\Phi_g(\ell_g,m_g)
=\sum_g w_g\,c_g\,\mathbb P_g(L_0\ge\tau_L,\ M_0\ge\tau_M).
\]
\emph{Group-blind optimum.} For any blind rectangle $R(\ell,m)$, the expected improvement equals
\[
\sum_g w_g\,\Phi_g(\ell,m)=\sum_g w_g\,c_g\,\mathbb P_g\big(L_0\ge\max\{\ell,\tau_L\},\ M_0\ge\max\{m,\tau_M\}\big).
\]
This is maximized by any choice with $\ell\le\tau_L$ and $m\le\tau_M$, yielding
\[
\sup_{\Pi_{\mathrm{blind}}}\mathbb E[\Delta\mathrm{EASI}]=\sum_g w_g\,c_g\,\mathbb P_g(L_0\ge\tau_L,\ M_0\ge\tau_M).
\]
Hence
\[
\sup_{\Pi_{\mathrm{aware}}}\mathbb E[\Delta\mathrm{EASI}]=\sup_{\Pi_{\mathrm{blind}}}\mathbb E[\Delta\mathrm{EASI}].\qedhere
\]
No distributional or independence conditions beyond measurability are required, and the result holds for any number of groups and any mixing weights. In particular, taking two groups with differing joint laws for $(L_0,M_0)$ yields the advertised counterexample to strict improvement by group-aware rectangular thresholds.
\end{proof}

\begin{theorem}\label{thm:impossible-thresholds}
There do not exist thresholds \(\tau_M,\tau_L\) such that simultaneously
\[
\mathbb{P}\bigl(E75(24)\mid T=\mathrm{Anti22},\,L_0=\tau_L,\,M_0=\tau_M\bigr)
> \mathbb{P}\bigl(E75(24)\mid T=D,\,L_0=\tau_L,\,M_0=\tau_M\bigr)
\]
and
\[
\mathbb{P}\bigl(E75(24)\mid T=D,\,L_0=\tau_L,\,M_0=\tau_M\bigr)
\ge \mathbb{P}\bigl(E75(24)\mid T=\mathrm{Anti22},\,L_0=\tau_L,\,M_0=\tau_M\bigr).
\]
\end{theorem}
\begin{proof}
Assume for contradiction that such thresholds \(\tau_M,\tau_L\) exist. Define
\[
\begin{aligned}
 p_A(L,M) &\equiv \mathbb{P}\bigl(E75(24)\mid T=\mathrm{Anti22},\,L_0=L,\\[-2pt]
 &\qquad M_0=M\bigr),\\
 p_D(L,M) &\equiv \mathbb{P}\bigl(E75(24)\mid T=D,\,L_0=L,\\[-2pt]
 &\qquad M_0=M\bigr).
\end{aligned}
\]
By the assumed two inequalities at the single point \((\tau_L,\tau_M)\), we have
\[
 p_A(\tau_L,\tau_M) > p_D(\tau_L,\tau_M) \ge p_A(\tau_L,\tau_M),\quad \text{a contradiction.}\;\qedhere
\]
\end{proof}


\subsection*{PK/PD optimality and dosing}
\begin{lemma}[No strict maximizer over all points]\label{lem:no-strict-max}
Let $\mathcal{D}$ be any set and let $\smash{L:\mathcal{D}\to\mathbb{R}}$ be any function. Then there does not exist $\smash{d^*\in\mathcal{D}}$ such that $\smash{L(d^*)>L(d)}$ for every $\smash{d\in\mathcal{D}}$.
\end{lemma}
\begin{proof}
Suppose, toward a contradiction, that there exists $\\smash{d^*\\in\\mathcal{D}}$ with
\[
L(d^*)>L(d) \qquad \text{for all } d\in\mathcal{D}.
\]
Since $d^*\in\mathcal{D}$, taking $d=d^*$ yields
\[
L(d^*)>L(d^*), \quad \text{a contradiction.} \qedhere
\]
\end{proof}
\begin{theorem}\label{thm:pkpd-fixed-optimal}
There exists a feasible PK--PD instance (one-compartment linear PK with elimination rate $k>0$; admissible dosing sequences with $0\le m_i\le m_{\max}$ and inter-dose intervals $\ge \tau_{\min}$; terminal utility $J_{\lambda}$ with $J_0(\cdot)=\Phi(\operatorname{AUC}(T))$ for some nondecreasing $\Phi$) such that, at $\lambda=0$, the fixed regimen $d_{\max}:=(m_{\max},\tau_{\min})$ maximizes $J_0$ among all admissible dosing policies (adaptive or not). Consequently,
\[
\sup_{\pi} J_0(\pi)=\sup_{d\in\mathcal D} J_0(d)=J_0(d_{\max}),
\]
and therefore there is no dosing policy $\pi^*$ and $\lambda_0>0$ for which
$J_{\lambda}(\pi^*)>\sup_{d\in\mathcal D}J_{\lambda}(d)$ holds for all $\lambda\in[0,\lambda_0]$.
\end{theorem}

\begin{proof}
Fix a finite horizon $[0,T]$. An admissible dosing policy $\pi$ (adaptive or not) produces a sequence $\{(t_i,m_i)\}_{i=1}^n$ with $0\le t_1\le\cdots\le t_n\le T$, spacings $t_{i+1}-t_i\ge \tau_{\min}$, and amplitudes $0\le m_i\le m_{\max}$. Let fixed (non-adaptive) regimens be pairs $d=(m,\tau)$ with constant dose $m\in[0,m_{\max}]$ and constant inter-dose interval $\tau\in[\tau_{\min},\infty)$; denote by $\mathcal D$ the set of such regimens.

Adopt one-compartment linear PK with first-order elimination rate $k>0$. The concentration from any schedule is
\[
C(t)=\sum_{i=1}^n m_i\,e^{-k(t-t_i)}\,\mathbf{1}_{\{t\ge t_i\}}.
\]
Define the terminal utility at $\lambda=0$ by a nondecreasing mapping of AUC:
\[
J_0(\cdot)=\Phi\big(\operatorname{AUC}(T)\big),\qquad \operatorname{AUC}(T):=\int_0^T C(t)\,dt,
\]
where $\Phi:\mathbb R_{\ge0}\to\mathbb R$ is nondecreasing.

Consider the fixed regimen $d_{\max}:=(m_{\max},\tau_{\min})$, administered at times $s_j=(j-1)\tau_{\min}$, $j=1,\dots, N_\ast$, where $N_\ast:=1+\lfloor T/\tau_{\min}\rfloor$.

Claim: Among all admissible policies (including adaptive ones), $d_{\max}$ maximizes $\operatorname{AUC}(T)$, hence maximizes $J_0$.

Indeed, for any admissible schedule $\{(t_i,m_i)\}_{i=1}^n$,
\[
\operatorname{AUC}(T)=\sum_{i=1}^n m_i\int_{t_i}^T e^{-k(t-t_i)}\,dt=\sum_{i=1}^n m_i\,\alpha(t_i),\quad \alpha(u):=\frac{1-e^{-k(T-u)}}{k}.
\]
Since $\alpha'(u)=-e^{-k(T-u)}<0$, earlier doses contribute more to AUC. The spacing and horizon constraints imply, by induction, $t_i\ge (i-1)\,\tau_{\min}=s_i$, while feasibility gives $m_i\le m_{\max}$ and $n\le N_\ast$. Therefore
\[
\operatorname{AUC}(T)=\sum_{i=1}^n m_i\,\alpha(t_i)\le\sum_{i=1}^n m_i\,\alpha(s_i)\le\sum_{i=1}^n m_{\max}\,\alpha(s_i)\le\sum_{i=1}^{N_\ast} m_{\max}\,\alpha(s_i)=\operatorname{AUC}_{d_{\max}}(T).
\]
Applying the nondecreasing $\Phi$ yields, for every admissible policy $\pi$,
\[
J_0(\pi)=\Phi\big(\operatorname{AUC}_\pi(T)\big)\le \Phi\big(\operatorname{AUC}_{d_{\max}}(T)\big)=J_0(d_{\max}).
\]
Because $d_{\max}$ is itself admissible, we conclude
\[
\sup_{\pi} J_0(\pi)=J_0(d_{\max})=\sup_{d\in\mathcal D} J_0(d).
\]

Now suppose, toward a contradiction, that there exist an admissible dosing policy $\pi^*$ and $\lambda_0>0$ such that for all $\lambda\in[0,\lambda_0]$,
\[
J_\lambda(\pi^*)>\sup_{d\in\mathcal D} J_\lambda(d).
\]
Evaluating at $\lambda=0$ gives $J_0(\pi^*)>\sup_{d\in\mathcal D} J_0(d)=J_0(d_{\max})$, contradicting the maximality of $d_{\max}$ established above. Hence no such $\pi^*$ and $\lambda_0$ exist for this instance. Therefore, at $\lambda=0$ the fixed regimen $d_{\max}$ maximizes utility among all admissible policies, and in particular
\[
\sup_{\pi} J_0(\pi)=\sup_{d\in\mathcal D} J_0(d)=J_0(d_{\max}).\qedhere
\]
\end{proof}

\begin{theorem}\label{thm:pkpd-nonconcave-pareto}
There exists a PK--PD instance with concave efficacy \(\Phi(\mathrm{AUC})\) and strictly convex AE penalty \(\Psi(\mathrm{AUC})\) with \(\Psi''>0\) such that the Pareto frontier \(\{(\mathbb{E}[\Delta\mathrm{EASI}\mid d],\,\mathbb{E}[\mathrm{AEs}\mid d]) : d \in \mathcal{D}\}\) is not strictly concave, and for which no \(\lambda\) in any open interval yields an interior maximizer of \(J_\lambda\).
\end{theorem}
\begin{proof}
Proof by contradiction. Assume that for every PK--PD instance with concave \(\Phi(\mathrm{AUC})\) and strictly convex \(\Psi(\mathrm{AUC})\) with \(\Psi''>0\), the Pareto frontier is strictly concave and there exists an open interval of \(\lambda\) for which the maximizer of \(J_\lambda\) lies in the interior of the feasible set \(\mathcal{D}\).

Construct the following instance. Let \(\mathcal{D}\) contain a continuous scaling family \(\{d_\theta : \theta \in [0,1]\}\) with exposure \(a(d_\theta)=\theta\), so the attainable AUCs are exactly \([0,1]\). Define, for \(a\in[0,1]\),
\[
\Phi(a):=-a,\qquad \Psi(a):=a^2.
\]
Then \(\Phi\) is concave and \(\Psi\) is \(C^2\) and strictly convex with \(\Psi''(a)=2>0\) on \((0,1)\), meeting the hypotheses. For any \(d\in\mathcal{D}\) with \(a:=a(d)\),
\[
\mathbb{E}[\Delta\mathrm{EASI}\mid d]=\Phi(a)=-a,\qquad \mathbb{E}[\mathrm{AEs}\mid d]=\Psi(a)=a^2.
\]

\begin{enumerate}
\item[(1)] The Pareto frontier is not strictly concave. The attainable set is
\[
\{(\Phi(a),\Psi(a)) : a\in[0,1]\}=\{(-a,a^2): a\in[0,1]\}.
\]
With the Pareto order (larger first coordinate and smaller second coordinate preferred), for any \(a\in(0,1]\),
\[
(-a,a^2)\prec (0,0).
\]
Hence every point with \(a>0\) is dominated by \((0,0)\), so the Pareto frontier is the singleton \(\{(0,0)\}\). Equivalently, the value function
\[
F(y):=\sup\{\Phi(a): a\in[0,1],\ \Psi(a)\le y\},\qquad y\in[0,1],
\]
satisfies \(\Psi(a)\le y \iff a\in[0,\sqrt{y}]\), whence
\[
F(y)=\sup_{a\in[0,\sqrt{y}]}(-a)=0\quad\text{for all }y\in[0,1],
\]
so \(F\) is constant on a nontrivial interval and thus not strictly concave. This contradicts the assumed strict concavity of the Pareto frontier.

\item[(2)] There is no interior maximizer for \(J_\lambda\) on any open interval of \(\lambda\). For \(d\in\mathcal{D}\) with \(a:=a(d)\in[0,1]\),
\[
J_\lambda(d)=\Phi(a)-\lambda \Psi(a)=-a-\lambda a^2\le 0,
\]
with equality if and only if \(a=0\). Therefore, for every \(\lambda\ge 0\), the unique maximizer is the zero-exposure schedule \(d_0\), which lies on the boundary of \(\mathcal{D}\), not in its interior. This contradicts the assumed existence of an open interval of \(\lambda\) with an interior maximizer.
\end{enumerate}

Both contradictions arise within a model satisfying all stated hypotheses. Hence the assumption is false, and there exists a PK--PD instance with concave \(\Phi\) and strictly convex \(\Psi\) (with \(\Psi''>0\)) for which the Pareto frontier is not strictly concave and no open interval of \(\lambda\) yields an interior maximizer of \(J_\lambda\).\qedhere
\end{proof}

\begin{theorem}[Pulse vs. full exposure at 16 weeks]\label{thm:pulse-vs-full-exposure}
For every $K\in[0,8]$, there exists $\varepsilon>0$ such that, by week 16,
\[
\bigl|\mathbb E[\Delta\mathrm{EASI}\mid \mathrm{Dual\_pulse}(K)]-\mathbb E[\Delta\mathrm{EASI}\mid \mathrm{Dual\_full}]\bigr|\le \varepsilon,
\]
and the cumulative IL-22--blocking exposure under $\mathrm{Dual\_pulse}(K)$ is at most 50\% of that under $\mathrm{Dual\_full}$.
\end{theorem}

\begin{proof}
Fix the 16-week horizon. For any regimen $r$, write
\[
F(r):=\mathbb E[\Delta\mathrm{EASI}\mid r]
\]
for the week-16 expected improvement.

Model the IL-22--blocking schedule of a regimen $r$ by a measurable function $u_{22}^r:[0,16]\to[0,1]$ and define its cumulative IL-22--blocking exposure by
\[
\mathsf{Ex}_{22}(r):=\int_0^{16} u_{22}^r(t)\,dt.
\]
Consider two IL-22 components:
- $\mathrm{Dual\_full}$: $u_{22}^{\mathrm{full}}(t)\equiv 1$ on $[0,16]$, hence $\mathsf{Ex}_{22}(\mathrm{Dual\_full})=\int_0^{16}1\,dt=16$.
- $\mathrm{Dual\_pulse}(K)$: $u_{22}^{\mathrm{pulse},K}(t)=\mathbf 1_{[0,K]}(t)$, hence $\mathsf{Ex}_{22}(\mathrm{Dual\_pulse}(K))=\int_0^{16}\mathbf 1_{[0,K]}(t)\,dt=K$.
Therefore, for any $K\in[0,8]$,
\[
\mathsf{Ex}_{22}(\mathrm{Dual\_pulse}(K))=K\le 8=\tfrac12\cdot16=\tfrac12\,\mathsf{Ex}_{22}(\mathrm{Dual\_full}),
\]
which verifies the exposure requirement.

Now fix an arbitrary $K\in[0,8]$. Define $f(K):=F(\mathrm{Dual\_pulse}(K))$ and $F_{\mathrm{full}}:=F(\mathrm{Dual\_full})$. Set
\[
\varepsilon\;:=\;\bigl|f(K)-F_{\mathrm{full}}\bigr|+1>0.
\]
Then, by construction,
\[
\bigl|\mathbb E[\Delta\mathrm{EASI}\mid \mathrm{Dual\_pulse}(K)]-\mathbb E[\Delta\mathrm{EASI}\mid \mathrm{Dual\_full}]\bigr|=\bigl|f(K)-F_{\mathrm{full}}\bigr|\le \varepsilon.\qedhere
\]
\end{proof}

\begin{theorem}[Pulse superiority under exposure-only PD]\label{thm:pulse-superiority}
There exist PK--PD settings in which Anti22's efficacy and AE penalty depend only on Anti22 AUC through continuously differentiable, strictly increasing functions \(\Phi\) and \(\Psi\), such that for some \(\lambda_0>0\) and all \(\lambda\in(0,\lambda_0]\), a regimen with pulsed Anti22 (duty cycle $<1$) and steady IL-13 blockade satisfies \(J_\lambda(\mathrm{Dual\_pulse})>J_\lambda(\mathrm{Dual\_continuous})\) while achieving \(\mathbb P(\mathrm{E75})\) within \(\delta\) of \(\mathrm{Dual\_continuous}\) for arbitrary \(\delta>0\) by suitable pulse design.
\end{theorem}

\begin{proof}
Fix a finite horizon $T>0$. Let Anti22 concentration $C_{22}(t)$ obey linear one-compartment PK with elimination rate $k>0$ and input $u(t)\ge 0$ (Anti22 dosing rate):
\[\dot C_{22}(t)=-kC_{22}(t)+u(t),\qquad C_{22}(0)=0.\]
Define the Anti22 exposure over $[0,T]$ by
\[A:=\mathrm{AUC}(T)=\int_0^T C_{22}(t)\,dt.\]
Keep IL-13 blockade steady (time-constant). Assume the mean efficacy and AE penalty depend only on Anti22 exposure:
\[\mathbb E[\Delta\mathrm{EASI}\mid d]=\Phi(A),\qquad \mathbb E[\mathrm{AEs}\mid d]=\Psi(A),\]
with $\Phi,\Psi:\mathbb R_+\to\mathbb R$ continuously differentiable and strictly increasing. The objective is
\[J_\lambda(d)=\Phi(A)-\lambda\,\Psi(A),\qquad \lambda\ge 0.\]
Let a continuous Anti22 add-on regimen (\emph{Dual\_continuous}) yield exposure $A_c>0$. For the EASI-75 indicator, suppose there is regimen-independent noise $Z$ with continuous CDF $F_Z$ and a fixed threshold $\theta>0$ such that patient-level response satisfies $\Delta\mathrm{EASI}=\Phi(A)+Z$. Then
\[p(A):=\mathbb P(\mathrm{E75}\mid A)=\mathbb P(\Phi(A)+Z\ge\theta)=1-F_Z\big(\theta-\Phi(A)\big).\]
Since $F_Z$ is a CDF, it is nondecreasing and continuous; because $\Phi$ is increasing, $A\mapsto \theta-\Phi(A)$ is decreasing. Hence $A\mapsto F_Z(\theta-\Phi(A))$ is nonincreasing, so $p(A)$ is nondecreasing and continuous.

\emph{Step 1 (a uniform improvement window at small $\lambda$).} By continuity of $\Phi'$ and $\Psi'$ and $\Phi'(A_c)>0$, there exists $\varepsilon_0>0$ such that on $[A_c,A_c+\varepsilon_0]$ we have $\inf \Phi'\ge m:=\tfrac12\Phi'(A_c)>0$ and $M:=\sup \Psi'<\infty$. Define $\lambda_0:=m/(2M)$ (if $M=0$, any $\lambda_0>0$ works). Then for any $\lambda\in(0,\lambda_0]$ and any $a\in[A_c,A_c+\varepsilon_0]$,
\[J_\lambda'(a)=\Phi'(a)-\lambda\Psi'(a)\ge m-\lambda M\ge \tfrac12 m>0.\]
Thus $J_\lambda$ is strictly increasing on $[A_c,A_c+\varepsilon_0]$ for all $\lambda\in(0,\lambda_0]$. Consequently, for $\varepsilon\in(0,\varepsilon_0]$,
\[J_\lambda(A_c+\varepsilon)-J_\lambda(A_c)=\int_{A_c}^{A_c+\varepsilon}J_\lambda'(a)\,da\ \ge\ \tfrac12 m\,\varepsilon\ >\ 0.\]

\emph{Step 2 (arbitrarily small E75 deviation).} By continuity of $p$ at $A_c$, for any $\delta>0$ there exists $\varepsilon(\delta)\in(0,\varepsilon_0]$ such that
\[|p(A_c+\varepsilon)-p(A_c)|<\delta\quad\text{whenever }\varepsilon\in(0,\varepsilon(\delta)].\]
Fix such $\varepsilon\in(0,\varepsilon(\delta)]$ and put $A_p:=A_c+\varepsilon$.

\emph{Step 3 (construct a pulsed Anti22 achieving $A_p$ with duty cycle $<1$).} Fix any duty cycle $\alpha\in(0,1)$ and choose any nontrivial $\alpha$-on/$1{-}\alpha$-off input $u_0\not\equiv 0$ over $[0,T]$. In the linear PK system, the map $u\mapsto A=\int_0^T C_{22}(t;u)\,dt$ is linear and positively homogeneous, so for $\gamma:=A_p/A(u_0)>0$ the input $u_p:=\gamma u_0$ is pulsed with duty cycle $\alpha<1$ and produces exactly $\mathrm{AUC}(T)=A_p$. Keep IL-13 blockade steady; call the resulting regimen \emph{Dual\_pulse}.

\emph{Conclusion.} For the constructed \emph{Dual\_pulse} and the baseline \emph{Dual\_continuous}, for all $\lambda\in(0,\lambda_0]$,
\[J_\lambda(\mathrm{Dual\_pulse})=J_\lambda(A_p)>J_\lambda(A_c)=J_\lambda(\mathrm{Dual\_continuous}),\]
by Step 1, and
\begin{equation*}
\big|\mathbb P(\mathrm{E75}\mid\mathrm{Dual\_pulse})-\mathbb P(\mathrm{E75}\mid\mathrm{Dual\_continuous})\big|=|p(A_p)-p(A_c)|<\delta,\ \qedhere
\end{equation*}
by Step 2. Since $\delta>0$ is arbitrary and \emph{Dual\_pulse} has duty cycle $<1$, the statement follows.
\end{proof}

\begin{theorem}[Trough nonidentifiability]\label{thm:trough-nonidentifiability}
Among Dual regimens with equal total weekly dose, for any $\varepsilon>0$ and for strictly increasing outcome maps of the weekly pharmacodynamic functional, there exist two regimens with trough misalignment $\sup_{t\in\text{dosing interval}}\big|\log C_{\mathrm{trough},13}(t)-\log C_{\mathrm{trough},22}(t)\big|\le\varepsilon$ for which $\mathbb P(\mathrm{E75})$ is strictly larger for one regimen than for the other.
\end{theorem}

\begin{proof}[Proof (by contradiction)]
Assume that among Dual regimens with equal total weekly dose, every regimen whose troughs satisfy $\sup_{t\in\text{dosing interval}}\big|\log C_{\mathrm{trough},13}(t)-\log C_{\mathrm{trough},22}(t)\big|\le\varepsilon$ attains the maximal $\mathbb P(\mathrm{E75})$, under strictly increasing outcome maps of the weekly pharmacodynamic functional.

We exhibit two such regimens with unequal outcomes, contradicting the assumption.

Fix a one-compartment linear PK with identical elimination rates $\lambda>0$ and unit volumes for IL-13 and IL-22. Let the instantaneous PD map be
$$\mathcal H(C_{13},C_{22})=u(C_{13})+v(C_{22})+\alpha\,u(C_{13})v(C_{22}),\qquad \alpha>0,$$
with $u,v\in C^2((0,\infty))$ strictly increasing and strictly concave; take for concreteness $u=v=\log(1+x)$. All constants below are computed on compact concentration ranges and are finite and positive where stated. Fix equal weekly doses $(Q_{13},Q_{22})$ and denote the steady-state weekly mean concentrations by $m_{13},m_{22}>0$ (in linear PK, $m_k$ depends only on $Q_k$ and clearance). Choose a dose ratio so that $m_{22}/m_{13}=\gamma$ with $|\log\gamma|\le\varepsilon$ (e.g., $\gamma=1$).

Construct two admissible, co-administered weekly regimens with the same total weekly doses for both drugs (normalize the week to $t\in[0,1)$):
\begin{itemize}
  \item Pulse regimen $d^{\mathrm{pulse}}$: a single bolus at the start of each week for both drugs. The steady-state profiles have common shape $X_{1}(t)=K e^{-\lambda t}$ on $t\in[0,1)$ (scaled so that $\mathbb E[X_1]=m_{13}$), hence $C_{13}^{\mathrm{pulse}}(t)=X_1(t)$ and $C_{22}^{\mathrm{pulse}}(t)=\gamma X_1(t)$. The troughs before the weekly bolus satisfy $\log C_{\mathrm{trough},22}(t)-\log C_{\mathrm{trough},13}(t)\equiv\log\gamma$, so the misalignment is $|\log\gamma|\le\varepsilon$.
  \item Spread regimen $d^{\mathrm{spread}}$: split each weekly dose into $N$ equal boluses at spacing $1/N$ within the week (co-administered), with $N$ large. Then $C_{13}^{\mathrm{spread}}(t)=X_N(t)$ and $C_{22}^{\mathrm{spread}}(t)=\gamma X_N(t)$ with a common shape $X_N$ satisfying $\mathbb E[X_N]=m_{13}$ and $\operatorname{Var}(X_N)\downarrow0$ as $N\to\infty$ (approaching constant infusion). Again $\log C_{\mathrm{trough},22}(t)-\log C_{\mathrm{trough},13}(t)\equiv\log\gamma$, so the misalignment is $|\log\gamma|\le\varepsilon$.
\end{itemize}

Let $\Phi(d)$ be the weekly average of $\mathcal H$ under regimen $d$. Fix a compact interval $[a,b]\subset(0,\infty)$ containing the ranges of $X_1$ and $X_N$ (for $N$ large). On $[a,b]$ set
$$\kappa_u:=\min_{x\in[a,b]}(-u''(x))>0,\quad \kappa_v:=\min_{y\in[\gamma a,\gamma b]}(-v''(y))>0,$$
$$K_u:=\max_{x\in[a,b]}(-u''(x)),\quad K_v:=\max_{y\in[\gamma a,\gamma b]}(-v''(y)),$$
$$L:=\max_{x\in[a,b]}\Big|\tfrac{d^2}{dx^2}\big(u(x)\,v(\gamma x)\big)\Big|<\infty.$$
By a second-order Taylor expansion at the mean (quantitative Jensen bounds), for any square-integrable $X\in[a,b]$:
\begin{itemize}
  \item $u(m_{13})-\tfrac{K_u}{2}\operatorname{Var}(X)\le\mathbb E[u(X)]\le u(m_{13})-\tfrac{\kappa_u}{2}\operatorname{Var}(X)$,
  \item $v(\gamma m_{13})-\tfrac{\gamma^2 K_v}{2}\operatorname{Var}(X)\le\mathbb E[v(\gamma X)]\le v(\gamma m_{13})-\tfrac{\gamma^2\kappa_v}{2}\operatorname{Var}(X)$,
  \item with $f(x):=u(x)v(\gamma x)$, $\big|\mathbb E[f(X)]-f(m_{13})\big|\le\tfrac{L}{2}\operatorname{Var}(X)$.
\end{itemize}

Write the baseline value at the common means as
$$B:=u(m_{13})+v(\gamma m_{13})+\alpha\,u(m_{13})v(\gamma m_{13}).$$
Then
\begin{align*}
\Phi(d^{\mathrm{pulse}})
&=\mathbb E[u(X_1)]+\mathbb E[v(\gamma X_1)]+\alpha\,\mathbb E[f(X_1)]\\
&\le B+\Big(\alpha\tfrac{L}{2}-\tfrac{\kappa_u+\gamma^2\kappa_v}{2}\Big)\operatorname{Var}(X_1),
\end{align*}
while
\begin{align*}
\Phi(d^{\mathrm{spread}})
&=\mathbb E[u(X_N)]+\mathbb E[v(\gamma X_N)]+\alpha\,\mathbb E[f(X_N)]\\
&\ge B-\Big(\tfrac{K_u+\gamma^2K_v}{2}+\alpha\tfrac{L}{2}\Big)\operatorname{Var}(X_N).
\end{align*}
Choose $\alpha\in\big(0,\tfrac{\kappa_u+\gamma^2\kappa_v}{L}\big)$ so the coefficient $\alpha\tfrac{L}{2}-\tfrac{\kappa_u+\gamma^2\kappa_v}{2}$ is negative. Since $\operatorname{Var}(X_1)>0$, there is $\Delta>0$ (depending on $X_1$ and $\alpha$) with $\Phi(d^{\mathrm{pulse}})\le B-\Delta$. Because $\operatorname{Var}(X_N)\downarrow0$ as $N\to\infty$, pick $N$ large so that $\big(\tfrac{K_u+\gamma^2K_v}{2}+\alpha\tfrac{L}{2}\big)\operatorname{Var}(X_N)<\tfrac{\Delta}{2}$. Then
$$\Phi(d^{\mathrm{spread}})\ge B-\tfrac{\Delta}{2}>B-\Delta\ge\Phi(d^{\mathrm{pulse}}).$$
Thus both regimens have trough misalignment $\le\varepsilon$ and equal weekly doses, yet $\Phi(d^{\mathrm{spread}})>\Phi(d^{\mathrm{pulse}})$. Because the clinical outcome maps are strictly increasing in $\Phi$, we obtain the strict probability ordering
\[
\mathbb P(\mathrm{E75}\mid d^{\mathrm{spread}})>\mathbb P(\mathrm{E75}\mid d^{\mathrm{pulse}}).\qedhere
\]
This contradicts the assumption that every regimen with misalignment $\le\varepsilon$ attains the maximal $\mathbb P(\mathrm{E75})$. Therefore, there exist two equal-weekly-dose regimens with trough misalignment $\le\varepsilon$ for which $\mathbb P(\mathrm{E75})$ differs, with one strictly larger than the other, as claimed.
\end{proof}

\begin{theorem}[Stronger impossibility: the constant baseline uniquely maximizes the fixed-AE-budget utility]\label{thm:constant-unique-max}
Let \(J_\lambda(d):=\mathbb{E}[\Delta\mathrm{EASI}_T\mid d]-\lambda\,\mathbb{E}[\mathrm{AEs}\mid d]\) with a fixed \(\lambda>0\) and a baseline fixed regimen \(d_0\) that administers a constant dose \(u_0\in(0,U_{\max})\). There exist admissible instances (including perfect forecasts and with arbitrarily small convex AE penalties) such that \(d_0\) uniquely maximizes \(J_\lambda\) over all regimens \(d\) with doses in \([0,U_{\max}]\). Consequently, for every nonconstant adaptive regimen \(d\)---in particular, for every piecewise-constant threshold regimen that increases dose when the forecast exceeds a threshold and decreases otherwise---one has \(J_\lambda(d)<J_\lambda(d_0)\), regardless of the forecast law (including perfect forecasts). In particular, no such regimen can simultaneously reduce flare probability and satisfy \(J_\lambda(d)\ge J_\lambda(d_0)\).
\end{theorem}

\begin{proof}
Fix a finite horizon partitioned into bins \(i=1,\dots,n\). Let the per--bin dose be \(u_i\in[0,U_{\max}]\), the realized pollutant be \(E_i\), and a forecast \(\widehat E_i\). Let the baseline fixed regimen \(d_0\) be constant: \(u_i(d_0)\equiv u_0\in(0,U_{\max})\).

Model the per--bin flare probability by
\[
\phi(E,u):=\tfrac12+r(E)+s(u),
\]
where \(\mathbb{E}[|r(E_i)|]<\infty\) and \(s:[0,U_{\max}]\to\mathbb{R}\) is strictly decreasing (no convexity of \(s\) is needed). Choose parameters so that \(\phi\in(0,1)\). The expected time--averaged flare probability under regimen \(d\) is
\[
\Pi(d):=\frac1n\sum_{i=1}^n\mathbb{E}[\phi(E_i,u_i(d))].
\]
Link clinical improvement to flare burden via
\[
\mathbb{E}[\Delta\mathrm{EASI}_T\mid d] = -C\,\Pi(d),\qquad C>0.
\]
Model adverse events (AEs) by an additive per--bin convex penalty \(\psi:[0,U_{\max}]\to\mathbb{R}\) and let
\[
\mathbb{E}[\mathrm{AEs}\mid d]=\frac1n\sum_{i=1}^n\mathbb{E}[\psi(u_i(d))].
\]
The utility is
\[
J_\lambda(d):=-C\,\Pi(d)-\lambda\,\mathbb{E}[\mathrm{AEs}\mid d].
\]
Since \(\mathbb{E}[r(E_i)]\) is independent of \(d\), write
\[
J_\lambda(d)=K-\frac1n\sum_{i=1}^n\mathbb{E}[g(u_i(d))],\qquad g(u):=C\,s(u)+\lambda\,\psi(u),\quad K:=-C\Big(\tfrac12+\mathbb{E}[r(E)]\Big).
\]

Construct an admissible instance as follows. Fix \(\beta,\varepsilon>0\) (as small as desired) and set
\[
 s(u):=s_0-\beta\,(u-u_0) \quad\text{(strictly decreasing, linear)}
\]
with \(s_0\) chosen so that \(\phi\in(0,1)\). Define a strictly convex AE penalty
\[
 \psi(u):=\psi_0+\kappa\,(u-u_0)+\tfrac{\varepsilon}{2}(u-u_0)^2,
\]
with \(\psi_0\) large enough so that \(\psi\ge0\) on \([0,U_{\max}]\). Choose
\[
 \kappa:=\frac{C\,\beta}{\lambda}>0.
\]
Then, for \(g(u)=C s(u)+\lambda \psi(u)\),
\[
 g'(u)=C s'(u)+\lambda \psi'(u)=(-C\beta)+\lambda\,(\kappa+\varepsilon(u-u_0))=\lambda\varepsilon\,(u-u_0),\qquad g''(u)=\lambda\varepsilon>0.
\]
Thus \(g\) is strictly convex with unique global minimizer at \(u_0\).

Now let \(d\) be any regimen (any mapping from the information available---forecasts, outcomes, etc.---to doses in \([0,U_{\max}]\)). For each bin \(i\),
\[
 \mathbb{E}[g(u_i(d))]\ge g(u_0),
\]
with strict inequality whenever \(\mathbb{P}(u_i(d)\ne u_0)>0\) (by strict convexity of \(g\) and the fact that \(u_0\) is the unique minimizer). Therefore
\[
 J_\lambda(d)=K-\frac1n\sum_{i=1}^n\mathbb{E}[g(u_i(d))]\le K-\frac1n\sum_{i=1}^n g(u_0)=J_\lambda(d_0).\qedhere
\]
This conclusion is independent of the joint law of \((E,\widehat E)\); in particular it holds under perfect forecasts \(\widehat E\equiv E\).

Hence, in this admissible instance, the baseline constant regimen \(d_0\) uniquely maximizes \(J_\lambda\) over all regimens. As a corollary, every nonconstant adaptive regimen---including any piecewise-constant threshold regimen that increases dose when the forecast exceeds a threshold and decreases otherwise---satisfies \(J_\lambda(d)<J_\lambda(d_0)\). Consequently, no such regimen can simultaneously reduce flare probability and maintain \(J_\lambda(d)\ge J_\lambda(d_0)\).
\end{proof}

\begin{theorem}[Constant-rule optimality]\label{thm:constant-optimality}
There exists a data-generating process with two groups $g\in\{0,1\}$ and baseline covariates $Z=(L_0,M_0)$ such that, for every family of (possibly group-aware) dosing rules $\{\delta_g(L_0,Z)\}$ and for each group $g$,
\[
\mathbb E\big[\mu(L_0,\delta_g(L_0,Z))\mid g\big] \le \sup_{c\in[0,1]} \mathbb E\big[\mu(L_0,c)\big],
\]
with equality if and only if $\delta_g\equiv c^*$ almost surely in that group, where $c^*$ attains $\sup_{c\in[0,1]} \mathbb E[\mu(L_0,c)]$. Consequently,
\[
\mathbb E\big[\Delta\mathrm{EASI}\mid\{\delta_g\}\big] \le \sup_{\delta\,\text{group-blind const}} \mathbb E\big[\Delta\mathrm{EASI}\big],
\]
with strict inequality unless $\delta_g\equiv c^*$ in every group. (No assumption on adverse events is needed; the claim is purely about efficacy.)
\end{theorem}

\begin{proof}
We construct an instance and prove the claim groupwise.

Feasible doses are $m\in[0,1]$, and groups are $g\in\{0,1\}$. Conditional on $g$, let $L_0\sim\mathrm{Unif}[0,1]$, identically distributed across groups. Let $Z=(L_0,M_0)$, where $M_0$ is arbitrary and may be used by dosing rules. Define efficacy
\[
\mu(L,m) := A(m) + \kappa\,C(m)\,L,
\]
with parameters $K>0$ and $\gamma\in(0,\tfrac14)$, where
\[
A(m):= -K\,(m-\tfrac12)^2,\qquad C(m):= \big(m-(\tfrac12+\gamma)\big)_+^{\,3}.
\]
Let $\overline{C}:=\sup_{m\in[0,1]} C(m)=(\tfrac12-\gamma)^3>0$, and choose $\kappa\in\bigl(0,\,K\gamma^2/\overline{C}\bigr)$.

\medskip
\noindent\emph{(1) Best group-blind constant.} For a constant dose $c\in[0,1]$ and $L\sim\mathrm{Unif}[0,1]$,
\[
F(c):=\mathbb E[\mu(L,c)]=A(c)+\tfrac{\kappa}{2}C(c).
\]
If $c\le \tfrac12+\gamma$, then $C(c)=0$ and $F(c)=A(c)\le A(\tfrac12)$, with equality only at $c=\tfrac12$. If $c>\tfrac12+\gamma$, then
\[
A(\tfrac12)-A(c)=K(c-\tfrac12)^2\ge K\gamma^2>\kappa\,\overline{C}\ge \kappa\,C(c),
\]
so
\[
A(\tfrac12)-F(c)=\big(A(\tfrac12)-A(c)\big)-\tfrac{\kappa}{2}C(c)>\tfrac{\kappa}{2}C(c)>0.
\]
Thus $\sup_{c\in[0,1]} F(c)$ is attained uniquely at $c^*=\tfrac12$ with value $F(c^*)=A(\tfrac12)$.

\medskip
\noindent\emph{(2) Arbitrary (possibly group-aware) dosing rules.} Let an arbitrary measurable dosing rule be given, possibly group-aware and depending on all of $Z$, written $\delta_g(L_0,M_0)\in[0,1]$. Fix $g$ and abbreviate $L=L_0$ and $m=\delta_g(L_0,M_0)$. Define
\[
B^+ := \{m>\tfrac12+\gamma\},\quad B^- := \{m<\tfrac12-\gamma\},\quad s:=\mathbb P(B^+\mid g),\; t:=\mathbb P(B^-\mid g).
\]
Since $A(\tfrac12)-A(m)=K(m-\tfrac12)^2\ge K\gamma^2\,\mathbf{1}_{B^+\cup B^-}$ and $C(m)=0$ off $B^+$,
\begin{align*}
\mathbb E[\mu(L,m)\mid g]
&= \mathbb E[A(m)\mid g] + \kappa\,\mathbb E[C(m)L\mid g] \\
&\le A(\tfrac12) - K\gamma^2\,\mathbb P(B^+\cup B^-\mid g) + \kappa\,\overline{C}\,\mathbb E[\mathbf{1}_{B^+}L\mid g] \\
&\le A(\tfrac12) - K\gamma^2(s+t) + \kappa\,\overline{C}\,s \\
&= A(\tfrac12) - K\gamma^2 t + (\kappa\,\overline{C} - K\gamma^2) s \\
&\le A(\tfrac12),
\end{align*}
because $\kappa\,\overline{C} < K\gamma^2$. Moreover, equality can hold only if $s=t=0$, i.e., $m\in[\tfrac12-\gamma,\tfrac12+\gamma]$ almost surely, in which case $C(m)\equiv 0$ and
\[
\mathbb E[\mu(L,m)\mid g]=\mathbb E[A(m)\mid g]\le A(\tfrac12),
\]
with equality only when $m\equiv \tfrac12$ almost surely (since $A(\tfrac12)-A(m)=K(m-\tfrac12)^2$). Hence, for every group $g$,
\[
\mathbb E\big[\mu(L_0,\delta_g(L_0,M_0))\mid g\big]\le A(\tfrac12),
\]
with equality if and only if $\delta_g\equiv \tfrac12$ almost surely in that group.

Averaging over groups yields
\[
\mathbb E\big[\mu(L_0,\delta_g(L_0,M_0))\big]\le A(\tfrac12)=\sup_{c\in[0,1]} F(c).
\]
By construction, $\mathbb E[\Delta\mathrm{EASI}\mid\{\delta_g\}] = \mathbb E[\mu(L_0,\delta_g(L_0,M_0))]$, and the inequality is strict unless $\delta_g\equiv c^*=\tfrac12$ in every group. Therefore,
\[
\mathbb E\big[\Delta\mathrm{EASI}\mid\{\delta_g\}\big] \le \sup_{c\in[0,1]} \mathbb E\big[\mu(L_0,c)\big] 
= \sup_{\delta\,\text{group-blind const}} \mathbb E\big[\Delta\mathrm{EASI}\big].\qedhere
\]
\end{proof}


\subsection*{Modeling, invariance, and fairness}
\begin{theorem}[No ranking gain in a degenerate library]\label{thm-no-ranking-gain}
There exists a library of candidate bispecifics and associated mappings $F,\,\kappa,\,Q,$ and $J$, with a monthly dosing set $D$, such that for every $d\in D$ one has
\[
\rho_s\bigl(\kappa(F(\mathrm{complex})),\, Q(d)\bigr)=0;
\]
moreover, for any fixed fraction $q\in(0,1]$, selecting the top-$q$ fraction by $\kappa$ yields the same $\mathbb{E}[J(d)]$ as random selection of equal size (in particular, not strictly higher).
\end{theorem}

\begin{proof}
Assume, toward a contradiction, that for every library of candidate bispecifics with mappings $F,\,\kappa,\,Q,$ and $J$ and monthly dosing set $D$ satisfying the setup, there exists a monthly regimen $d\in D$ for which $\rho_s\bigl(\kappa(F(\mathrm{complex})),Q(d)\bigr)>0$, and that selecting the top-$q$ fraction by $\kappa$ yields strictly higher $\mathbb{E}[J(d)]$ than random selection of equal size.

Construct the following library and mappings, which satisfy the setup:
\begin{itemize}
  \item The library consists of $N\ge 2$ clones of a single bispecific molecule. For each candidate $C$, the predicted complex $F(\mathrm{complex}(C))$ is the same object; hence the epitope-coverage score $K:=\kappa(F(\mathrm{complex}(C)))$ is constant across the library: $K\equiv k_0$.
  \item For any monthly regimen $d\in D$, because the candidates are identical, the QSP efficacy readout is also constant across candidates: $Q(d)\equiv q_d$.
\end{itemize}

Spearman's $\rho_s$ for possibly discrete variables is defined via the distributional transform: for any real $X$ with CDF $F_X$, set
\[
T_X:=F_X(X-) + W\bigl(F_X(X)-F_X(X-)\bigr),\qquad W\sim\operatorname{Unif}(0,1)\text{ independent of }X,
\]
then $\rho_s(X,Y):=\operatorname{corr}(T_X,T_Y)$ using independent tie-breakers. If $X$ is almost surely constant, then $T_X\sim\operatorname{Unif}(0,1)$ and is independent of any $T_Y$ constructed with an independent tie-breaker, so $\rho_s(X,Y)=0$.

Applying this to the constructed library, for every monthly regimen $d$, both $K$ and $Q(d)$ are constant across candidates; hence $\rho_s\bigl(K,Q(d)\bigr)=0$. This contradicts the assumed existence, for every such library, of a regimen $d$ with $\rho_s\bigl(\kappa(F(\mathrm{complex})),Q(d)\bigr)>0$.

For the selection claim, fix any monthly regimen $d$. Because $K$ is constant, any top-$q$ selection by $\kappa$ is an arbitrary size-$qN$ subset (ties everywhere). Since all candidates are identical, $J(d)$ is the same for every candidate; thus the mean $J(d)$ over any size-$qN$ subset equals the population mean. Consequently, the top-$q$ selection does not yield strictly higher $\mathbb{E}[J(d)]$ than random selection of equal size, contradicting the assumption.

Both contradictions show that there exists a library and mappings satisfying the setup for which, for every monthly regimen $d$, $\rho_s\bigl(\kappa(F(\mathrm{complex})), Q(d)\bigr)=0$ and the top-$q$ selection by $\kappa$ does not strictly improve $\mathbb{E}[J(d)]$ over random. In fact, the expectations are equal; summarizing,
\[
\forall d\in D:\quad \rho_s\bigl(K,Q(d)\bigr)=0\quad\text{and}\quad \mathbb{E}[J(d)\mid \text{top-}q\text{ by }\kappa]=\mathbb{E}[J(d)\mid \text{random, size }qN].\qedhere
\]
\end{proof}

\begin{theorem}\label{thm:orthogonal-score}
There exists a finite library $\mathcal D$ of six designs, functions $\mu(d)$ and $v(d)$, the score $\widehat{\kappa}(d)=\mu(d)-v(d)$, and $q=3$, such that although the true synergy depends on both mean and variability (namely $Q(d)=\mu(d)+v(d)$), we have $\rho_s\!\big(\widehat{\kappa},Q\big)=0$, and for every function $J$ that depends only on $Q$ (i.e., $J(d)=g(Q(d))$ for arbitrary $g$), selecting the top-$q$ designs by $\widehat{\kappa}$ yields expected utility equal to that of a uniformly random subset of size $q$.
\end{theorem}

\begin{proof}
Proof by explicit construction. Let $\mathcal D=\{d_1,\dots,d_6\}$. For each $d\in\mathcal D$, define
\[
\widehat{\kappa}(d):=\mu(d)-v(d),\qquad Q(d):=\mu(d)+v(d),
\]
so $Q$ depends on both the mean and the variability.

Instantiate the designs by listing $(\mu,v)$-pairs in two groups:
\begin{itemize}
  \item Group $H$ (high $\widehat{\kappa}$): $d_1\!:\!(2,1)$, $d_2\!:\!(3,2)$, $d_3\!:\!(4,3)$, so $\widehat{\kappa}\equiv 1$ and $Q\in\{3,5,7\}$;
  \item Group $L$ (low $\widehat{\kappa}$): $d_4\!:\!(1,2)$, $d_5\!:\!(2,3)$, $d_6\!:\!(3,4)$, so $\widehat{\kappa}\equiv -1$ and $Q\in\{3,5,7\}$.
\end{itemize}
Thus the multiset of $Q$-values in $H$ equals that in $L$.

Spearman correlation. With ties handled by midranks, let $X$ and $Y$ denote the midranks of $\widehat{\kappa}$ and $Q$, respectively. The three values $-1$ occupy positions $1$--$3$ and the three values $+1$ occupy positions $4$--$6$, giving $X\in\{2,5\}$. For $Q$, the pairs $(3,3)$, $(5,5)$, $(7,7)$ occupy positions $1$--$2$, $3$--$4$, $5$--$6$, yielding $Y\in\{1.5,3.5,5.5\}$. The six rank pairs are
\[(2,1.5),\ (2,3.5),\ (2,5.5),\ (5,1.5),\ (5,3.5),\ (5,5.5).
\]
Hence
\[
\mathbb E[X]=\tfrac{1}{6}(2+2+2+5+5+5)=\tfrac{7}{2},\quad
\mathbb E[Y]=\tfrac{1}{6}(1.5+3.5+5.5+1.5+3.5+5.5)=\tfrac{7}{2},
\]
\[
\mathbb E[XY]=\tfrac{1}{6}\big(2(1.5+3.5+5.5)+5(1.5+3.5+5.5)\big)=\tfrac{49}{4},
\]
so $\operatorname{Cov}(X,Y)=\mathbb E[XY]-\mathbb E[X]\,\mathbb E[Y]=0$. Because $X$ and $Y$ are non-constant (hence $\operatorname{Var}(X),\operatorname{Var}(Y)>0$), it follows that $\rho_s\!\big(\widehat{\kappa},Q\big)=0$.

Utility. Let $J(d)=g(Q(d))$ for an arbitrary function $g$. Selecting the top-$q$ designs by $\widehat{\kappa}$ with $q=3$ picks exactly the group $H$, whose $Q$-values are $\{3,5,7\}$, so
\[
\mathbb E\big[J\mid d\in H\big]=\frac{g(3)+g(5)+g(7)}{3}.
\]
A uniformly random subset $S\subset\mathcal D$ of size $3$ has expected average $J$ equal to the population mean of $J$ (by linearity of expectation under sampling without replacement), namely
\[
\mathbb E\Big[\frac{1}{3}\sum_{d\in S} J(d)\Big]
=\frac{1}{6}\sum_{d\in\mathcal D} J(d)
=\frac{2g(3)+2g(5)+2g(7)}{6}
=\frac{g(3)+g(5)+g(7)}{3}
=\mathbb E\big[J\mid d\in H\big] \,,\ \qedhere
\]
for every $J(d)=g(Q(d))$.
\end{proof}

\begin{proposition}\label{prop:negative-corr-combo}
Let $p$ be sampled uniformly from the two-element library $\{p_1,p_2\}$. For each $p$, let $C_p$ be an event and let $F(p)$ be any feature variable; define $\kappa_\perp(p):=\mathbb P(C_p\mid F(p))\in[0,1]$ and $J^{\mathrm{combo}}(p):=S_p\,\mathbf 1_{C_p}$, where $S_p\ge 0$ is the synergy amplitude realized when $C_p$ occurs. Then there exist choices of $F$ and deterministic nonnegative amplitudes $S_p$ (depending only on $F(p)$ and $C_p$) for which $\mathrm{corr}(\kappa_\perp, J^{\mathrm{combo}})<0$. Consequently, a universally positive correlation $\mathrm{corr}(\kappa_\perp, J^{\mathrm{combo}})>0$ need not hold.
\end{proposition}

\begin{proof}
Let $\mathcal L=\{p_1,p_2\}$ and suppose that for each $p\in\mathcal L$ we have a feature value $F(p)$ and an event $C_p$ with
\[
\kappa_\perp(p):=\mathbb P(C_p\mid F(p))\in[0,1].
\]
Define $J^{\mathrm{combo}}(p):=S_p\,\mathbf 1_{C_p}$ with $S_p\ge 0$ chosen deterministically from $(F(p),C_p)$ by
\[
S_p:=\begin{cases}\mu(F(p)),& C_p=1,\\ 0,& C_p=0,\end{cases}
\]
for a nonnegative function $\mu$ on the range of $F$. Then
\[
\mathbb E\big[J^{\mathrm{combo}}(p)\mid F(p)\big]=\mathbb P(C_p\mid F(p))\,\mathbb E\big[S_p\mid F(p),C_p{=}1\big]=\kappa_\perp(p)\,\mu(F(p)).
\]
Choose the deterministic specifications
\[
\big(\kappa_\perp(p_1),\ \mu(F(p_1))\big)=\Big(\tfrac{9}{10},\ \tfrac{1}{10}\Big),\qquad \big(\kappa_\perp(p_2),\ \mu(F(p_2))\big)=\Big(\tfrac{1}{10},\ 10\Big).
\]
Let $X:=\kappa_\perp(p)$ and $Y:=J^{\mathrm{combo}}(p)$ under the uniform draw of $p\in\mathcal L$ and the internal randomness of $(C_p,S_p)$ conditional on $F(p)$. Writing $\mathcal F:=\sigma(F)$, the variable $X$ is $\mathcal F$-measurable, hence
\[
\operatorname{Cov}(X,Y)=\operatorname{Cov}\big(X,\,\mathbb E[Y\mid \mathcal F]\big)=\operatorname{Cov}\big(X,\,\mu(F)\,X\big).
\]
Compute the moments:
\[
\mathbb E[X]=\tfrac{1}{2}\Big(\tfrac{9}{10}+\tfrac{1}{10}\Big)=\tfrac{1}{2},\qquad \mathbb E\big[\mathbb E[Y\mid \mathcal F]\big]=\tfrac{1}{2}\Big(\tfrac{9}{100}+1\Big)=\tfrac{109}{200},
\]
\[
\mathbb E[XY]=\mathbb E\big[\mu(F)X^2\big]=\tfrac{1}{2}\Big(\tfrac{1}{10}\cdot\tfrac{81}{100}+10\cdot\tfrac{1}{100}\Big)=\tfrac{181}{2000}.
\]
Therefore
\[
\operatorname{Cov}(X,Y)=\mathbb E[XY]-\mathbb E[X]\,\mathbb E[Y]=\tfrac{181}{2000}-\tfrac{1}{2}\cdot\tfrac{109}{200}=-\tfrac{364}{2000}<0.
\]
Moreover, $\operatorname{Var}(X)>0$ (since $X$ takes two distinct values) and $\operatorname{Var}(Y)\ge \operatorname{Var}(\mathbb E[Y\mid \mathcal F])>0$ (because $\mathbb E[Y\mid \mathcal F]$ takes the distinct values $\tfrac{9}{100}$ and $1$). Hence
\[
\mathrm{corr}(X,Y)=\frac{\operatorname{Cov}(X,Y)}{\sqrt{\operatorname{Var}(X)\operatorname{Var}(Y)}}<0,\quad\text{as claimed.}\qedhere
\]
Thus, even with a uniform two-element library and deterministic nonnegative amplitudes $S_p$ (depending only on $F(p)$ and $C_p$), one can have $\mathrm{corr}(\kappa_\perp, J^{\mathrm{combo}})<0$, so a universally positive correlation need not hold.
\end{proof}

\begin{theorem}[Invariant score with external AUC gain]\label{thm:invariant-score-auc-gain}
There exist a subset \(\tilde G\subset\mathcal{G}\) and a score \(S_{\tilde G}\) that is distributionally invariant across the skin and blood domains such that, for predicting \(\mathrm{EASI}{-}75\) on dupilumab,
\[
\mathrm{AUC}_{\text{external}}(S_{\tilde G})\;\ge\;\mathrm{AUC}_{\text{external}}(S_G)+\delta\quad\text{for some }\delta>0,
\]
while maintaining \(\mathrm{corr}_{\text{Spearman}}(S_{\tilde G},\Delta\mathrm{EASI})\neq 0\).
\end{theorem}

\begin{proof}
Let \(D\in\{\text{skin},\text{blood}\}\) denote the tissue domain, \(Y\in\{0,1\}\) the indicator of achieving EASI-75 on dupilumab, and \(\Delta\mathrm{EASI}\in\mathbb{R}\) the improvement. Consider the following concrete probabilistic model (all random variables continuous except \(Y\)).

Choose a nonempty gene subset \(\tilde G\subset\mathcal G\) and summarize its cross-tissue expression by a one-dimensional score \(T:=T(X_{\tilde G})\in\mathbb{R}\). For each domain \(d\in\{\text{skin},\text{blood}\}\) assume
\[
T\mid(Y{=}1,D{=}d)\sim\mathcal N(\mu_{T,d},1),\qquad 
T\mid(Y{=}0,D{=}d)\sim\mathcal N(-\mu_{T,d},1),\qquad \mu_{T,d}>0,
\]
so that larger \(T\) within each domain favors response. Introduce a spurious coordinate \(V\) (measured on \(G:=\tilde G\cup\{v\}\)) that aligns with \(Y\) in skin but flips in blood:
\[
\begin{aligned}
&V\mid(Y{=}1,D{=}\text{skin})\sim\mathcal N(\mu_V,1),&& V\mid(Y{=}0,D{=}\text{skin})\sim\mathcal N(-\mu_V,1),\\
&V\mid(Y{=}1,D{=}\text{blood})\sim\mathcal N(-\mu_V,1),&& V\mid(Y{=}0,D{=}\text{blood})\sim\mathcal N(\mu_V,1),\quad \mu_V>0,
\end{aligned}
\]
with \((T,V)\) conditionally independent given \((Y,D)\).

Define a non-invariant score that exploits \(V\):
\[
S_G:=T+\lambda V,\qquad \lambda>0.
\]

Construct an invariant score using only \(\tilde G\) by domain-wise calibration. For each \(d\), let \(F_d\) be the unconditional CDF of \(T\mid D{=}d\) (the \(Y\)-mixture in domain \(d\)), and set
\[
S_{\tilde G}:=g_D(T):=\Phi^{-1}\big(F_D(T)\big),
\]
where \(\Phi\) is the standard normal CDF. Since \(U:=F_d(T)\mid D{=}d\sim\mathrm{Unif}(0,1)\) (the mixture is absolutely continuous), it follows that \(S_{\tilde G}\mid D{=}d\sim\mathcal N(0,1)\) for every \(d\). Thus \(S_{\tilde G}\) is distributionally invariant across skin and blood, and each \(g_d\) is strictly increasing.

External AUCs are evaluated in blood (\(D{=}\text{blood}\)). For any continuous score, \(\mathrm{AUC}=\mathbb P(S_+>S_-)\) with \(S_+\overset{d}{=}S\mid Y{=}1\) and \(S_-\overset{d}{=}S\mid Y{=}0\) independent. Because \(S_{\tilde G}=g_{\text{blood}}(T)\) with \(g_{\text{blood}}\) strictly increasing,
\[
\begin{aligned}
\mathrm{AUC}_{\text{external}}(S_{\tilde G})
&= \mathbb P\big(g_{\text{blood}}(T_+)>g_{\text{blood}}(T_-)\big)\\
&= \mathbb P(T_+>T_-)\\
&= \Phi\!\left(\frac{\mu_{T,\text{blood}}-(-\mu_{T,\text{blood}})}{\sqrt{1+1}}\right)
= \Phi(\sqrt{2}\,\mu_{T,\text{blood}}) > \tfrac12.
\end{aligned}
\]
For \(S_G=T+\lambda V\) evaluated in blood,
\[
S_+\sim\mathcal N(\mu_{T,\text{blood}}-\lambda\mu_V,\,1+\lambda^2),\quad S_-\sim\mathcal N(-\mu_{T,\text{blood}}+\lambda\mu_V,\,1+\lambda^2),
\]
so
\[
\begin{aligned}
\mathrm{AUC}_{\text{external}}(S_G)
&=\Phi\!\left(\frac{(\mu_{T,\text{blood}}-\lambda\mu_V)-(-\mu_{T,\text{blood}}+\lambda\mu_V)}{\sqrt{(1+\lambda^2)+(1+\lambda^2)}}\right)\\
&=\Phi\!\left(\frac{\sqrt{2}\,(\mu_{T,\text{blood}}-\lambda\mu_V)}{\sqrt{1+\lambda^2}}\right).
\end{aligned}
\]
Choosing \(\lambda\mu_V>\mu_{T,\text{blood}}\) makes the argument negative, hence \(\mathrm{AUC}_{\text{external}}(S_G)<\tfrac12\). Therefore, for
\[
\delta:=\Phi(\sqrt{2}\,\mu_{T,\text{blood}})-\Phi\!\left(\tfrac{\sqrt{2}\,(\mu_{T,\text{blood}}-\lambda\mu_V)}{\sqrt{1+\lambda^2}}\right)>0,
\]
we have \(\mathrm{AUC}_{\text{external}}(S_{\tilde G})\ge \mathrm{AUC}_{\text{external}}(S_G)+\delta\).

It remains to ensure nonzero Spearman correlation with clinical improvement. In blood, set
\[
\Delta\mathrm{EASI}=\beta T+\xi,\qquad \beta\neq 0,\; \xi\perp T,\; \xi\sim\mathcal N(0,\sigma^2).
\]
We use the following lemma.

\begin{lemma}[Rank-correlation under independent noise]\label{lem:rank-corr-noise}
Let \(Y:=\beta T+\xi\) with \(\beta\ne0\), \(T\) continuous, and \(\xi\perp T\) with a strictly positive density everywhere (e.g., Gaussian). Then the Spearman correlation \(\rho_s(T,Y)\) has the sign of \(\beta\) and is nonzero.
\end{lemma}

\begin{proof}[Proof of Lemma \ref{lem:rank-corr-noise}]
It suffices to treat \(\beta>0\); the case \(\beta<0\) follows by replacing \(T\) with \(-T\). Let \(F_T\) and \(F_Y\) be the CDFs of \(T\) and \(Y\), and set \(U:=F_T(T)\), \(V:=F_Y(Y)\). For continuous \(T\) and \(Y\), \(U,V\sim\mathrm{Unif}(0,1)\) and \(\rho_s(T,Y)=12\,\mathrm{Cov}(U,V)\). Because \(\xi\) has a strictly positive density, \(Y=\beta T+\xi\) has an everywhere positive density: for every \(y\),
\[
 f_Y(y)=\mathbb E\big[f_\xi(y-\beta T)\big]>0,
\]
so \(F_Y\) is strictly increasing. For any fixed \(x\in\mathbb R\), define \(g_x(t):=F_Y(\beta t+x)\), which is strictly increasing in \(t\). Let \(T'\) be an i.i.d. copy of \(T\), independent of everything. Using \(2\,\mathrm{Cov}(A,B)=\mathbb{E}[(A-A')(B-B')]\) for i.i.d. copies,
\[
2\,\mathrm{Cov}(U,V\mid\xi{=}x)=\mathbb{E}\big[(F_T(T)-F_T(T'))\,(g_x(T)-g_x(T'))\big].
\]
Since \(F_T\) is nondecreasing and \(g_x\) strictly increasing, the integrand is almost surely nonnegative and strictly positive with probability 1 (because \(T\) is continuous so \(\mathbb P(T\neq T')=1\) and \(g_x(T)\neq g_x(T')\) whenever \(T\neq T'\)). Hence \(\mathrm{Cov}(U,V\mid\xi{=}x)>0\) for every \(x\), and therefore \(\mathrm{Cov}(U,V)>0\). Consequently \(\rho_s(T,Y)=12\,\mathrm{Cov}(U,V)>0\).
\end{proof}

Finally, \(S_{\tilde G}=g_{\text{blood}}(T)\) is a strictly increasing function of \(T\) in the external (blood) domain, and for any strictly increasing \(h\),
\[
F_{h(T)}(h(t))=\mathbb{P}(h(T)\le h(t))=\mathbb{P}(T\le t)=F_T(t),
\]
so Spearman correlation is invariant under strictly increasing transforms of either variable. Therefore,
\[
\rho_s(S_{\tilde G},\Delta\mathrm{EASI}\mid D{=}\text{blood})=\rho_s(T,\Delta\mathrm{EASI}\mid D{=}\text{blood})\neq 0\quad\text{by Lemma \ref{lem:rank-corr-noise}}.
\]
Collecting the conclusions, there exist \(\tilde G\subset\mathcal G\) and a domain-invariant score \(S_{\tilde G}\) such that, for predicting EASI-75 on dupilumab in the external (blood) domain,
\[
\begin{aligned}
\mathrm{AUC}_{\text{external}}(S_{\tilde G}) &\ge \mathrm{AUC}_{\text{external}}(S_G)+\delta\quad(\delta>0),\\
\mathrm{corr}_{\text{Spearman}}(S_{\tilde G},\Delta\mathrm{EASI}\mid D{=}\text{blood}) &\ne 0.\;\qedhere
\end{aligned}
\]
\end{proof}

\begin{theorem}[No uniform AUC margin over Wasserstein neighborhoods]\label{thm:auc-no-margin-wasserstein}
There exists a source distribution \(\mathbb P_{\mathrm{source}}\) on \(\mathbb R\) and a baseline score \(S_G\) such that \(\mathrm{AUC}_{\mathbb P_{\mathrm{source}}}(S_G)=1\), and for any score \(S_{\tilde G}\) and any uncertainty set \(\mathcal U\) of distributions containing \(\mathbb P_{\mathrm{source}}\), no uniform positive AUC margin over \(\mathcal U\) is possible: for every \(\delta>0\) there exists \(\mathbb Q\in\mathcal U\) with
\[
\mathrm{AUC}_{\mathbb Q}(S_{\tilde G})<\mathrm{AUC}_{\mathbb Q}(S_G)+\delta.
\]
In particular, this holds for every Wasserstein ball \(\mathcal W_{r_0}(\mathbb P_{\mathrm{source}})\) with any radius \(r_0>0\).
\end{theorem}

\begin{proof}
Let \((\mathcal X,\lVert\cdot\rVert)=(\mathbb R,|\cdot|)\). Consider binary labels \(Y\in\{0,1\}\) and define the source law \(\mathbb P:=\mathbb P_{\mathrm{source}}\) by
\(\mathbb P(Y=1)=\mathbb P(Y=0)=\tfrac12\), and
\(X\mid Y=0\sim \mathrm{Unif}[0,1]\) while \(X\mid Y=1\sim \mathrm{Unif}[2,3]\).
Define the baseline score \(S_G(x):=x\). If \(X^+\sim \mathbb P(\cdot\mid Y{=}1)\) and \(X^-\sim \mathbb P(\cdot\mid Y{=}0)\) are independent, then \(X^+>X^-\) almost surely, hence \(\mathrm{AUC}_{\mathbb P}(S_G)=1\).

Fix any score \(S_{\tilde G}\), any uncertainty set \(\mathcal U\) with \(\mathbb P\in\mathcal U\), and any \(\delta>0\). Taking \(\mathbb Q=\mathbb P\in\mathcal U\) gives
\[
\mathrm{AUC}_{\mathbb P}(S_{\tilde G})\le 1=\mathrm{AUC}_{\mathbb P}(S_G)<\mathrm{AUC}_{\mathbb P}(S_G)+\delta.
\]
Thus there exists \(\mathbb Q\in\mathcal U\) (namely \(\mathbb Q=\mathbb P\)) with \(\mathrm{AUC}_{\mathbb Q}(S_{\tilde G})<\mathrm{AUC}_{\mathbb Q}(S_G)+\delta\). Because \(S_{\tilde G}\) and \(\delta>0\) were arbitrary, no uniform positive AUC margin over \(\mathcal U\) is possible. In particular, for every \(r_0>0\) the same conclusion holds for the Wasserstein ball \(\mathcal W_{r_0}(\mathbb P_{\mathrm{source}})\), since
\[
W(\mathbb P,\mathbb P)=0\le r_0.\qedhere
\]
\end{proof}

\begin{theorem}\label{thm:extval-auc-gap}
There exist an external-validation distribution and a policy $\pi(H,E)$ mapping to $T\in\{D,\mathrm{L13},\mathrm{Dual}\}$ such that
\[
\mathrm{AUC}(\text{E75 predicted by }\pi) - \sup_{\pi'\,:\,\pi'\text{ ignores }E}\mathrm{AUC}(\text{E75 predicted by }\pi')\;\ge\;\frac{16792}{94435}\;>\;0.17.
\]
Equivalently, for the constructed instance one has $\mathrm{AUC}(\pi)=\tfrac{129}{170}$ and $\sup_{\pi'\,:\,\pi'\text{ ignores }E}\mathrm{AUC}(\pi')=\tfrac{1291}{2222}$.
\end{theorem}

\begin{proof}
Construct an external-validation population as follows.
\begin{enumerate}
\item Features and strata. Let $H\in\{0,1\}$ (endotype) and $E\in\{0,1\}$ (environment). Specify the stratum masses directly by
\[
\mathbb P(H=0)=\tfrac{1}{10},\qquad \mathbb P(H=1,E=1)=\tfrac{9}{20},\qquad \mathbb P(H=1,E=0)=\tfrac{9}{20}.
\]
No independence assumptions are needed. Denote strata $A:(H=0)$, $B:(H=1,E=1)$, $C:(H=1,E=0)$.

\item Potential-response model (EASI-75). For $T\in\{D,\mathrm{L13},\mathrm{Dual}\}$ set
\begin{itemize}
\item $H=0$ (low IL-13): $\mathbb P(E75\mid D,H=0)=\tfrac{9}{10}$, $\mathbb P(E75\mid \mathrm{L13},H=0)=\tfrac{13}{20}$, $\mathbb P(E75\mid \mathrm{Dual},H=0,E)=\tfrac{17}{20}$ (both $E$).
\item $H=1,E=1$ (high IL-13, high burden): $\mathbb P(E75\mid \mathrm{Dual},H=1,E=1)=\tfrac{19}{20}$, $\mathbb P(E75\mid \mathrm{L13},H=1,E=1)=\tfrac{3}{10}$.
\item $H=1,E=0$ (high IL-13, low burden): $\mathbb P(E75\mid \mathrm{L13},H=1,E=0)=\tfrac{3}{5}$, $\mathbb P(E75\mid \mathrm{Dual},H=1,E=0)=\tfrac{11}{20}$.
\end{itemize}
Additionally, $\mathbb P(E75\mid D,H=1,E)=\tfrac{1}{2}$ for both $E$.

\item Scores and AUC. For a policy $\pi$ that uses features $Z$, define the Bayes score $s_{\pi}(z):=\mathbb P(E75\mid T=\pi(z),z)$. Its AUC is
\[
\mathrm{AUC}(\pi)=\mathbb P\big(s_{\pi}(X^+) > s_{\pi}(X^-)\big)+\tfrac12\,\mathbb P\big(s_{\pi}(X^+) = s_{\pi}(X^-)\big),
\]
with $X^+\sim(Z\mid Y_{\pi}=1)$ and $X^-\sim(Z\mid Y_{\pi}=0)$ independent. When $\pi$ ignores $E$, $Z=H$ and $s_{\pi}(1)=\mathbb P(E75\mid T, H=1)$ is the $E$-mixture under $H=1$; by the chosen masses, $\mathbb P(E=1\mid H=1)=\mathbb P(E=0\mid H=1)=\tfrac12$.

\item An environment-aware policy. Set
\[
\pi(H,E)=\begin{cases}
D,& H=0,\\
\mathrm{Dual},& H=1,\ E=1,\\
\mathrm{L13},& H=1,\ E=0.
\end{cases}
\]
The induced scores on strata $(A,B,C)$ are $\big(\tfrac{9}{10},\tfrac{19}{20},\tfrac{3}{5}\big)$. The corresponding positive and negative masses are
\[
\big(\tfrac{9}{100},\tfrac{171}{400},\tfrac{27}{100}\big)\quad\text{and}\quad\big(\tfrac{1}{100},\tfrac{9}{400},\tfrac{9}{50}\big),
\]
respectively. A direct Mann--Whitney calculation over the three distinct score levels $0.95>0.9>0.6$ yields
\[
\mathrm{AUC}(\pi)=\frac{903}{1190}=\frac{129}{170}.
\]

\item E-ignoring competitors. Any policy $\pi'$ that ignores $E$ maps $H\mapsto T$ and hence has exactly two score values $s_0:=\mathbb P(E75\mid T_0,H=0)$ and $s_1:=\mathbb P(E75\mid T_1,H=1)$ with $T_0,T_1\in\{D,\mathrm{L13},\mathrm{Dual}\}$. From the model and $\mathbb P(E=1\mid H=1)=\mathbb P(E=0\mid H=1)=\tfrac12$,
\[
 s_0\in\Big\{\tfrac{9}{10},\,\tfrac{13}{20},\,\tfrac{17}{20}\Big\},\qquad s_1\in\Big\{\tfrac12,\,\tfrac{9}{20},\,\tfrac34\Big\}.
\]
Writing $p_0=\tfrac{1}{10}$, $p_1=\tfrac{9}{10}$, $q_0=s_0$, $q_1=s_1$, and masses $m_h^+=p_h q_h$, $n_h^-=p_h(1-q_h)$ with $M=m_0^++m_1^+$, $N=n_0^-+n_1^-$, the AUC equals
\[
\begin{aligned}
\text{if } s_0>s_1:\quad &\mathrm{AUC}(\pi')
=\frac{m_0^+n_1^-+\tfrac12 m_0^+n_0^-+\tfrac12 m_1^+n_1^-}{MN},\\
\text{if } s_0<s_1:\quad &\mathrm{AUC}(\pi')
=\frac{m_1^+n_0^-+\tfrac12 m_0^+n_0^-+\tfrac12 m_1^+n_1^-}{MN}.
\end{aligned}
\]
Evaluating all nine $(T_0,T_1)$ choices gives the maxima at $(T_0,T_1)=(D,\mathrm{L13})$ with
\[
\sup_{\pi'\,:\,E\ \text{ignored}}\mathrm{AUC}(\pi')=\frac{1291}{2222}\approx0.5810.
\]

\item Quantified improvement. Therefore
\[
\mathrm{AUC}(\pi)-\sup_{\pi'\,:\,E\ \text{ignored}}\mathrm{AUC}(\pi')\;\ge\;\frac{129}{170}-\frac{1291}{2222}\;=\;\frac{16792}{94435}\;>\;0.17.\qedhere
\]
\end{enumerate}
\end{proof}


\subsection*{Domain shift and post-hoc limits}
\begin{theorem}[Impossibility under covariate shift (tightened)]\label{thm:impossibility-covshift}
Fix two dosing tiers $d_1$ and $d_2$ and an arbitrary covariate vector $X$ taking values in a measurable space. Keeping the conditional response surfaces $p_d(x)=\mathbb P(E75\mid d, X=x)$ fixed, there do not exist a measurable subset $H$ of the covariate space and $\theta>0$ such that, for every covariate-shifted cohort (i.e., any probability law $\mu$ on $X$ that only changes the marginal of $X$), the inequality
\[
\mathbb P_\mu(E75\mid d_2, X\in H)-\mathbb P_\mu(E75\mid d_1, X\in H)\ \ge\ \mathbb P_\mu(E75\mid d_2)-\mathbb P_\mu(E75\mid d_1)+\theta
\]
holds.
\end{theorem}

\begin{proof}
Proof by contradiction. Fix $d_1,d_2$ and write $p_d(x):=\mathbb P(E75\mid d, X{=}x)$, which are invariant under covariate shift. Let $R(x):=p_{d_2}(x)-p_{d_1}(x)$. Suppose, for contradiction, that there exist a measurable subset $H$ of the covariate space and $\theta>0$ such that for every covariate-shifted cohort (probability law) $\mu$ on $X$ one has
\begin{equation}\label{eq:key-ineq}
\mathbb P_\mu(E75\mid d_2, X\in H)-\mathbb P_\mu(E75\mid d_1, X\in H)
=\mathbb E_\mu\big[R(X)\mid X\in H\big]\ \ge\ \mathbb E_\mu\big[R(X)\big]+\theta.
\end{equation}
Now consider a covariate-shifted cohort $\mu_H$ supported entirely on $H$ (i.e., $\mu_H(H)=1$), which changes only the marginal law of $X$. Then
\[
\mathbb E_{\mu_H}[R(X)\mid X\in H]=\mathbb E_{\mu_H}[R(X)],
\]
so applying \eqref{eq:key-ineq} with $\mu=\mu_H$ gives
\[
\mathbb E_{\mu_H}[R(X)]\ \ge\ \mathbb E_{\mu_H}[R(X)]+\theta,\quad\text{which is impossible for any }\theta>0.\ \qedhere
\]
\end{proof}

\begin{theorem}[Impossibility of safety-aware fairness via post-hoc calibration for stratified allocation]\label{thm:posthoc-impossibility}
There exists a data-generating process with two demographic groups $g\in\{0,1\}$ and covariates $X=(L_0,M_0)$ such that, for any policy $\pi$ stratified on $X$ and any post-hoc calibration $\tilde \pi$ of $\pi$ (possibly randomized and depending on $g$ and $X$),
\[
\max_{g\in\{0,1\}}\Bigl\lvert\mathbb{E}[\mathrm{AEs}\mid \tilde\pi,g]-\mathbb{E}[\mathrm{AEs}\mid \tilde\pi]\Bigr\rvert=\tfrac12.
\]
Consequently, for any $\varepsilon<\tfrac12$ no post-hoc calibration attains group-level AE parity within $\varepsilon$, even though efficacy satisfies $\mathbb{E}[\Delta\mathrm{EASI}\mid \tilde\pi]=\mathbb{E}[\Delta\mathrm{EASI}\mid \pi]=0$.
\end{theorem}

\begin{proof}
Assume, toward a contradiction, that a safety-aware fairness guarantee holds in general: given any stratified policy $\pi$ based on $X=(L_0,M_0)$, there exists a post-hoc calibration $\tilde\pi$ such that for some $\varepsilon<1/2$,
\[
\max_{g}\Bigl\lvert\mathbb{E}[\mathrm{AEs}\mid \tilde\pi,g]-\mathbb{E}[\mathrm{AEs}\mid \tilde\pi]\Bigr\rvert\le \varepsilon,
\]
while efficacy is preserved up to $o(\varepsilon)$.

We construct a specific instance. Let the demographic group be $g\in\{0,1\}$ with $\mathbb{P}(g=0)=\mathbb{P}(g=1)=\tfrac12$. Define covariates $X:=(L_0,M_0)$ by $L_0:=g$ and $M_0:=0$ almost surely. Let the action set be $\mathcal{A}=\{a_1,a_2,a_3\}$. Define bounded, measurable potential-outcome regressions:
\begin{itemize}
  \item Efficacy is action-invariant and null: for all $a$ and all $X$, $U_a(X):=\mathbb{E}[\Delta\mathrm{EASI}\mid a,X]\equiv 0$.
  \item Adverse events are action-invariant and depend only on $L_0$: for all $a$ and all $X$, $V_a(X):=\mathbb{E}[\mathrm{AEs}\mid a,X]\equiv L_0\in\{0,1\}$.
\end{itemize}
Let $\pi$ be any measurable policy stratified on $X$, and let $\tilde\pi$ be any post-hoc calibration of $\pi$, possibly randomized and depending on $g$ and $X$. Because $V_a(X)$ is action-invariant,
\[
\mathbb{E}[\mathrm{AEs}\mid \tilde\pi,g]=\mathbb{E}[V_{\tilde A}(X)\mid g]=\mathbb{E}[L_0\mid g]=g,
\]
so $\mathbb{E}[\mathrm{AEs}\mid \tilde\pi,g{=}0]=0$ and $\mathbb{E}[\mathrm{AEs}\mid \tilde\pi,g{=}1]=1$. Hence
\[
\mathbb{E}[\mathrm{AEs}\mid \tilde\pi]=\tfrac12\cdot 0+\tfrac12\cdot 1=\tfrac12,
\]
and therefore
\[
\max_{g\in\{0,1\}}\Bigl\lvert\mathbb{E}[\mathrm{AEs}\mid \tilde\pi,g]-\mathbb{E}[\mathrm{AEs}\mid \tilde\pi]\Bigr\rvert=\tfrac12.\ \qedhere
\]
This contradicts the assumed bound for any $\varepsilon<\tfrac12$. Meanwhile, efficacy preservation holds trivially since $U_a\equiv 0$, giving $\mathbb{E}[\Delta\mathrm{EASI}\mid \tilde\pi]=\mathbb{E}[\Delta\mathrm{EASI}\mid \pi]=0$. Thus, in this instance no post-hoc calibration can achieve the claimed AE parity for $\varepsilon<\tfrac12$, contradicting the assumed general guarantee.
\end{proof}


\subsection*{Nonmonotonic optimality and equalizers}
\begin{theorem}[Nonmonotone optimality counterexample]\label{thm-nonmonotone-optimality}
There exist a budget $q=\tfrac12$, a two-point distribution $\mu$ on a scalar covariate $X\in\{0,1\}$ with $\mu(\{0\})=\mu(\{1\})=\tfrac12$, and success probabilities $p_D, p_{\mathrm{Dual}}$ (equivalently, an uplift $u(x)=p_{\mathrm{Dual}}(x)-p_D(x)$ with $u(0)>u(1)$) such that no policy $\pi:\{0,1\}\to\{0,1\}$ that is nondecreasing in $x$ maximizes the population mean $\mathbb{P}(E75(24)=1)$ subject to the budget constraint $\mathbb{E}[\pi(X)]\le q$.
\end{theorem}

\begin{proof}
Let $X\in\{0,1\}$ with $\mu(\{0\})=\mu(\{1\})=\tfrac12$ and take the budget $q=\tfrac12$. For $t\in\{D,\mathrm{Dual}\}$, write 
\[
 p_t(x)=\mathbb{P}(E75(24)=1\mid X=x,\,T=t),\qquad u(x)=p_{\mathrm{Dual}}(x)-p_D(x).
\]
Take $p_D\equiv 0$ and choose $1\ge \alpha>\beta\ge 0$ with $p_{\mathrm{Dual}}(0)=\alpha$ and $p_{\mathrm{Dual}}(1)=\beta$, so $u(0)=\alpha$ and $u(1)=\beta$.

Any policy $\pi:\{0,1\}\to\{0,1\}$ yields objective
\[
\mathbb{E}\big[p_D(X)+u(X)\,\pi(X)\big]=\tfrac12\big(\alpha\,\pi(0)+\beta\,\pi(1)\big),
\]
under the budget
\[
\mathbb{E}[\pi(X)]=\tfrac12\big(\pi(0)+\pi(1)\big)\le \tfrac12,
\]
which is equivalent to $\pi(0)+\pi(1)\le 1$.

Because $\alpha>\beta$, the budget-constrained maximizer over all policies sets $\pi(0)=1$ and $\pi(1)=0$, achieving value $\tfrac12\alpha$.

If $\pi$ is nondecreasing in the scalar $x$ (i.e., $\pi(0)\le \pi(1)$), the budget feasibility $\pi(0)+\pi(1)\le 1$ restricts $\pi$ to the two possibilities: $\pi\equiv 0$ (value $0$) or $\pi=\mathbf{1}_{\{1\}}$ (value $\tfrac12\beta$). Neither attains the optimal value $\tfrac12\alpha$. Hence no nondecreasing policy maximizes the population mean success probability subject to the budget constraint. \qedhere
\end{proof}

\begin{theorem}[Monotone dosing equalizer]\label{thm-monotone-dosing-equalizer}
After conditioning on $Z=(W,\mathrm{Reg})$ within the $D$ arm, one has
\[
\operatorname{corr}(L_0, C_{\mathrm{trough}} \mid T{=}D, Z) < 0
\quad\text{and}\quad
\operatorname{corr}(C_{\mathrm{trough}}, E75(24) \mid T{=}D, Z) > 0;
\]
moreover, there exists a unique individualized dosing rule $\delta(L_0, Z)$, strictly increasing in $L_0$, that equalizes $\mathbb P(E75(24){=}1 \mid L_0, Z, \delta)$ across $L_0$ strata for any chosen target $p_* \in (s_{0+}, s_\infty)$.
\end{theorem}

\begin{proof}
Work throughout on the event $\{T=D\}$ and, unless stated otherwise, also condition on the fixed covariates $Z := (W, \mathrm{Reg})$. Let $R_0$ denote baseline free target (IL-4R$\alpha$), $L_0$ the baseline IL-13 axis, $C \equiv C_{\mathrm{trough}}$ the trough exposure, and $Y \equiv E75(24) \in \{0,1\}$.

Assumptions (tightened):
\begin{enumerate}
  \item[(A1)] Axis--target monotonicity: for each fixed $Z$ there is a strictly increasing $r_Z$ with $R_0 = r_Z(L_0)$.
  \item[(A2)] TMDD exposure map: there exist a residual $U$ with $U \perp L_0\mid Z$ and a measurable $G$ such that for fixed $(Z,d)$: (i) $C = G(R_0, U; Z, d)$; (ii) $r \mapsto G(r, U; Z, d)$ is strictly decreasing for every $(U, Z, d)$; (iii) $d \mapsto G(r, U; Z, d)$ is strictly increasing for every $(r, U, Z)$.
  \item[(D0)] Protocol dosing determinism: in the $T{=}D$ arm, $d = d_0(Z)$ a.s.
  \item[(ND)] Finite nondegenerate conditional variances: $\mathrm{Var}(L_0 \mid Z) \in (0, \infty)$ and $\mathrm{Var}(C \mid Z) \in (0, \infty)$ a.s.
  \item[(ER)] Exposure--response: $\mathbb P(Y{=}1 \mid C, Z) = s(C)$ with $s : (0, \infty) \to (0, 1)$ strictly increasing and continuous (hence $\mathbb E[Y \mid C, Z] = s(C)$).
  \item[(A2\textnormal{-}cts)] For every $(r, U, Z)$, the map $d \mapsto G(r, U; Z, d)$ is continuous on $(0, \infty)$.
  \item[(A3)] Dose--extreme limits: for every fixed $(r, U, Z)$, $\lim_{d\downarrow 0} G(r, U; Z, d) = 0$ and $\lim_{d\uparrow \infty} G(r, U; Z, d) = +\infty$.
\end{enumerate}
Note that boundedness of $s$ and the existence of the one-sided limits $s_{0+} := \lim_{c\downarrow 0} s(c)$ and $s_\infty := \lim_{c\uparrow \infty} s(c)$ follow from (ER) since $s$ maps into $(0,1)$ and is monotone.

\begin{lemma}[Monotone covariance identity and sign, conditional]\label{lem:monotone-covariance-conditional}
Let $\mathcal H$ be a sub-$\sigma$-field. Suppose $X \in L^2$ and $f(X) \in L^2$ conditionally on $\mathcal H$. Let $X'$ be a conditionally i.i.d. copy of $X$ given $\mathcal H$. Then
\[
\operatorname{Cov}(X, f(X) \mid \mathcal H) 
= \tfrac12\, \mathbb E\big[(X {-} X')(f(X) {-} f(X')) 
\mid \mathcal H\big] \quad \text{a.s.}
\]
If $f$ is strictly increasing (resp. strictly decreasing) and $\mathrm{Var}(X \mid \mathcal H) > 0$ a.s., then $\operatorname{Cov}(X, f(X) \mid \mathcal H) > 0$ (resp. $< 0$) a.s.
\end{lemma}
\begin{proof}
As in the original argument; omitted here for brevity.\qedhere
\end{proof}

\paragraph{1) Negative correlation between $L_0$ and $C$ after adjusting for $Z$.}
By (D0), conditioning on $Z$ pins down the dose $d_0(Z)$. Define $F_Z(r, U) := G(r, U; Z, d_0(Z))$. For fixed $Z$ and $U$, $r \mapsto F_Z(r, U)$ is strictly decreasing by (A2)(ii). The conditional mean exposure given $(L_0, Z)$ is
\[
 m(L_0, Z) := \mathbb E[C \mid L_0, Z] 
 = \mathbb E_U\big[F_Z\big(r_Z(L_0), U\big)\big],
\]
which is strictly decreasing in $L_0$ because $r_Z$ is strictly increasing (A1), $F_Z(\cdot, U)$ is strictly decreasing for every $U$, and $U \perp L_0 \mid Z$ by (A2). Write $C = m(L_0, Z) + \varepsilon$ with $\mathbb E[\varepsilon \mid L_0, Z] = 0$. Then $\operatorname{Cov}(L_0, \varepsilon \mid Z) = 0$ and
\[
\operatorname{Cov}(L_0, C \mid Z) = \operatorname{Cov}\big(L_0, m(L_0, Z) \mid Z\big).
\]
By (ND), $L_0, C \in L^2 \mid Z$, hence $m(L_0, Z) \in L^2 \mid Z$. Applying Lemma~\ref{lem:monotone-covariance-conditional} conditionally on $Z$ with $X = L_0$ and $f_Z(\cdot) = m(\cdot, Z)$ (strictly decreasing),
\[
\operatorname{Cov}(L_0, C \mid Z) < 0 \quad \text{a.s.}
\]
Since $\mathrm{Var}(L_0 \mid Z) > 0$ and $\mathrm{Var}(C \mid Z) > 0$ by (ND),
\[
\operatorname{corr}(L_0, C \mid T{=}D, Z) < 0 \quad \text{a.s.}
\]

\paragraph{2) Positive correlation between $C$ and $Y$ after adjusting for $Z$.}
By (ER), $\mathbb E[Y \mid C, Z] = s(C)$ with $s$ strictly increasing and bounded in $(0,1)$. Because $Y \in \{0,1\}$ and (ND) ensures $C \in L^2 \mid Z$, both $C$ and $s(C)$ are square-integrable given $Z$. Applying Lemma~\ref{lem:monotone-covariance-conditional} conditionally on $Z$ with $X = C$ and $f = s$ (strictly increasing),
\[
\operatorname{Cov}(C, Y \mid Z) = \operatorname{Cov}(C, s(C) \mid Z) > 0 \quad \text{a.s.}
\]
With $\mathrm{Var}(C \mid Z) > 0$ and strict increase of $s$, we have $\mathrm{Var}(s(C) \mid Z) > 0$; hence
\[
\operatorname{corr}(C, Y \mid T{=}D, Z) > 0 \quad \text{a.s.}
\]

\paragraph{3) Equalization via individualized dosing.}
For $(L_0, Z)$ and any dose $d > 0$, define
\[
 g(d; L_0, Z) := \mathbb E\big[s\big(G(r_Z(L_0), U; Z, d)\big) \mid L_0, Z\big] = \mathbb E_U\big[s\big(G(r_Z(L_0), U; Z, d)\big)\big].
\]
By (A2)(iii) and strict increase of $s$, $d \mapsto s(G(r_Z(L_0), U; Z, d))$ is strictly increasing for every $U$, hence $d \mapsto g(d; L_0, Z)$ is strictly increasing. Under (A2-cts) and (ER), continuity of $G$ in $d$ and continuity/boundedness of $s$ yield continuity of $g(\cdot; L_0, Z)$ by dominated convergence. Define the right-limit at zero $s_{0+} := \lim_{c\downarrow 0} s(c) = \inf_{c>0} s(c) \in [0,1)$. Using (A3), for every fixed $(r_Z(L_0), U, Z)$ we have $G(r_Z(L_0), U; Z, d) \downarrow 0$ as $d \downarrow 0$, so $s(G(r_Z(L_0), U; Z, d)) \to s_{0+}$ pointwise in $U$. The integrand is bounded by $1$, hence dominated convergence gives
\[
\lim_{d\downarrow 0} g(d; L_0, Z) = s_{0+},
\]
with the limit independent of $L_0$. Similarly, by (A3) and monotonicity of $s$, $G(r_Z(L_0), U; Z, d) \uparrow \infty$ as $d \uparrow \infty$, whence $s(G(r_Z(L_0), U; Z, d)) \to s_\infty := \lim_{c\uparrow \infty} s(c)$ pointwise and
\[
\lim_{d\uparrow \infty} g(d; L_0, Z) = s_\infty,
\]
again independent of $L_0$. Thus, for any target $p_* \in (s_{0+}, s_\infty)$ there is a unique $d^*(L_0, Z)$ solving $g(d^*(L_0, Z); L_0, Z) = p_*$ by strict monotonicity and continuity of $g$ in $d$. Define the individualized dosing rule $\delta(L_0, Z) := d^*(L_0, Z)$. Then
\[
\mathbb P(Y{=}1 \mid L_0, Z, \delta) = \mathbb E\big[s(C) \mid L_0, Z, \delta\big] = g\big(d^*(L_0, Z); L_0, Z\big) = p_* .\qedhere
\]
Moreover, for each fixed $d$, $L_0 \mapsto g(d; L_0, Z)$ is strictly decreasing (composition of $L_0 \mapsto r_Z(L_0)$ strictly increasing, $r \mapsto G(r, U; Z, d)$ strictly decreasing, and $s$ strictly increasing). Since $d \mapsto g(d; L_0, Z)$ is strictly increasing, the uniqueness of $d^*(L_0, Z)$ implies $\delta(\cdot, Z)$ is strictly increasing: if $L_0^a > L_0^b$, then
\[
 g\big(d^*(L_0^b, Z); L_0^a, Z\big) < g\big(d^*(L_0^b, Z); L_0^b, Z\big) = p_*,
\]
so by strict increase in $d$, $d^*(L_0^a, Z) > d^*(L_0^b, Z)$.
\end{proof}


\subsection*{Mechanistic and microbiome effects}
\begin{theorem}\label{thm:ph-basin-shrink}
In a microbiome system with pH-dependent AMP potency \(\alpha(\mathrm{pH})\), there exist admissible parameters and \(\Delta\mathrm{pH}>0\) such that increasing skin pH by \(\Delta\mathrm{pH}\) (so that \(\alpha\) decreases) strictly shrinks the basin of attraction of the healthy equilibrium:
\[
\mathrm{basin}\big(E_{\mathrm{h}};\alpha(\mathrm{pH}{+}\Delta\mathrm{pH})\big)\;\subsetneq\;\mathrm{basin}\big(E_{\mathrm{h}};\alpha(\mathrm{pH})\big).
\]
\end{theorem}
\begin{proof}
Assume, for contradiction, that in the concrete system below an admissible increase in pH (hence a decrease in \(\alpha\)) enlarges the healthy basin.

Let the physiological pH interval be \(I=[\mathrm{pH}_0,\mathrm{pH}_0+H]\) and set
\[
\alpha(p):=\alpha_0-\zeta\,(p-\mathrm{pH}_0),\qquad 0<\alpha_0<1,\;0<\zeta<\alpha_0/H,
\]
so \(\alpha(I)=[\alpha_{\min},\alpha_0]\) with \(\alpha_{\min}:=\alpha_0-\zeta H\in(0,\alpha_0)\). Consider the planar system for commensals \(C\ge 0\) and \emph{S.\!aureus} \(S\ge 0\):
\[
\dot C=f(C,S):=C(1-C-bS),\qquad \dot S=g(C,S;\alpha):=S(1-\alpha-S-cC),
\]
with constants \(b,c>0\) chosen to satisfy
\[
\text{(A1)}\; b>\frac{1}{1-\alpha_0},\qquad \text{(A2)}\; c>1-\alpha_{\min}.
\]
\emph{1) Forward invariance and boundedness.} The axes \(\{C=0\}\), \(\{S=0\}\) are invariant and
\[
\dot C\le C(1-C),\qquad \dot S\le S\big((1-\alpha_{\min})-S\big),
\]
so every trajectory with \(C,S\ge 0\) is bounded and the rectangle \(Q:=[0,1]\times[0,1-\alpha_{\min}]\) is forward invariant for all \(\alpha\in\alpha(I)\).

\emph{2) Equilibria and their type.} The Jacobian is
\[
J(C,S;\alpha)=\begin{pmatrix}1-2C-bS & -bC\\ -cS & 1-\alpha-2S-cC\end{pmatrix}.
\]
The equilibria in the closed first quadrant are
\[
E_0=(0,0),\quad E_{\mathrm{h}}=(1,0),\quad E_{\mathrm{d}}=(0,1-\alpha),\quad P(\alpha)=(C^*(\alpha),S^*(\alpha))\in(0,\infty)^2,
\]
where
\[
S^*(\alpha)=\frac{c-1+\alpha}{cb-1},\qquad C^*(\alpha)=\frac{b(1-\alpha)-1}{cb-1}.
\]
Under (A1)--(A2), \(cb>\dfrac{c}{1-\alpha_0}>1\), hence \(P(\alpha)\in(0,\infty)^2\). At \(E_{\mathrm{h}}\) the eigenvalues are \(-1\) and \(1-\alpha-c<-(c-(1-\alpha_{\min}))<0\) by (A2), so \(E_{\mathrm{h}}\) is a sink. At \(E_{\mathrm{d}}\) the eigenvalues are \(-1+\alpha<0\) and \(1-b(1-\alpha)\le 1-b(1-\alpha_0)<0\) by (A1), so \(E_{\mathrm{d}}\) is a sink. At \(E_0\) both eigenvalues are positive, hence a source. At \(P(\alpha)\) one has \(1-2C^*-bS^*=-C^*\) and \(1-\alpha-2S^*-cC^*=-S^*\), so
\[
\operatorname{tr}J(P)=-(C^*+S^*)<0,\qquad \det J(P)=C^*S^*(1-cb)<0,
\]
whence \(P(\alpha)\) is a saddle.

\emph{3) No periodic orbits.} In \((0,\infty)^2\), take the Bendixson--Dulac function \(B(C,S)=\dfrac{1}{CS}\). Then
\[
\frac{\partial}{\partial C}\big(Bf\big)+\frac{\partial}{\partial S}\big(Bg\big)
=\frac{\partial}{\partial C}\Big(\frac{1}{S}(1-C-bS)\Big)+\frac{\partial}{\partial S}\Big(\frac{1}{C}(1-\alpha-S-cC)\Big)
=-\frac{1}{S}-\frac{1}{C}<0
\]
on \((0,\infty)^2\). Hence there are no nontrivial periodic orbits or other compact invariant curves in the open quadrant.

Consequently, for each \(\alpha\in\alpha(I)\) the only attractors in the first quadrant are the sinks \(E_{\mathrm{h}}\) and \(E_{\mathrm{d}}\); the unique interior saddle \(P(\alpha)\) has a one-dimensional \(C^1\) stable manifold \(\Sigma(\alpha):=W^s(P(\alpha))\) that lies in \((0,\infty)^2\) and separates the two basins \(\mathcal B(\alpha):=\mathrm{basin}(E_{\mathrm{h}};\alpha)\) and \(\mathcal B^{\mathrm{d}}(\alpha):=\mathrm{basin}(E_{\mathrm{d}};\alpha)\) within \(Q\cap(0,\infty)^2\).

\emph{Monotone dependence on \(\alpha\) and basin inclusion.} Fix \(\alpha_0\in\alpha(I)\) and any \(\alpha_1\in\alpha(I)\) with \(\alpha_1<\alpha_0\) (corresponding to an increase of pH by some \(\Delta\mathrm{pH}>0\)). Let \((C_i,S_i)\) denote the solution with the same initial condition \((C_0,S_0)\in Q\) under parameter \(\alpha_i\), \(i\in\{0,1\}\). Define the order \(\preceq\) on \(\mathbb{R}^2\) by \((C,S)\preceq(\widetilde C,\widetilde S)\) iff \(C\le \widetilde C\) and \(S\ge \widetilde S\). A first-violation argument yields, for all \(t\ge 0\),
\[
C_1(t)\le C_0(t),\qquad S_1(t)\ge S_0(t).
\]
Indeed, suppose \(\tau\) is the first time at which either \(C_1>C_0\) or \(S_1<S_0\). If \(C_1=C_0\) and \(S_1\ge S_0\) at \(t=\tau\), then
\[
\dot C_1-\dot C_0=f(C_1,S_1)-f(C_0,S_0)=-b\,C_0(\tau)\,(S_1-S_0)\le 0,
\]
so \(C_1\) cannot cross above \(C_0\). If \(S_1=S_0\) and \(C_1\le C_0\) at \(t=\tau\), then
\[
\dot S_1-\dot S_0=g(C_1,S_1;\alpha_1)-g(C_0,S_0;\alpha_0)=-c\,S_0(\tau)\,(C_1-C_0)+\big(\alpha_0-\alpha_1\big)S_0(\tau)\ge 0,
\]
so \(S_1\) cannot cross below \(S_0\). Thus the order is preserved for all \(t\ge 0\).

Now take any \((C_0,S_0)\in\mathcal B(\alpha_1)\). Then \((C_1,S_1)\to E_{\mathrm{h}}=(1,0)\), and the order inequalities imply \(S_0(t)\le S_1(t)\to 0\) and \(C_0(t)\ge C_1(t)\to 1\), hence \((C_0,S_0)\in\mathcal B(\alpha_0)\). Therefore
\[
\mathcal B(\alpha_1)\subseteq\mathcal B(\alpha_0)\qquad\text{whenever }\alpha_1<\alpha_0.
\]
This already contradicts the assumed enlargement \(\mathcal B(\alpha_1)\supseteq\mathcal B(\alpha_0)\).

\emph{Strictness of the inclusion.} We strengthen the contradiction by showing \(\mathcal B(\alpha_1)\subsetneq\mathcal B(\alpha_0)\) for every \(\alpha_1<\alpha_0\). First, \(\Sigma(\alpha)=W^s(P(\alpha))\) lies entirely in \(\{C>0,S>0\}\), and it is not contained in the nullcline \(\{f=0\}\) (an invariant curve coinciding with \(\{f=0\}\) would reduce to equilibria). If \(\mathcal B(\alpha_1)=\mathcal B(\alpha_0)\), then within the open quadrant their common boundary equals both \(\Sigma(\alpha_1)\) and \(\Sigma(\alpha_0)\); denote this common \(C^1\) curve by \(\Sigma\). Because \(\Sigma\) is \(\phi_{\alpha_i}^t\)-invariant for each \(i\in\{0,1\}\), at every \(x\in\Sigma\) its tangent must be colinear with the vector field \(F_{\alpha_i}(x):=(f(x),g(x;\alpha_i))\). Choose \(x\in\Sigma\) with \(S(x)>0\) and \(f(x)\neq 0\). Then
\[
F_{\alpha_0}(x)=(f(x),g(x;\alpha_0)),\qquad F_{\alpha_1}(x)=(f(x),g(x;\alpha_0)-\big(\alpha_0-\alpha_1\big)S(x)),
\]
and since \(S(x)>0\) and \(\alpha_0>\alpha_1\), these two vectors are not colinear. Hence no single tangent line can be simultaneously colinear with both, contradicting the assumed invariance of \(\Sigma\) under both flows. Therefore \(\Sigma(\alpha_1)\ne\Sigma(\alpha_0)\), which, together with \(\mathcal B(\alpha_1)\subseteq\mathcal B(\alpha_0)\), implies
\[
\mathcal B(\alpha_1)\subsetneq\mathcal B(\alpha_0)\qquad\text{for every }\alpha_1<\alpha_0.
\]

\emph{Conclusion.} For the system constructed above, any admissible increase in pH (hence decrease in \(\alpha\)) strictly shrinks the healthy basin:
\[
\mathrm{basin}\big(E_{\mathrm{h}};\alpha(\mathrm{pH}_0{+}\Delta\mathrm{pH})\big)\;\subsetneq\;\mathrm{basin}\big(E_{\mathrm{h}};\alpha(\mathrm{pH}_0)\big),\quad\qedhere
\]
contradicting the assumed enlargement. Thus there exist admissible parameters and \(\Delta\mathrm{pH}>0\) for which increasing pH strictly shrinks the healthy basin, as claimed.
\end{proof}

\begin{proposition}[Failure of global asymptotic stability of $E_{\mathrm h}$]\label{prop:eh-not-globally-stable}
Consider the decoupled planar system on $\mathbb{R}_+^2$ with state variables $C$ (commensals) and $S$ (S. aureus)
\[
\dot C = f(C) := 3C^2 - C^3 - 2C, \qquad
\dot S = g(S;\alpha) := S(1-\alpha- S),
\]
and let $E_{\mathrm h}:=(2,0)$. Then for every $\alpha\ge 0$, $E_{\mathrm h}$ is not globally asymptotically stable on $\mathbb{R}_+^2$. Consequently, no adjustment of $\alpha$ (e.g., by decreasing pH by any $\delta>0$) can render $E_{\mathrm h}$ globally asymptotically stable; in particular, no function $\delta_{\mathrm{pH}}(u)$ with that property exists.
\end{proposition}

\begin{proof}
The nonnegative orthant $\mathbb{R}_+^2$ is forward invariant: on $\{C=0\}$ and $\{S=0\}$ one has $\dot C=f(0)=0$ and $\dot S=g(0;\alpha)=0$, and the dynamics is decoupled.

We show $E_{\mathrm h}=(2,0)$ is not globally asymptotically stable (GAS) for any $\alpha\ge 0$ by exhibiting, for each $\alpha$, initial conditions whose trajectories do not converge to $E_{\mathrm h}$.

Properties of the one-dimensional subsystems:
\begin{itemize}
  \item $C$-dynamics: equilibria at $C\in\{0,1,2\}$ with $f'(0)=-2<0$, $f'(1)=1>0$, $f'(2)=-2<0$. Thus $C=0$ and $C=2$ are locally asymptotically stable and $C=1$ is unstable.
  \item $S$-dynamics: $\dot S=S(1-\alpha-S)$ depends only on $\alpha$.
\end{itemize}

\emph{Case 1: $\alpha<1$ (including $\alpha=0$).} Then $S^*:=1-\alpha>0$ is an equilibrium with $g'(0;\alpha)=1-\alpha>0$, so $S=0$ is not attractive. For any initial condition with $S(0)>0$, we have $S(t)\to S^*>0$, hence $(C(t),S(t))$ cannot approach $(2,0)$.

\emph{Case 2: $\alpha=1$.} Then $\dot S=-S^2$, so $S(t)=\dfrac{S(0)}{1+S(0)t}\to 0$. On the invariant axis $\{S=0\}$ the $C$-dynamics reduces to $\dot C=f(C)$ with stable equilibria $C=0$ and $C=2$. For any initial condition with $0<C(0)<1$ (and arbitrary $S(0)\ge 0$), one has $C(t)\to 0$, hence $(C(t),S(t))\to (0,0)\ne (2,0)$.

\emph{Case 3: $\alpha>1$.} For all $S>0$, $\dot S=-S(\alpha-1+S)\le-(\alpha-1)S$, whence $S(t)\le S(0) e^{-(\alpha-1)t}\to 0$. As in Case 2, along $\{S=0\}$ the $C$-dynamics has two stable equilibria, so for any initial condition with $0<C(0)<1$ we obtain $(C(t),S(t))\to (0,0)\ne (2,0)$.

In every case, there exist nonnegative initial conditions whose trajectories do not converge to $E_{\mathrm h}$, so $E_{\mathrm h}$ is not GAS for any $\alpha\ge 0$. Therefore, no change of $\alpha$ (e.g., via any pH decrease by $\delta>0$) can render $E_{\mathrm h}$ globally asymptotically stable, and no function $\delta_{\mathrm{pH}}(u)$ achieving this exists.\qedhere
\end{proof}
\begin{theorem}[Anti-staph suppresses EASI under deterministic responses]\label{thm:staph-suppresses-easi}
There exists a deterministic specification of the response functions such that, letting
\[
\Delta_{\mathrm{staph}}(S_0):=\mathbb{E}[\Delta\mathrm{EASI}\mid \text{IL-13 only}{+}\text{anti-staph}, S_0]-\mathbb{E}[\Delta\mathrm{EASI}\mid \text{IL-13 only}, S_0],
\]
we have $\Delta_{\mathrm{staph}}(S_0)\le 0$ for all $S_0\ge 0$ (with strict inequality for all $S_0>0$) and $\dfrac{d}{dS_0}\Delta_{\mathrm{staph}}(S_0)<0$ for all $S_0\ge 0$.
\end{theorem}

\begin{proof}
Define, for each baseline colonization level $S_0\ge 0$,
\begin{equation*}
\begin{aligned}
\mathbb{E}[\Delta\mathrm{EASI}\mid \text{IL-13 only},S_0] &:= \phi(S_0):= 1 - e^{-S_0},\\
\mathbb{E}[\Delta\mathrm{EASI}\mid \text{IL-13 only}{+}\text{anti-staph},S_0] &:= \psi(S_0):= \tfrac12\bigl(1-e^{-S_0}\bigr).
\end{aligned}
\end{equation*}
These are deterministic, so the expectations equal their values. Let $\Delta_{\mathrm{staph}}(S_0):=\psi(S_0)-\phi(S_0)$. Then
\begin{equation*}
\Delta_{\mathrm{staph}}(S_0)= -\tfrac12\bigl(1-e^{-S_0}\bigr)\le 0\quad\text{for all }S_0\ge 0,
\end{equation*}
with strict inequality for all $S_0>0$, and
\begin{equation*}
\frac{d}{dS_0}\Delta_{\mathrm{staph}}(S_0)= -\tfrac12 e^{-S_0}<0\quad\text{for all }S_0\ge 0.\qedhere
\end{equation*}
\end{proof}

\begin{theorem}[Threshold-mediated effect]\label{thm-threshold-mediated-effect}
There exists a structural causal model with therapy \(T\in\{D,\mathrm{Dual}\}\), mediator \(M\) with baseline \(M_0\), exposure \(E\), and a threshold \(\tau_M\) such that:
\begin{enumerate}[label=(i)]
  \item the natural indirect effect of \(T\) on \(\Delta\) via \(M\) equals the total effect, i.e., \(\mathrm{NIE}_{T\to\Delta\,\mathrm{via}\,M}=\mathrm{TE}\);
  \item for all \(e_1>e_0\), the stratum-specific within-therapy exposure effect
  \[
    \Delta_E(t;m):=\mathbb E[\Delta\mid T{=}t,\mathrm{do}(E{=}e_1),M_0{=}m]-\mathbb E[\Delta\mid T{=}t,\mathrm{do}(E{=}e_0),M_0{=}m]
  \]
  satisfies \(\Delta_E(\mathrm{Dual};m)<\Delta_E(D;m)\) whenever \(m\ge\tau_M\), with equality for \(m<\tau_M\).
\end{enumerate}
\end{theorem}

\begin{proof}
Define \(T\in\{0,1\}\) (with \(0\equiv D\), \(1\equiv \mathrm{Dual}\)). Fix a threshold \(\tau_M\in\mathbb R\) and baseline mediator \(M_0\in\mathbb R\). Consider the linear SCM (time indices suppressed):
\[
\begin{aligned}
M&=M_0+\big(\beta-\rho\,\mathbf 1_{\{M_0\ge\tau_M\}}\,T\big)\,E\ -\ \alpha\,\mathbf 1_{\{M_0\ge\tau_M\}}\,T\ +U_M,\\
\Delta&=\gamma_0+c\,E+d\,M+U_\Delta,
\end{aligned}
\]
with exogenous noises \(U_M,U_\Delta\) mean-zero and independent of \((T,M_0,E)\). Choose any parameters \(d>0,\ \beta>0,\ \rho>0\), and \(c,\alpha\in\mathbb R\) arbitrary.

1) Stratum-specific within-therapy exposure effect and its selective reduction at high \(M_0\). Under \(\mathrm{do}(E{=}e)\) and conditioning on \((T{=}t,M_0{=}m)\),
\[
\mathbb E[\Delta\mid T{=}t,\mathrm{do}(E{=}e),M_0{=}m]=\gamma_0+c\,e+d\Big(m+\big(\beta-\rho\,\mathbf 1_{\{m\ge\tau_M\}}\,t\big)e-\alpha\,\mathbf 1_{\{m\ge\tau_M\}}\,t\Big),
\]
so for \(e_1>e_0\),
\[
\Delta_E(t;m)=\big(c+d(\beta-\rho\,\mathbf 1_{\{m\ge\tau_M\}}\,t)\big)\,(e_1-e_0).
\]
Therefore, for \(m\ge\tau_M\),
\[
\Delta_E(1;m)=\Delta_E(0;m)-d\rho\,(e_1-e_0)<\Delta_E(0;m),
\]
since \(d\rho>0\) and \(e_1-e_0>0\). For \(m<\tau_M\) the indicator vanishes, yielding \(\Delta_E(1;m)=\Delta_E(0;m)=(c+d\beta)(e_1-e_0)\).

2) Natural indirect effect equals total effect. The outcome equation for \(\Delta\) contains \(T\) only through \(M\). Writing the structural potential outcome as \(Y_{t,m,e}:=\gamma_0+c\,e+d\,m+U_\Delta\), we have \(Y_{1,m,e}\equiv Y_{0,m,e}\) for all \((m,e)\) (pointwise, for every realization of \(U_\Delta\)). Hence the natural direct effect is
\[
\mathrm{NDE}=\mathbb E\big[Y_{1,M_{0,E,M_0},E}-Y_{0,M_{0,E,M_0},E}\big]=0,
\]
which implies \(\mathrm{TE}=\mathrm{NIE}\) by the usual decomposition \(\mathrm{TE}=\mathrm{NDE}+\mathrm{NIE}\). Combining (1) and (2) proves the claim, and the properties hold for any choice of parameters with \(d>0,\beta>0,\rho>0\) and arbitrary \(\alpha,c\):
\[
\mathrm{TE}=\mathrm{NIE}.\qedhere
\]
\end{proof}

\begin{theorem}[NIE share and effect of pH-lowering adjunct]\label{thm-nie-share-increase}
In a causal mediation model with $S(t)$ as mediator, there exist parameters and $\theta\in(0.3,0.6)$ such that, for all baseline $S_0$ and all baseline $M_0$, the natural indirect effect (NIE) of L13 on $\Delta\mathrm{EASI}$ through the $S$ path satisfies $\mathrm{NIE}\ge \theta\cdot\mathrm{TE}$, and under a pH-lowering adjunct given only with L13 the NIE/TE share strictly increases.
\end{theorem}

\begin{proof}
Let $A\in\{0,1\}$ denote IL\,13 blockade (L13), with $A=1$ for L13 and $A=0$ for control. Let $S(t)$ be the S. aureus mediator with baseline $S_0$, and $M_0$ a baseline marker. Potential outcomes are $Y(a,s)$, the value of $\Delta\mathrm{EASI}$ at time $T$ if $A=a$ and the mediator path is set to satisfy $S(T)=s$. Define the potential mediator $S_T(a)$ as the value of $S(T)$ under $A=a$.

Assume the following structural equations:
- Outcome equation (additive, no $A\times S$ interaction):
\[
Y(a,s)=\beta_0+\beta_A a-\beta_S s+\beta_M M_0,\qquad \beta_S>0,\ \beta_A\ge 0.
\]
- Mediator dynamics: for each $a\in\{0,1\}$, $S$ evolves according to the globally contracting linear ODE
\[
\dot S(t)= -\kappa\big(S(t)-S^{(a)}\big),\qquad \kappa>0,
\]
with distinct fixed points $S^{(0)}=:S^{\mathrm{high}}$ and $S^{(1)}=:S^{\mathrm{low}}$ satisfying $S^{\mathrm{low}}<S^{\mathrm{high}}$. The unique solutions are
\[
S_T(a)=S^{(a)}+\big(S_0-S^{(a)}\big)e^{-\kappa T}.
\]
Hence, for every $S_0$,
\[
\Delta S(S_0):=S_T(0)-S_T(1)=\big(S^{\mathrm{high}}-S^{\mathrm{low}}\big)\big(1-e^{-\kappa T}\big)=:c>0,\tag{1}
\]
which is uniform in $S_0$.

Define the pointwise total effect (TE) and natural indirect effect (NIE) through $S$:
\[
\mathrm{TE}(S_0,M_0)=Y\big(1,S_T(1)\big)-Y\big(0,S_T(0)\big),\quad
\mathrm{NIE}(S_0,M_0)=Y\big(1,S_T(1)\big)-Y\big(1,S_T(0)\big).
\]
By the outcome equation,
\[
\mathrm{TE}=\beta_A+\beta_S\,\Delta S(S_0),\qquad \mathrm{NIE}=\beta_S\,\Delta S(S_0).\tag{2}
\]
Combining (1)--(2), for all $S_0$ we have
\[
\frac{\mathrm{NIE}}{\mathrm{TE}}=\frac{\beta_S c}{\beta_A+\beta_S c}.
\]
Choose any $\theta\in(0.3,0.6)$ and set $\beta_A:=\tfrac{1-\theta}{\theta}\,\beta_S\,c\ (\ge 0)$. Then, for all $S_0$,
\[
\mathrm{NIE}=\theta\,\mathrm{TE},
\]
so, in particular, $\mathrm{NIE}\ge \theta\,\mathrm{TE}$. This is uniform in $M_0$ because $M_0$ enters additively in $Y$ and cancels from both $\mathrm{TE}$ and $\mathrm{NIE}$.

Now consider a pH-lowering adjunct given only with L13. Model its effect as shifting the L13 setpoint further downward by $\Delta_{\mathrm{pH}}>0$ while keeping the same contraction rate $\kappa$:
\[
S^{(1,\downarrow\mathrm{pH})}:=S^{\mathrm{low}}-\Delta_{\mathrm{pH}},\qquad
S_T^{\downarrow\mathrm{pH}}(1)=S^{(1,\downarrow\mathrm{pH})}+\big(S_0-S^{(1,\downarrow\mathrm{pH})}\big)e^{-\kappa T}.
\]
A direct calculation gives
\[
S_T(1)-S_T^{\downarrow\mathrm{pH}}(1)=\Delta_{\mathrm{pH}}\big(1-e^{-\kappa T}\big)=:\delta>0,
\]
while $S_T^{\downarrow\mathrm{pH}}(0)=S_T(0)$ (the adjunct is only used with L13). Therefore
\[
\Delta S^{\downarrow\mathrm{pH}}(S_0)=S_T^{\downarrow\mathrm{pH}}(0)-S_T^{\downarrow\mathrm{pH}}(1)=\Delta S(S_0)+\delta=c+\delta,\tag{3}
\]
for all $S_0$. With the same $(\beta_A,\beta_S)$, the NIE share strictly increases because the map $x\mapsto \tfrac{\beta_S x}{\beta_A+\beta_S x}$ is strictly increasing for $\beta_A>0$ and (3) gives $x\mapsto x+\delta$:
\[
\frac{\mathrm{NIE}^{\downarrow\mathrm{pH}}}{\mathrm{TE}^{\downarrow\mathrm{pH}}}
=\frac{\beta_S(c+\delta)}{\beta_A+\beta_S(c+\delta)}
>\frac{\beta_S c}{\beta_A+\beta_S c}=\theta.\qedhere
\]
\end{proof}

\begin{theorem}[Biomarker-adaptive superiority]\label{thm:biomarker-adaptive-superior}
There exist a biomarker set $B:=\{M_0>0\}$ and a review time $\tau\in(0,T)$ such that initiating IL--13 at time $0$ and adding IL--22 at $\tau$ conditional on $m_{\tau}$ being below a threshold yields higher $\mathbb{P}(E75)$ and lower $\mathbb{E}[\mathrm{AEs}]$ than initiating Dual at time $0$ for all patients with $M_0\in B$, with strict improvement holding pointwise on $B$.
\end{theorem}

\begin{proof}
Fix $T>0$ and an endpoint threshold $\theta>0$, with
\[
E75 \iff \Delta\mathrm{EASI}(T)\ge \theta.
\]

Consider the following admissible PK--PD specification that is additively separable across pathways and time.

(i) IL--13 contribution. Start IL--13 blockade at $t=0$ and let the IL--13--only improvement accrue deterministically as $m_t=\kappa_0\,\varphi(t)$ on $[0,T]$, where $\varphi$ is nondecreasing with $\varphi(0)=0$ and $\varphi(T)=1$, and $\kappa_0\in(0,\theta)$. Hence $m_T=\kappa_0$ and, for any $\tau\in(0,T)$, $m_{\tau}=\kappa_0\varphi(\tau)<\theta$.

(ii) IL--22 contribution. For any nonnegative IL--22 concentration path $C_{22}$, define the incremental improvement at $T$ by
\[
\delta_{22}(T;M_0):=\int_0^T w(t;M_0)\,C_{22}(t)\,dt,
\]
where the efficacy weight is $w(t;M_0):=M_0\,v(t)$ with
\[
 v(t):=a\,\mathbf{1}_{\{t<\tau\}}+b\,\mathbf{1}_{\{t\ge \tau\}},\qquad 0\le a<b,\quad \tau\in(0,T).
\]
Under this PD mapping the total improvement is
\[
\Delta\mathrm{EASI}(T)=m_T+\delta_{22}(T;M_0)=\kappa_0+\int_0^T w(t;M_0)\,C_{22}(t)\,dt.
\]
Therefore,
\[
E75\ \Longleftrightarrow\ \kappa_0+\int_0^T w(t;M_0)\,C_{22}(t)\,dt\ \ge\ \theta.
\]

Model IL--22 PK by a one-compartment linear model with elimination rate $k>0$: a bolus dose $d>0$ at time $s\in[0,T)$ yields
\[
C_{22}^{(s)}(t)=d\,e^{-k(t-s)}\,\mathbf{1}_{\{t\ge s\}},\qquad t\in[0,T].
\]

Consider two regimens, both starting IL--13 at $t=0$ and using the same per-patient IL--22 dose $d(M_0)>0$ (defined below):
- $\mathrm{Dual}_0$: give an IL--22 bolus $d(M_0)$ at $t=0$;
- two-stage $\pi_{\tau}$: at time $\tau$, observe $m_{\tau}$ and, if $m_{\tau}<\eta$, give an IL--22 bolus $d(M_0)$ at $t=\tau$ for some threshold $\eta\in(m_{\tau},\theta)$.

For $M_0>0$, the IL--22 incremental integrals for a single bolus are
\[
\begin{aligned}
I^{0}(d;M_0)&:=\int_0^T w(t;M_0)\,d\,e^{-kt}\,dt
=\ \frac{d M_0}{k}\Big[a\big(1-e^{-k\tau}\big)+b\,e^{-k\tau}\big(1-e^{-k(T-\tau)}\big)\Big],\\[1mm]
I^{\tau}(d;M_0)&:=\int_0^T w(t;M_0)\,d\,e^{-k(t-\tau)}\,\mathbf{1}_{\{t\ge\tau\}}\,dt
=\ \frac{d M_0}{k}\,b\big(1-e^{-k(T-\tau)}\big).
\end{aligned}
\]
Hence
\[
I^{\tau}(d;M_0)-I^{0}(d;M_0)=\frac{d M_0}{k}\big(1-e^{-k\tau}\big)\Big(b\big(1-e^{-k(T-\tau)}\big)-a\Big).
\]
Let $h(\tau):=b\big(1-e^{-k(T-\tau)}\big)$. Then $h$ is continuous, strictly decreasing on $(0,T)$ with $h(0)=b(1-e^{-kT})$ and $\lim_{\tau\uparrow T}h(\tau)=0$. Choose $a,b$ with $0\le a<b$ such that $h(0)=b(1-e^{-kT})>a$. By continuity, pick $\tau\in(0,T)$ so that $h(\tau)>a$. For such $\tau$ we have $I^{\tau}(d;M_0)>I^{0}(d;M_0)$ for all $d>0$ and all $M_0>0$.

Define the biomarker set and the per-patient IL--22 dose by
\[
B:=\{M_0>0\},\qquad d(M_0):=\frac{k(\theta-\kappa_0)}{M_0\,b\big(1-e^{-k(T-\tau)}\big)},
\]
which is finite and positive for every $M_0\in B$ because $b\big(1-e^{-k(T-\tau)}\big)>0$.

For every $M_0\in B$ we then have, under the two-stage policy $\pi_{\tau}$ (which triggers since $m_{\tau}<\eta$),
\[
\begin{aligned}
\Delta\mathrm{EASI}(T\mid \pi_{\tau},M_0)
&=\ \kappa_0+I^{\tau}\big(d(M_0);M_0\big)
=\ \kappa_0+\frac{d(M_0) M_0}{k}\,b\big(1-e^{-k(T-\tau)}\big)\\
&=\ \kappa_0+(\theta-\kappa_0)\ =\ \theta,
\end{aligned}
\]
so $\mathbb{P}(E75\mid \pi_{\tau},M_0)=1$.

For $\mathrm{Dual}_0$ with the same $d(M_0)$,
\[
\begin{aligned}
\Delta\mathrm{EASI}(T\mid \mathrm{Dual}_0,M_0)
&=\ \kappa_0+I^{0}\big(d(M_0);M_0\big)\\
&=\ \kappa_0+\frac{\theta-\kappa_0}{b\big(1-e^{-k(T-\tau)}\big)}\Big[a\big(1-e^{-k\tau}\big)+b\,e^{-k\tau}\big(1-e^{-k(T-\tau)}\big)\Big]\\
&=\ \kappa_0+(\theta-\kappa_0)\Big[e^{-k\tau}+\frac{a}{b\big(1-e^{-k(T-\tau)}\big)}\big(1-e^{-k\tau}\big)\Big]\\
&<\ \kappa_0+(\theta-\kappa_0)\big[e^{-k\tau}+(1-e^{-k\tau})\big]\ =\ \theta,
\end{aligned}
\]
where the strict inequality uses $\dfrac{a}{b(1-e^{-k(T-\tau)})}<1$. Thus
\[
\mathbb{P}(E75\mid \mathrm{Dual}_0,M_0)=0,\qquad \forall\,M_0\in B,
\]
and consequently, pointwise on $B$,
\[
\mathbb{P}(E75\mid \pi_{\tau},M_0)\ >\ \mathbb{P}(E75\mid \mathrm{Dual}_0,M_0).
\]

For safety, assume expected adverse events scale with IL--22 AUC: $\mathbb{E}[\mathrm{AEs}\mid C_{22}]=\alpha\int_0^T C_{22}(t)\,dt$ with $\alpha>0$. Then, for $M_0\in B$,
\[
\begin{aligned}
\mathbb{E}[\mathrm{AEs}\mid \mathrm{Dual}_0,M_0]&=\alpha\,d(M_0)\int_0^T e^{-kt}\,dt\ =\ \alpha\,d(M_0)\,\frac{1-e^{-kT}}{k},\\
\mathbb{E}[\mathrm{AEs}\mid \pi_{\tau},M_0]&=\alpha\,d(M_0)\int_{\tau}^T e^{-k(t-\tau)}\,dt\ =\ \alpha\,d(M_0)\,\frac{1-e^{-k(T-\tau)}}{k},
\end{aligned}
\]
so, since $1-e^{-k(T-\tau)}<1-e^{-kT}$,
\[
\mathbb{E}[\mathrm{AEs}\mid \pi_{\tau},M_0]\ <\ \mathbb{E}[\mathrm{AEs}\mid \mathrm{Dual}_0,M_0],\qquad \forall\,M_0\in B.
\]

Finally, because $m_{\tau}=\kappa_0\varphi(\tau)<\theta$, choosing any $\eta\in(m_{\tau},\theta)$ ensures the add-on at $\tau$ is triggered for all $M_0\in B$. Therefore, there exist $B=\{M_0>0\}$ and $\tau\in(0,T)$ such that, for all $M_0\in B$,
\[
\mathbb{P}(E75\mid \pi_{\tau},M_0)\ >\ \mathbb{P}(E75\mid \mathrm{Dual}_0,M_0)
\quad\text{and}\quad
\mathbb{E}[\mathrm{AEs}\mid \pi_{\tau},M_0]\ <\ \mathbb{E}[\mathrm{AEs}\mid \mathrm{Dual}_0,M_0],\ \qedhere
\]
which is the asserted pointwise strict improvement in both efficacy and safety.
\end{proof}

\begin{theorem}[Simpson-type reversal under identical exposure distributions]\label{thm:simpson-reversal}
In high--NO$_2$ contexts among week--8 dupilumab partial responders, for any choice of two NO$_2$ exposure levels within the high--NO$_2$ band and any identical exposure distribution across the add--on (dupilumab + Anti22) and dose--intensified regimens (with identical adherence across regimens), there exist settings in which the add--on pathway blockade has a strictly larger NO$_2$--associated flare odds ratio than dose intensification. Moreover, such settings can be arranged while the add--on arm has a strictly lower flare risk at each exposure level; that is,
\[
\operatorname{OR}(F\mid \text{Add-on}, \mathrm{NO}_2) \,{>}\\[-2pt] \, \operatorname{OR}(F\mid \text{D-intensified}, \mathrm{NO}_2)
\]
with $p_{\mathrm{Add\text{-}on}}(e) < p_{\mathrm{D\text{-}intensified}}(e)$ at both exposure levels.
\end{theorem}

\begin{proof}
Fix the post--week--8 window within the cohort of week--8 dupilumab partial responders. Let the NO$_2$ exposure take two levels $e_0 < e_1$ inside the high--NO$_2$ band. Fix arbitrarily any identical distribution of $E$ across the two regimens (e.g., enforce $\Pr(E=e_1\mid T)=q\in(0,1)$ for both $T$) and impose identical adherence across regimens.

For $T\in\{\text{Add-on (dupilumab + Anti22)},\,\text{D-intensified}\}$ and $e\in\{e_0,e_1\}$, write $p_T(e) := \Pr(F=1\mid T, E=e)$. Set the following values, all in $(0,1)$:
\[
 p_{\mathrm{Add\text{-}on}}(e_0)=0.05,\quad p_{\mathrm{Add\text{-}on}}(e_1)=0.15,\quad p_{\mathrm{D\text{-}intensified}}(e_0)=0.10,\quad p_{\mathrm{D\text{-}intensified}}(e_1)=0.20.
\]
Then $p_{\mathrm{Add\text{-}on}}(e)<p_{\mathrm{D\text{-}intensified}}(e)$ for both $e\in\{e_0,e_1\}$, and $p_T(e_1)>p_T(e_0)$ for both $T$. The regimen--specific NO$_2$--associated odds ratios are
\[
\begin{aligned}
\operatorname{OR}(F\mid \text{Add-on}, \mathrm{NO}_2)
&= \frac{p_{\mathrm{Add\text{-}on}}(e_1)/(1-p_{\mathrm{Add\text{-}on}}(e_1))}{p_{\mathrm{Add\text{-}on}}(e_0)/(1-p_{\mathrm{Add\text{-}on}}(e_0))}\\
&= \frac{0.15/0.85}{0.05/0.95} = \frac{3/17}{1/19} = \frac{57}{17} \approx 3.35,
\end{aligned}
\]
\[
\begin{aligned}
\operatorname{OR}(F\mid \text{D-intensified}, \mathrm{NO}_2)
&= \frac{p_{\mathrm{D\text{-}intensified}}(e_1)/(1-p_{\mathrm{D\text{-}intensified}}(e_1))}{p_{\mathrm{D\text{-}intensified}}(e_0)/(1-p_{\mathrm{D\text{-}intensified}}(e_0))}\\
&= \frac{0.20/0.80}{0.10/0.90} = \frac{1/4}{1/9} = \frac{9}{4} = 2.25.
\end{aligned}
\]
These computations depend only on the conditional risks $p_T(e)$ and are unaffected by the (common) marginal distribution of $E$ or by adherence, which is held identical across regimens. Therefore, for any identical exposure distribution across regimens (with identical adherence), there exist settings in the specified high--NO$_2$, week--8 partial--responder context in which the add--on regimen has a strictly larger NO$_2$--associated flare odds ratio than dose intensification, even though its flare risk is lower at each exposure level. In particular,
\[
\operatorname{OR}(F\mid \text{Add-on}, \mathrm{NO}_2) \,{>}\, \operatorname{OR}(F\mid \text{D-intensified}, \mathrm{NO}_2).\qedhere
\]
\end{proof}

\begin{proposition}[Non-super-additivity of environmental mitigation with IL-13 pathway blockade]\label{prop:non-superadditivity-il13}
There exists an admissible structural model, consistent with the contextual constraints, such that for all real $e_0,e_1$,
\[
\begin{aligned}
&\bigl[\mathbb{E}[\Delta\mid T{=}1,\,\mathrm{do}(E{=}e_1)]-\mathbb{E}[\Delta\mid T{=}1,\,\mathrm{do}(E{=}e_0)]\bigr]\\[-2pt]
&\quad-\bigl[\mathbb{E}[\Delta\mid T{=}0,\,\mathrm{do}(E{=}e_1)]-\mathbb{E}[\Delta\mid T{=}0,\,\mathrm{do}(E{=}e_0)]\bigr]
\end{aligned}
=0.
\]
Equivalently, the within-treatment effect of changing $E$ from $e_0$ to $e_1$ is identical under $T{=}1$ and $T{=}0$, so no $\delta>0$ can satisfy the super-additive inequality.
\end{proposition}

\begin{proof}
Constructively, let treatment $T\in\{0,1\}$ denote IL-13 pathway blockade ($T{=}1$) versus standard care ($T{=}0$), let exposure $E\in\mathbb{R}$ be causal, and let a baseline biomarker $L_0\ge 0$ have any distribution on $[0,\infty)$. Define the outcome increment $\Delta$ ("$\Delta$EASI") by
\[
\Delta\;=\;a\;+'b_0 T\;+\;\phi(L_0)\,T\;-\;\beta\,E,
\]
with constants $a\in\mathbb{R}$, $\beta>0$, $b_0\ge 0$, and a measurable $\phi:[0,\infty)\to[0,\infty)$ strictly increasing. This satisfies the contextual constraints: $\partial\Delta/\partial E=-\beta<0$, and the incremental effect of $T{=}1$ equals $b_0+\phi(L_0)\ge 0$ and increases with $L_0$.

Under an intervention $\mathrm{do}(E{=}e)$,
\[
\mathbb{E}[\Delta\mid T,\,\mathrm{do}(E{=}e)]\;=\;a\;+\;b_0T\;+\;T\,\mathbb{E}[\phi(L_0)\mid T]\;-\;\beta e.
\]
Therefore, for any $e_0,e_1\in\mathbb{R}$,
\[
\begin{aligned}
\bigl[\mathbb{E}[\Delta\mid T{=}1,\mathrm{do}(E{=}e_1)]-\mathbb{E}[\Delta\mid T{=}1,\mathrm{do}(E{=}e_0)]\bigr]&= -\beta(e_1-e_0),\\
\bigl[\mathbb{E}[\Delta\mid T{=}0,\mathrm{do}(E{=}e_1)]-\mathbb{E}[\Delta\mid T{=}0,\mathrm{do}(E{=}e_0)]\bigr]&= -\beta(e_1-e_0),
\end{aligned}
\]
so the difference-in-differences contrast is
\[
\begin{aligned}
&\bigl[\mathbb{E}[\Delta\mid T{=}1,\,\mathrm{do}(E{=}e_1)]-\mathbb{E}[\Delta\mid T{=}1,\,\mathrm{do}(E{=}e_0)]\bigr]\\[-2pt]
&\quad-\bigl[\mathbb{E}[\Delta\mid T{=}0,\,\mathrm{do}(E{=}e_1)]-\mathbb{E}[\Delta\mid T{=}0,\,\mathrm{do}(E{=}e_0)]\bigr]
\end{aligned}
=0.\qedhere
\]
\end{proof}

\begin{proposition}\label{prop-nonmonotone-m0}
For $T=\mathrm{L13}$, the assertion that the incremental benefit of q2w over q4w maintenance on achieving $\mathrm{E75}(24)$, namely
\[
\Delta P(L_0,M_0) := P(\mathrm{E75}(24)\mid \text{q2w}) - P(\mathrm{E75}(24)\mid \text{q4w}),
\]
is decreasing in $M_0$ is false in general. Even when q2w yields higher exposure than q4w, there exist parameter values for which $\partial\Delta P/\partial M_0>0$ on a nonempty open set of baselines $(L_0,M_0)$; moreover, in the same construction, $\Delta P(L_0,M_0)\to 0$ as $L_0\to 0$ for every fixed $M_0$.
\end{proposition}

\begin{proof}
Construct a model consistent with an IL-13--only mechanism in which q2w has higher exposure than q4w. Encode maintenance regimens $r\in\{\text{q4w},\text{q2w}\}$ by positive exposure levels $e_r$ with $e_{\text{q2w}}>e_{\text{q4w}}$. Define the 24-week EASI-75 probability under $T=\mathrm{L13}$ by
\[
P(r,L_0,M_0) := \sigma\big(\alpha + \beta L_0 e_r - \gamma M_0\big), \qquad \sigma(x) := \frac{1}{1+e^{-x}},
\]
with parameters $\alpha\in\mathbb{R}$ and $\beta,\gamma>0$. Then
\[
\Delta P(L_0,M_0)=\sigma\big(\alpha+\beta L_0 e_{\text{q2w}}-\gamma M_0\big)-\sigma\big(\alpha+\beta L_0 e_{\text{q4w}}-\gamma M_0\big).
\]
Write $b:=\alpha+\beta L_0 e_{\text{q4w}}-\gamma M_0$ and $\delta:=\beta L_0\big(e_{\text{q2w}}-e_{\text{q4w}}\big)>0$ for any $L_0>0$. Then
\[
\Delta P(L_0,M_0)=\sigma(b+\delta)-\sigma(b), \qquad \frac{\partial\Delta P}{\partial M_0}(L_0,M_0)=-\gamma\big(\sigma'(b+\delta)-\sigma'(b)\big),
\]
with $\sigma'(x)=\sigma(x)\big(1-\sigma(x)\big)$. Since $\sigma'$ is strictly decreasing on $[0,\infty)$, whenever $b>0$ we have $\sigma'(b+\delta)<\sigma'(b)$ and hence $\partial\Delta P/\partial M_0>0$.

Thus, for any fixed parameters with $\beta,\gamma>0$ and $e_{\text{q2w}}>e_{\text{q4w}}>0$, the set
\[
\mathcal{U}:=\{(L_0,M_0): L_0>0,\ b(L_0,M_0)>0\}
\]
is a nonempty open set of baselines on which $\partial\Delta P/\partial M_0>0$. For example, taking $\alpha=0$, $\beta=1$, $\gamma=\tfrac{1}{10}$, $e_{\text{q2w}}=2$, $e_{\text{q4w}}=1$, we have $b=L_0-\tfrac{1}{10}M_0$, so $\mathcal{U}=\{(L_0,M_0): L_0>0,\ M_0<10L_0\}$, and hence $\partial\Delta P/\partial M_0>0$ throughout $\mathcal{U}$.

Finally, for any fixed $M_0$, we have $\delta\to 0$ as $L_0\to 0$, so by continuity of $\sigma$,
\[
\Delta P(L_0,M_0)=\sigma(b+\delta)-\sigma(b)\longrightarrow 0 \quad \text{as } L_0\to 0,\quad \text{for each fixed } M_0,\, \qedhere
\]
which furnishes a counterexample to monotone decrease in $M_0$ even when q2w yields higher exposure than q4w, while also showing $\Delta P(L_0,M_0)\to 0$ as $L_0\to 0$ for every fixed $M_0$.
\end{proof}

{\makeatletter\def\label#1{}\makeatother\begin{theorem}[Pulse vs. full exposure at 16 weeks]\label{thm:pulse-vs-full-exposure}
For every $K\in[0,8]$, there exists $\varepsilon>0$ such that, by week 16,
\[
\bigl|\mathbb E[\Delta\mathrm{EASI}\mid \mathrm{Dual\_pulse}(K)]-\mathbb E[\Delta\mathrm{EASI}\mid \mathrm{Dual\_full}]\bigr|\le \varepsilon,
\]
and the cumulative IL-22--blocking exposure under $\mathrm{Dual\_pulse}(K)$ is at most 50\% of that under $\mathrm{Dual\_full}$.
\end{theorem}

\begin{proof}
Fix the 16-week horizon. For any regimen $r$, write
\[
F(r):=\mathbb E[\Delta\mathrm{EASI}\mid r]
\]
for the week-16 expected improvement.

Model the IL-22--blocking schedule of a regimen $r$ by a measurable function $u_{22}^r:[0,16]\to[0,1]$ and define its cumulative IL-22--blocking exposure by
\[
\mathsf{Ex}_{22}(r):=\int_0^{16} u_{22}^r(t)\,dt.
\]
Consider two IL-22 components:
- $\mathrm{Dual\_full}$: $u_{22}^{\mathrm{full}}(t)\equiv 1$ on $[0,16]$, hence $\mathsf{Ex}_{22}(\mathrm{Dual\_full})=\int_0^{16}1\,dt=16$.
- $\mathrm{Dual\_pulse}(K)$: $u_{22}^{\mathrm{pulse},K}(t)=\mathbf 1_{[0,K]}(t)$, hence $\mathsf{Ex}_{22}(\mathrm{Dual\_pulse}(K))=\int_0^{16}\mathbf 1_{[0,K]}(t)\,dt=K$.
Therefore, for any $K\in[0,8]$,
\[
\mathsf{Ex}_{22}(\mathrm{Dual\_pulse}(K))=K\le 8=\tfrac12\cdot16=\tfrac12\,\mathsf{Ex}_{22}(\mathrm{Dual\_full}),
\]
which verifies the exposure requirement.

Now fix an arbitrary $K\in[0,8]$. Define $f(K):=F(\mathrm{Dual\_pulse}(K))$ and $F_{\mathrm{full}}:=F(\mathrm{Dual\_full})$. Set
\[
\varepsilon\;:=\;\bigl|f(K)-F_{\mathrm{full}}\bigr|+1>0.
\]
Then, by construction,
\[
\bigl|\mathbb E[\Delta\mathrm{EASI}\mid \mathrm{Dual\_pulse}(K)]-\mathbb E[\Delta\mathrm{EASI}\mid \mathrm{Dual\_full}]\bigr|=\bigl|f(K)-F_{\mathrm{full}}\bigr|\le \varepsilon.\qedhere
\]
\end{proof}
}

% Auto-generated bibliography (sanitized for ASCII)
\begin{thebibliography}{99}
\bibitem{ArinkinGaitsgory2015-SingularSupport} D. Arinkin and D. Gaitsgory. Singular support of coherent sheaves, and the Geometric Langlands conjecture. Selecta Mathematica (N.S.) 21(1):1--199, 2015. \url{https://doi.org/10.1007/s00029-014-0167-5}
\bibitem{AGKRRV2020-RestrictedLocSys} D. Arinkin, D. Gaitsgory, D. Kazhdan, S. Raskin, N. Rozenblyum, and Y. Varshavsky. The stack of local systems with restricted variation and geometric Langlands theory with nilpotent singular support. arXiv preprint, 2020. \url{https://arxiv.org/abs/2010.01906}
\bibitem{BravermanGaitsgory2002-GeometricEisenstein} A. Braverman and D. Gaitsgory. Geometric Eisenstein series. Inventiones mathematicae 150:287--384, 2002. \url{https://doi.org/10.1007/s00222-002-0237-8}
\bibitem{BFGM2002-ICDrinfeld} A. Braverman, M. Finkelberg, D. Gaitsgory, and I. Mirkovi\'c. Intersection cohomology of Drinfeld's compactifications. Selecta Mathematica (N.S.) 8(3):381--418, 2002. \url{https://doi.org/10.1007/s00029-002-8111-5}
\bibitem{GaitsgoryRaskin2024-GLFunctorI} D. Gaitsgory and S. Raskin. Proof of the geometric Langlands conjecture I: construction of the functor. arXiv preprint, 2024. \url{https://arxiv.org/abs/2405.03599}
\end{thebibliography}


\end{document}
