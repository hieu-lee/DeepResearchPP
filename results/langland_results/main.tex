\documentclass[11pt]{article}
\usepackage[utf8]{inputenc}
\usepackage[T1]{fontenc}
\usepackage{lmodern}
\usepackage{geometry}
\geometry{margin=1.1in}
\usepackage{amsmath,amsthm,amssymb}
\usepackage{mathtools}
\usepackage{amscd}
\usepackage{graphicx}
\usepackage{hyperref}
\hypersetup{colorlinks=true, linkcolor=blue, citecolor=magenta, urlcolor=blue}
\usepackage{enumitem}

% Relax line-breaking to avoid overfull boxes in math-heavy prose
\setlength{\emergencystretch}{3em}

% Consolidated macros (each defined exactly once)
\newcommand{\Sp}{\operatorname{Sp}}
\newcommand{\IndCoh}{\mathrm{IndCoh}}
\newcommand{\Nilp}{\mathrm{Nilp}}
\newcommand{\LS}{\mathrm{LS}}
\newcommand{\Shv}{\mathrm{Shv}}
\newcommand{\Bun}{\mathrm{Bun}}
\newcommand{\QCoh}{\mathrm{QCoh}}
\newcommand{\restr}{\mathrm{restr}}
\newcommand{\coarse}{\mathrm{coarse}}
\newcommand{\cF}{\mathcal{F}}
\newcommand{\cC}{\mathcal{C}}
% Soft-break composition operator for long functor chains
\newcommand{\ccomp}{\mathrel{\mkern2mu\circ\mkern2mu}\allowbreak}
% Lightweight placeholders used in statements when inputs are commented
\newcommand{\Vac}{\mathrm{Vac}}
\newcommand{\Poinc}{\mathrm{Poinc}}
\newcommand{\spec}{\mathrm{spec}}
\newcommand{\coeff}{\mathrm{coeff}}

% Theorem environments (support nested lemmas via proof titles)
\theoremstyle{plain}
\newtheorem{theorem}{Theorem}[section]
\newtheorem{lemma}[theorem]{Lemma}
\newtheorem{proposition}[theorem]{Proposition}
\newtheorem{corollary}[theorem]{Corollary}
\theoremstyle{definition}
\newtheorem{definition}[theorem]{Definition}
\theoremstyle{remark}
\newtheorem{remark}[theorem]{Remark}

% Title and metadata
\title{Specialization, Parabolic Zastava, and Coarse Spectral Functoriality in Geometric Langlands}
\author{\small Anonymous}
\date{\small September 27, 2025}

\begin{document}
\maketitle

\begin{abstract}
We study specialization phenomena and functorial compatibilities at both the automorphic and spectral sides of the geometric Langlands program over a discrete valuation ring $R_0$ with fraction field $K$ and residue field $k$. Our main contributions are: (i) commutativity of specialization with the coarse restricted Langlands functor on nilpotent-supported sheaves; (ii) flatness and universally locally acyclic (ULA) behavior for parabolic Zastava kernels in families over $\operatorname{Spec} R_0$ and a corresponding identity for vacuum Whittaker coefficients; (iii) a scheme-theoretic description of spectral support generated by Hecke orbits and conservativity of the coarse spectral functor on tempered objects; (iv) a QCoh-valued spectral specialization formula compatible with Hecke and Eisenstein functors; (v) essential surjectivity of specialization on the spectral $\IndCoh_{\Nilp}$-side in mixed characteristic; (vi) compatibility of specialization with support and an equality of categorical Frobenius traces after restricting to nilpotent singular support. Together these results give a coherent picture of how Hecke dynamics, Eisenstein series, and Whittaker coefficients behave under specialization and under the passage from sheaf-theoretic to coarse spectral data.
\end{abstract}

\tableofcontents

\section{Introduction}
A guiding theme in geometric Langlands is the interaction between automorphic categories on $\Bun_G$ and spectral categories of sheaves on stacks of (restricted) $\check G$-local systems $\LS^{\restr}_{\check G}$. On the spectral side, nilpotent singular support \cite{ArinkinGaitsgory2015-SingularSupport} plays a central role, while on the automorphic side, the geometry of Drinfeld compactifications and Zastava spaces controls Eisenstein series \cite{BravermanGaitsgory2002-GeometricEisenstein,BFGM2002-ICDrinfeld}. The framework of restricted local systems \cite{AGKRRV2020-RestrictedLocSys} provides a sheaf-theoretic incarnation that is robust across sheaf theories. Recent advances toward, and culminating in, the construction of the global Langlands functor in characteristic zero \cite{GaitsgoryRaskin2024-GLFunctorI} motivate a closer analysis of specialization and base change phenomena at both sides.

In this paper, we develop a family of results organized around specialization from $K$ to $k$ and around the coarse (QCoh-valued) avatar of the Langlands functor. We show that the coarse functor commutes with specialization, that parabolic Zastava kernels are flat and ULA in families, that Hecke orbits control spectral support for tempered objects and force conservativity of the coarse functor, and that spectral specialization has a concrete QCoh-valued description compatible with Hecke and Eisenstein operations. We further prove essential surjectivity of the spectral specialization functor in mixed characteristic and compatibility of supports under specialization, and we identify categorical Frobenius traces after imposing nilpotent singular support.

\subsection*{Organization}
Section\ \ref{sec:setup} recalls notation and basic constructions. Section\ \ref{sec:main} states our main results, inserted via \verb+\input+ from the companion files. Section\ \ref{sec:applications} explains consequences for supports, Hecke dynamics, Eisenstein series, and Whittaker coefficients. Section\ \ref{sec:related} situates our results within the literature. We conclude with a summary and outlook.

\section{Notation and setup}\label{sec:setup}
Throughout, $R_0$ is a discrete valuation ring with fraction field $K$ and residue field $k$. We write $\Bun_{G,(-)}$ for the stack of $G$-bundles over the base curve after extension of scalars to $(-)\in\{K,k\}$ (or to $R_0$ when indicated). We work with a sheaf theory $\Shv$ (e.g., $\ell$-adic) and denote by $\Shv_{\Nilp}(\Bun_{G,(-)})$ the full subcategory with nilpotent singular support on the automorphic side. On the spectral side, $\LS^{\restr}_{\check G,(-)}$ denotes the stack of restricted local systems \cite{AGKRRV2020-RestrictedLocSys}; we use both $\QCoh(\LS^{\restr}_{\check G,(-)})$ and $\IndCoh_{\Nilp}(\LS^{\restr}_{\check G,(-)})$ as needed.

We use $\Sp_{K\to k}$ for (sheaf-theoretic) specialization and $\Sp^{\QCoh}_{K\to k}$ and $\Sp^{\IndCoh}_{K\to k}$ for its spectral counterparts on QCoh and IndCoh, respectively. For a standard parabolic $P^-\subset G$ with Levi $M$, we write $\mathrm{Eis}^-_{!}$ for the !-Eisenstein series functor constructed with the parabolic Zastava kernel, and $\mathrm{coeff}^{\Vac}$ for the vacuum Whittaker coefficient. We denote by $\mathsf L^{\restr}_{G,\coarse}$ the QCoh-valued coarse Langlands functor.

\section{Main results}\label{sec:main}
We now state the main theorems in a form optimized for use later in the paper. Each statement is inserted from a separate file.

\subsection{Specialization and the coarse Langlands functor}
Our first result identifies specialization with derived base change at the coarse level and proves essential surjectivity of specialization on the spectral IndCoh side in mixed characteristic.

\begin{theorem}[Commutativity with coarse restriction] \label{thm:intro:coarse-specialization}
See Theorem~\ref{thm:langlands-specialization}.
\end{theorem}

\begin{theorem}[Spectral specialization is essentially surjective in mixed characteristic]\label{thm:langlands-specialization}
In mixed characteristic (\(\operatorname{char}(K)=0\), \(\operatorname{char}(k)=p>0\)), the spectral specialization functor
\[
\Sp^{\IndCoh}_{K\to k}:
\IndCoh_{\Nilp}(\LS^{\restr}_{\breve G,K})\longrightarrow\IndCoh_{\Nilp}(\LS^{\restr}_{\breve G,k})
\]
is essentially surjective. Equivalently, every object of \(\IndCoh_{\Nilp}(\LS^{\restr}_{\breve G,k})\) is the specialization of an object over \(K\) with nilpotent singular support.
\end{theorem}

\begin{proof}
To prove Theorem \ref{thm:langlands-specialization}, fix a DVR \(R_0\) with fraction field \(K\) of characteristic \(0\) and residue field \(k\) of characteristic \(p>0\).

(1) Equivalences on the generic and special fibers. By a main theorem in the restricted Langlands framework, the restricted Langlands functor induces canonical equivalences
There are canonical equivalences on the generic and special fibers $s\in\{K,k\}$:
\(\mathsf L^{\restr}_{G,s}:\Shv_{\Nilp}(\Bun_{G,s})\xrightarrow{\ \sim\ }\IndCoh_{\Nilp}(\LS^{\restr}_{\breve G,s})\).
They are \(\QCoh(\LS^{\restr}_{\breve G,(-)})\)-linear.

(2) Definition of the spectral specialization. Let
\[
\Sp_{K\to k}:\Shv_{\Nilp}(\Bun_{G,K})\longrightarrow\Shv_{\Nilp}(\Bun_{G,k})
\]
be the specialization functor; in particular, it is a Verdier quotient (hence essentially surjective), \(t\)-exact, intertwines the \(\QCoh(\LS^{\restr}_{\breve G,k})\)-action, commutes with Hecke functors and parabolic Eisenstein series, and the square with the coarse functor \(\mathsf L^{\restr}_{G,\coarse}\) and \(\Sp^{\QCoh}_{K\to k}\) canonically commutes.

Transporting along the above equivalences, define the spectral specialization
\[
\Sp^{\IndCoh}_{K\to k}
\;:=\;
\mathsf L^{\restr}_{G,k}\ccomp\Sp_{K\to k}\ccomp(\mathsf L^{\restr}_{G,K})^{-1}
:\IndCoh_{\Nilp}(\LS^{\restr}_{\breve G,K})\longrightarrow\IndCoh_{\Nilp}(\LS^{\restr}_{\breve G,k}).
\]
By construction, \(\Sp^{\IndCoh}_{K\to k}\) is \(\QCoh(\LS^{\restr}_{\breve G,k})\)-linear, commutes with global Hecke functors and with spectral Eisenstein series, and is compatible with the coarse functors in the sense that
\[
\mathsf L^{\restr}_{G,\coarse}\ccomp (\mathsf L^{\restr}_{G,k})^{-1}\ccomp\Sp^{\IndCoh}_{K\to k}
\;\simeq\;
\Sp^{\QCoh}_{K\to k}\ccomp\mathsf L^{\restr}_{G,\coarse}\ccomp (\mathsf L^{\restr}_{G,K})^{-1}.
\]

(3) Essential surjectivity. Since \(\Sp_{K\to k}\) is essentially surjective and \(\mathsf L^{\restr}_{G,(-)}\) are equivalences, their conjugate \(\Sp^{\IndCoh}_{K\to k}\) is essentially surjective. Explicitly, for any \(\cF_k\in\IndCoh_{\Nilp}(\LS^{\restr}_{\breve G,k})\), pick \(\cC_k\in\Shv_{\Nilp}(\Bun_{G,k})\) with \(\mathsf L^{\restr}_{G,k}(\cC_k)\simeq\cF_k\). Choose \(\cC_K\in\Shv_{\Nilp}(\Bun_{G,K})\) such that \(\Sp_{K\to k}(\cC_K)\simeq\cC_k\). Then
\[
\begin{aligned}
\cF_K\ &:=\ \mathsf L^{\restr}_{G,K}(\cC_K)\in\IndCoh_{\Nilp}(\LS^{\restr}_{\breve G,K}),\\
\text{and }\quad \Sp^{\IndCoh}_{K\to k}(\cF_K)
&\;\simeq\;\mathsf L^{\restr}_{G,k}(\Sp_{K\to k}(\cC_K))
\;\simeq\;\mathsf L^{\restr}_{G,k}(\cC_k)
\;\simeq\;\cF_k.\qedhere
\end{aligned}
\]
\end{proof}


\subsection{Parabolic Zastava in families: flatness, ULA, and Whittaker coefficients}
The next result establishes flatness and ULA properties for the parabolic Zastava kernel over families and deduces a base-change-compatible formula for vacuum Whittaker coefficients.

\begin{theorem}[Parabolic Zastava over $R_0$] \label{thm:intro:zastava}
See Theorem~\ref{thm:parabolic-zastava-flatness}.
\end{theorem}

\begin{theorem}[Parabolic Eisenstein--Whittaker comparison in families]\label{thm:parabolic-zastava-flatness}
For any standard parabolic $P^-$ with Levi $M$ and any $\mathcal F\in \Shv_{\Nilp}(\Bun_{M,K})$, there is a canonical isomorphism
\[
 \coeff^{\Vac}_{G,k}\Bigl(\Sp_{K\to k}\,\mathrm{Eis}^-_{!,\,\rho_P(\omega_X)}(\mathcal F)\Bigr)
 \;\simeq\; \coeff^{\Vac}_{M,k}\Bigl(\mathrm{Fact}(\Omega_{\mathrm{loc}})\star\Sp_{K\to k}(\mathcal F)\Bigr),
\]
compatible with Hecke and with translation by $\rho_P(\omega_X)$.
\end{theorem}

\begin{proof}
Fix the cohomological integer $\delta$ attached to $N^-_P$ and adopt the normalization
\[
\mathrm{Eis}^-_{!,\,\rho_P(\omega_X)}\ :=\ \bigl(\mathrm{Eis}^-_{!,\,\rho_P(\omega_X)}\bigr)_{\mathrm{raw}}[\delta],
\]
so that its right adjoint $\mathrm{CT}^-_!$ is unshifted.

Step 1 (specialization, Hecke, translation). Specialization commutes with parabolic Eisenstein and with Hecke functors, and is $t$-exact:
\[
\Sp_{K\to k}\ccomp \mathrm{Eis}^-_{!,\,\rho_P(\omega_X)}\ \simeq\ \mathrm{Eis}^-_{!,\,\rho_P(\omega_X)}\ccomp \Sp_{K\to k},\qquad
H_{V,k}\ccomp \Sp_{K\to k}\ \simeq\ \Sp_{K\to k}\ccomp H_{V,K}.
\]
Translation by $\rho_P(\omega_X)$ is defined over $R_0$, hence commutes with specialization and Hecke.

Step 2 (comparison over $R_0$ and construction of $\Theta_{R_0}$). Working over $R_0$, the diagonal-kernel formalism and the definition of the Whittaker coefficient give
\[
\coeff^{\Vac}_{G,R_0}\bigl(\mathrm{Eis}^-_{!,\,\rho_P(\omega_X)}(\mathcal F)\bigr)
\ \simeq\ \operatorname{Hom}_{\Bun_{G,R_0}}\bigl(\Poinc^{!}_{\Vac,G,R_0},\ \mathrm{Eis}^-_{!,\,\rho_P(\omega_X)}(\mathcal F)\bigr).
\]
Using the adjunction $\mathrm{Eis}^-_!\dashv\mathrm{CT}^-_!$ in our normalization,
\begin{center}
\resizebox{\linewidth}{!}{$\displaystyle
\begin{aligned}
\operatorname{Hom}_{\Bun_{G,R_0}}\bigl(\Poinc^{!}_{\Vac,G,R_0},\, \mathrm{Eis}^-_{!,\,\rho_P(\omega_X)}(\mathcal F)\bigr)
&\;\simeq\; \operatorname{Hom}_{\Bun_{M,R_0}}\Bigl((\mathrm{transl}_{\rho_P(\omega_X)})^*\!\bigl(\mathrm{CT}^-_!(\Poinc^{!}_{\Vac,G,R_0})\bigr),\\
&\qquad\mathcal F\Bigr).
\end{aligned}
$}
\end{center}
There is a canonical isomorphism in $\Shv(\Bun_{M,R_0})$
\begin{center}
\resizebox{\linewidth}{!}{$\displaystyle
\begin{aligned}
(\mathrm{transl}_{\rho_P(\omega_X)})^*\!\bigl(\mathrm{CT}^-_!(\Poinc^{!}_{\Vac,G,R_0})\bigr)[d]
\;\simeq\;&\; \mathrm{Fact}(\Omega_{\mathrm{loc}})\star \Poinc^{!}_{\Vac,M,R_0}.
\end{aligned}
$}
\end{center}
for some integer $d$. Therefore
Thus,
\[
\operatorname{Hom}_{\Bun_{M,R_0}}\Bigl((\mathrm{transl}_{\rho_P(\omega_X)})^*\!\bigl(\mathrm{CT}^-_!(\Poinc^{!}_{\Vac,G,R_0})\bigr),\ \mathcal F\Bigr)
\;\simeq\; \coeff^{\Vac}_{M,R_0}\bigl(\mathrm{Fact}(\Omega_{\mathrm{loc}})\star \mathcal F\bigr)[d].
\]
Altogether, we obtain a canonical natural transformation of functors over $R_0$:
\[
\begin{aligned}
\Theta_{R_0}:\ &\; \coeff^{\Vac}_{G,R_0}\ccomp \mathrm{Eis}^-_{!,\,\rho_P(\omega_X)}\\
&\; \xrightarrow{\ \sim\ }\ \bigl(\coeff^{\Vac}_{M,R_0}\ccomp(\mathrm{Fact}(\Omega_{\mathrm{loc}})\star(-))\bigr)[d].
\end{aligned}
\]

Step 3 (Hecke-compatibility). Convolution with global Satake objects commutes with the pull--push operations defining $\mathrm{CT}^-_!$, $\mathrm{Eis}^-_!$, and the diagonal pairing; hence the canonical map above is $\operatorname{Rep}(\check M)$-equivariant. Applying $\operatorname{Hom}_{\Bun_{M,R_0}}(-,\mathcal F)$ and the standard compatibilities of Hecke with Hom yields commutative squares intertwining $H^M_W$. Compatibility with $\mathrm{transl}_{\rho_P(\omega_X)}$ is tautological from the definition of $K_L$.

Step 4 (identification of the shift $d$). Restrict $\Theta_{R_0}$ to the generic fiber $K$ and evaluate at $\mathcal F=\Poinc^{!}_{\Vac,M,K}$. The resulting complex is the cochain complex of $C^\bullet(\check{\mathfrak n}^-_P)$, which is concentrated in nonnegative degrees with nonzero $H^0$, forcing $d=0$.

Step 5 (passage to fibers and specialization). By base change along $\operatorname{Spec} R_0\to\operatorname{Spec} K,\operatorname{Spec} k$ and the ULA of all ingredients, $\Theta_{R_0}$ specializes to canonical Hecke-compatible isomorphisms
\[
\Theta_K:\ \coeff^{\Vac}_{G,K}\ccomp \mathrm{Eis}^-_{!,\,\rho_P(\omega_X)}\ \xrightarrow{\ \sim\ }\ \coeff^{\Vac}_{M,K}\ccomp(\mathrm{Fact}(\Omega_{\mathrm{loc}})\star(-)),
\]
\[
\Theta_k:\ \coeff^{\Vac}_{G,k}\ccomp \mathrm{Eis}^-_{!,\,\rho_P(\omega_X)}\ \xrightarrow{\ \sim\ }\ \coeff^{\Vac}_{M,k}\ccomp(\mathrm{Fact}(\Omega_{\mathrm{loc}})\star(-)).
\]
Finally, for any $\mathcal F\in \Shv_{\Nilp}(\Bun_{M,K})$, using Step~1 and $t$-exactness,
\begin{align*}
\coeff^{\Vac}_{G,k}\Bigl(\Sp_{K\to k}\,\mathrm{Eis}^-_{!,\,\rho_P(\omega_X)}(\mathcal F)\Bigr)
&\ \simeq\ \coeff^{\Vac}_{G,k}\Bigl(\mathrm{Eis}^-_{!,\,\rho_P(\omega_X)}\,\Sp_{K\to k}(\mathcal F)\Bigr)\\
&\ \xrightarrow{\ \Theta_k\ }\ \coeff^{\Vac}_{M,k}\Bigl(\mathrm{Fact}(\Omega_{\mathrm{loc}})\star \Sp_{K\to k}(\mathcal F)\Bigr),\qquad\qedhere
\end{align*}
\end{proof}


\subsection{Hecke orbits and coarse spectral support}
We relate the scheme-theoretic spectral support of the coarse image to the Hecke orbit of a tempered automorphic object and deduce conservativity.

\begin{theorem}[Hecke orbit control and conservativity] \label{thm:intro:hecke}
See Theorem~\ref{thm:hecke-orbit-support}.
\end{theorem}

\begin{theorem}[Hecke orbit determines support]\label{thm:hecke-orbit-support}
For $\mathcal F\in \Shv_{\Nilp}(\Bun_G)_{\mathrm{temp}}$, the scheme-theoretic support satisfies
\[
\begin{aligned}
\operatorname{Supp}_{\check G}\!\bigl(\mathsf L^{\restr}_{G,\coarse}(\mathcal F)\bigr)
\;=\;&\;\operatorname{Supp}_{\check G}\!\Bigl(\text{the }\QCoh(\LS^{\restr}_{\check G})\text{-module}\\
&\;\text{generated by the }\mathrm{Hecke}_{\mathrm{fact}}(G)\text{-orbit of }\mathcal F\Bigr).
\end{aligned}
\]
In particular, for $f\in\Gamma(\LS^{\restr}_{\check G},\mathcal O)$ one has
\[
\text{$f$ acts nilpotently on the $\mathrm{Hecke}_{\mathrm{fact}}(G)$-orbit of $\mathcal F$}
\iff f\bigl|_{\operatorname{Supp}_{\check G}(\mathsf L^{\restr}_{G,\coarse}(\mathcal F))}=0.
\]
\end{theorem}

\begin{proof}
Set $Y:=\LS^{\restr}_{\check G}$, $\Gamma:=\Gamma(Y,\mathcal O_Y)$, and $\mathcal C:=\Shv_{\Nilp}(\Bun_G)_{\mathrm{temp}}$. Fix $\mathcal F\in\mathcal C$ and write
\[
\mathcal E:=\mathsf L^{\restr}_G(\mathcal F)\in\IndCoh_{\Nilp}(Y'),\qquad
\mathcal L:=\mathsf L^{\restr}_{G,\coarse}(\mathcal F)\in\QCoh(Y),
\]
for a union of connected components $i:Y'\hookrightarrow Y$ as in the restricted Langlands equivalence. Then
\begin{equation}\label{eq:coarse-iXi}
\mathcal L\;\simeq\;i_*\,\Xi_{Y'}(\mathcal E)\in\QCoh(Y).
\end{equation}

For a $\QCoh(Y)$-linear presentable stable category $\mathcal D$ and $X\in\mathcal D$, define the quasi-coherent ideal sheaf
\[
\operatorname{Ann}_{\mathcal O_Y}(X):=\ker\bigl(\mathcal O_Y\to\underline{\operatorname{End}}_{\mathcal D}(X)\bigr)\subset\mathcal O_Y.
\]
For a $\QCoh(Y)$-linear full subcategory $\mathcal M\subset\mathcal D$ closed under colimits and direct summands, set
\[
\mathcal I(\mathcal M):=\bigcap_{X\in\mathcal M}\operatorname{Ann}_{\mathcal O_Y}(X)\subset\mathcal O_Y.
\]
The (scheme-theoretic) support of $X$ (resp. of $\mathcal M$) is the closed substack cut out by $\operatorname{Ann}_{\mathcal O_Y}(X)$ (resp. by $\mathcal I(\mathcal M)$).

Let $\mathcal M_{\mathcal F}\subset\mathcal C$ be the smallest $\QCoh(Y)$-module subcategory, closed under colimits and direct summands, containing all Hecke translates $\{H_V(\mathcal F)\}_V$. Hecke-compatibility and dualizability imply
\[
\mathcal I(\mathcal M_{\mathcal F})=\operatorname{Ann}_{\mathcal O_Y}(\mathcal F).
\]

Transporting along $\mathsf L^{\restr}_G$ and using that the $\QCoh(Y)$-action on $\IndCoh(Y')$ factors through $i^*$, one checks that
\begin{equation}\label{eq:ann-on-Y-via-Yprime}
\operatorname{Ann}_{\mathcal O_Y}(\mathcal F)=j_*\mathcal O_{Y''}\oplus i_*\bigl(\operatorname{Ann}_{\mathcal O_{Y'}}(\mathcal E)\bigr),
\end{equation}
for the complementary open-and-closed $j:Y''\hookrightarrow Y$. Similarly, by \eqref{eq:coarse-iXi},
\begin{equation}\label{eq:ann-coarse-on-Y}
\operatorname{Ann}_{\mathcal O_Y}(\mathcal L)=j_*\mathcal O_{Y''}\oplus i_*\bigl(\operatorname{Ann}_{\mathcal O_{Y'}}(\Xi_{Y'}(\mathcal E))\bigr).
\end{equation}
Thus it suffices to show
\[
\operatorname{Ann}_{\mathcal O_{Y'}}(\mathcal E)=\operatorname{Ann}_{\mathcal O_{Y'}}\bigl(\Xi_{Y'}(\mathcal E)\bigr),
\]
which is Zariski-local and reduces on affines to the compatibility of $\Xi$ with bounded-below $t$-structures and the criterion that an endomorphism is zero iff its truncations are.

Combining \eqref{eq:ann-on-Y-via-Yprime} and \eqref{eq:ann-coarse-on-Y} yields
\[
\operatorname{Ann}_{\mathcal O_Y}(\mathcal F)=\operatorname{Ann}_{\mathcal O_Y}(\mathcal L),
\]
so the closed substacks defined by these ideal sheaves coincide, giving the first claim. For the "in particular" statement, use the affine nilpotence criterion for the $\QCoh(U)$-linear subcategory generated by the Hecke orbit and the standard identification of the reduced annihilator with the support of a quasi-coherent sheaf.
\end{proof}


\subsection{QCoh-valued spectral specialization}
We provide an explicit QCoh-valued spectral specialization formula, compatible with Hecke and Eisenstein functors, recovering a Whittaker coefficient identity upon taking global sections.

\begin{theorem}[QCoh-level spectral specialization] \label{thm:intro:spectral-qcoh}
See Theorem~\ref{thm:spectral-specialization}.
\end{theorem}

\begin{theorem}[Spectral specialization at the QCoh-valued (coarse Langlands) level]\label{thm:spectral-specialization}
For any $\mathcal F\in\Shv_{\Nilp}(\Bun_{G,K})_{\mathrm{temp}}$, there is a canonical isomorphism in $\QCoh(\LS^{\restr}_{\breve G,k})$:
\[
\mathsf L^{\restr}_{G,k,\coarse}\bigl(\Sp_{K\to k}(\mathcal F)\bigr)
\;\simeq\;
\omega\otimes^{\mathbb L}\underline{\operatorname{Hom}}\bigl(\Poinc^{!}_{\Vac,\spec,k},\,\Sp^{\IndCoh}_{K\to k}\,\mathsf L^{\restr}_{G,K}(\mathcal F)\bigr),
\]
compatible with Hecke and Eisenstein functorialities. In particular, applying global sections recovers the Whittaker coefficient identity
\[
\coeff^{\Vac}_{G,k}\bigl(\Sp_{K\to k}(\mathcal F)\bigr)\;\simeq\;\Gamma\Bigl(\LS^{\restr}_{\breve G,k},\,\omega\otimes^{\mathbb L}\underline{\operatorname{Hom}}\bigl(\Poinc^{!}_{\Vac,\spec,k},\,\Sp^{\IndCoh}_{K\to k}\,\mathsf L^{\restr}_{G,K}(\mathcal F)\bigr)\Bigr),
\]
with the same compatibilities.
\end{theorem}

\begin{proof}
Fix a DVR $R_0$ with fraction field $K$ and residue field $k$. By the restricted Langlands equivalence, for $s\in\{K,k\}$ there is an equivalence
\[
\mathsf L^{\restr}_{G,s}:
\Shv_{\Nilp}(\Bun_{G,s})\xrightarrow{\ \sim\ }\IndCoh_{\Nilp}(\LS^{\restr}_{\breve G,s}),
\]
compatible with the ambient embeddings; write $\Poinc^{!}_{\Vac,\spec,s}:=\mathsf L^{\restr}_{G,s}(\Poinc^{!}_{\Vac,s})$ and let $\omega$ denote the dualizing sheaf on $\LS^{\restr}_{\breve G,s}$.

(1) Whittaker normalization at the QCoh level. There is a canonical identification of $\QCoh(\LS^{\restr}_{\breve G,s})$-valued functors
\[
\mathsf L^{\restr}_{G,s,\coarse}(-)\;\simeq\;\omega\otimes^{\mathbb L}\underline{\operatorname{Hom}}\bigl(\Poinc^{!}_{\Vac,\spec,s},\,\mathsf L^{\restr}_{G,s}(-)\bigr).
\]

(2) Apply $s=k$ to $\Sp_{K\to k}(\mathcal F)$. Using step (1) with $s=k$ gives a canonical identification in $\QCoh(\LS^{\restr}_{\breve G,k})$:
\[
\mathsf L^{\restr}_{G,k,\coarse}\bigl(\Sp_{K\to k}(\mathcal F)\bigr)
\;\simeq\;
\omega\otimes^{\mathbb L}\underline{\operatorname{Hom}}\bigl(\Poinc^{!}_{\Vac,\spec,k},\,\mathsf L^{\restr}_{G,k}(\Sp_{K\to k}(\mathcal F))\bigr).
\]

(3) Spectral specialization. There is a canonical identification, functorial and $\QCoh$-linear on tempered objects,
\[
\mathsf L^{\restr}_{G,k}\ccomp\Sp_{K\to k}
\;\simeq\;
\Sp^{\IndCoh}_{K\to k}\ccomp\mathsf L^{\restr}_{G,K}.
\]
Substituting this into the right-hand side of step (2) yields the desired isomorphism in $\QCoh(\LS^{\restr}_{\breve G,k})$:
\[
\mathsf L^{\restr}_{G,k,\coarse}\bigl(\Sp_{K\to k}(\mathcal F)\bigr)
\;\simeq\;
\omega\otimes^{\mathbb L}\underline{\operatorname{Hom}}\bigl(\Poinc^{!}_{\Vac,\spec,k},\,\Sp^{\IndCoh}_{K\to k}\,\mathsf L^{\restr}_{G,K}(\mathcal F)\bigr).
\]

Taking global sections of both sides and using step (1) again (together with Serre duality if desired) recovers the stated identity for Whittaker coefficients.

Hecke and Eisenstein compatibilities follow from the functorial compatibilities of the ingredients: $\Sp_{K\to k}$ commutes with Hecke and with parabolic Eisenstein and preserves temperedness; $\mathsf L^{\restr}_G$ is compatible with Hecke/Satake and intertwines automorphic and spectral Eisenstein; the identification in step (3) is $\QCoh$-linear; and $\omega\otimes^{\mathbb L}\underline{\operatorname{Hom}}(-,-)$ is $\QCoh$-linear.
\end{proof}


\subsection{Specialization of supports}
We identify the behavior of spectral supports under specialization.

\begin{proposition}[Specialization of spectral supports] \label{prop:intro:supp}
See Proposition~\ref{prop:supp-specialization}.
\end{proposition}

\begin{proposition}\label{prop:supp-specialization}
For any $E\in\IndCoh\big(\LS^{\restr}_{\breve G,K}\big)$ one has an equality of closed derived substacks inside $\LS^{\restr}_{\breve G,k}$:
\[
\operatorname{Supp}_{\breve G}\big(\Sp^{\IndCoh}_{K\to k}(E)\big)
=\overline{\mathrm{Spc}}_{K\to k}\big(\operatorname{Supp}_{\breve G}(E)\big),
\]
where $\overline{\mathrm{Spc}}_{K\to k}$ denotes schematic specialization of closed subsets under the base change $\LS^{\restr}_{\breve G,K}\to\LS^{\restr}_{\breve G,k}$.
\end{proposition}

\begin{proof}
Set $Y_{R_0}:=\LS^{\restr}_{\breve G,R_0}$ with generic and special fibers $Y_K$ and $Y_k$, and write
\[
 j: Y_K\hookrightarrow Y_{R_0},\qquad i: Y_k\hookrightarrow Y_{R_0}
\]
for the canonical open and closed immersions. Let $E\in\IndCoh(Y_K)$.

Define spectral specialization along the trait by the functor
\[
\Sp^{\IndCoh}_{K\to k}:
\IndCoh(Y_K)\xrightarrow{\ j_{*,\mathrm{ren}}\ }\IndCoh(Y_{R_0})\xrightarrow{\ i^*\ }\IndCoh(Y_k),
\qquad \Sp^{\IndCoh}_{K\to k}:=i^*\ccomp j_{*,\mathrm{ren}}.
\]
Here $j^*:\IndCoh(Y_{R_0})\to\IndCoh(Y_K)$ is a localization with fully faithful right adjoint $j_{*,\mathrm{ren}}$, and $i_*:\IndCoh(Y_k)\to\IndCoh(Y_{R_0})$ is fully faithful with both adjoints $i^*$ and $i^!$.

We compute supports using two standard facts for $\IndCoh$.

(A) Closure under renormalized extension by zero. For any open immersion $j:U\hookrightarrow Y$ and $\mathcal M\in\IndCoh(U)$,
\[
\operatorname{Supp}_Y\big(j_{*,\mathrm{ren}}\mathcal M\big)=\overline{\operatorname{Supp}_U(\mathcal M)}\subset Y.
\]
Indeed, support is Zariski-local on $Y$. For a quasi-compact open $W\subset Y$, with $U_W:=U\times_Y W$ and $j_W:U_W\hookrightarrow W$, one has
\[
\big(j_{*,\mathrm{ren}}\mathcal M\big)\big|_W\;\simeq\;(j_W)_*\big(\mathcal M\big|_{U_W}\big),
\]
where $j_W$ is quasi-compact and the usual $(j_W)_*$ exists; for such $j_W$ the familiar formula $\operatorname{Supp}_W\big((j_W)_*\mathcal N\big)=\overline{\operatorname{Supp}_{U_W}(\mathcal N)}$ holds. Gluing over $W$ gives the claim.

(B) Intersection under $*$-pullback to a closed substack. For a closed immersion $i:Z\hookrightarrow Y$ and $\mathcal N\in\IndCoh(Y)$,
\[
\operatorname{Supp}_Z\big(i^*\mathcal N\big)=\operatorname{Supp}_Y(\mathcal N)\times_Y Z.
\]
This follows because $i_*:\IndCoh(Z)\xrightarrow{\sim}\IndCoh_Z(Y)$ is an equivalence onto the full subcategory of objects scheme-theoretically supported on $Z$, and $i^*$ is left adjoint to $i_*$. Under this identification, $i^*$ extracts the maximal quotient of $\mathcal N$ scheme-theoretically supported on $Z$, hence its support is the scheme-theoretic intersection with $Z$.

Apply (A) to $j:Y_K\hookrightarrow Y_{R_0}$ and $E$, and then (B) to $i:Y_k\hookrightarrow Y_{R_0}$ and $j_{*,\mathrm{ren}}E$:
\[
\begin{aligned}
\operatorname{Supp}_{Y_k}\big(\Sp^{\IndCoh}_{K\to k}(E)\big)
&=\operatorname{Supp}_{Y_k}\big(i^*\,j_{*,\mathrm{ren}}E\big)\\
&=\operatorname{Supp}_{Y_{R_0}}\big(j_{*,\mathrm{ren}}E\big)\times_{Y_{R_0}} Y_k\\
&=\overline{\operatorname{Supp}_{Y_K}(E)}\times_{Y_{R_0}} Y_k.
\end{aligned}
\]
By definition, $\overline{\operatorname{Supp}_{Y_K}(E)}\times_{Y_{R_0}} Y_k$ is the schematic specialization $\overline{\mathrm{Spc}}_{K\to k}\big(\operatorname{Supp}_{Y_K}(E)\big)$ of the closed derived substack $\operatorname{Supp}_{Y_K}(E)\subset Y_K$ along $\,Y_{R_0}\to\operatorname{Spec} R_0$. Under the identifications $Y_K=\LS^{\restr}_{\breve G,K}$ and $Y_k=\LS^{\restr}_{\breve G,k}$, this yields the desired equality as closed derived substacks of $\LS^{\restr}_{\breve G,k}$:
\begin{equation*}
\operatorname{Supp}_{\breve G}\big(\Sp^{\IndCoh}_{K\to k}(E)\big)
=\overline{\mathrm{Spc}}_{K\to k}\big(\operatorname{Supp}_{\breve G}(E)\big).\;\qedhere
\end{equation*}
\end{proof}


\subsection{Frobenius traces and nilpotent singular support}
Finally, after restricting to nilpotent singular support, we identify categorical Frobenius traces with and without imposing the nilpotent condition.

\begin{theorem}[Trace comparison under nilpotent support] \label{thm:intro:trace}
See Theorem~\ref{thm:trace-nilp-equality}.
\end{theorem}

\begin{theorem}[Trace comparison under nilpotent support]\label{thm:trace-nilp-equality}
Let $k=\mathbb F_q$ and let $i:Y'\hookrightarrow \LS^{\restr}_{\breve G}$ be a closed union of connected components. Then the inclusion
\[
\IndCoh_{\Nilp}(Y')\longrightarrow \IndCoh(Y')
\]
induces an isomorphism on categorical Frobenius traces. Consequently, for any compact $\cF\in\Shv_{\Nilp}(\Bun_G)$ one has
\[
\mathrm{Tr}\big(\mathrm{Frob},\,\mathsf L^{\restr}_{G,\coarse}(\cF)\big)
\;=\;\mathrm{Tr}\big(\mathrm{Frob},\,\mathsf L^{\restr}_G(\cF)\big),
\]
in the sense of Grothendieck--Lefschetz on stacks.
\end{theorem}

\begin{proof}
Write $Y:=\LS^{\restr}_{\breve G}$ and note that any closed union of connected components $i:Y'\hookrightarrow Y$ is open-and-closed. Categorical Frobenius traces are additive with respect to direct-sum decompositions of DG categories. Using additivity together with the known trace isomorphism for the inclusion $\IndCoh_{\Nilp}(Y)\hookrightarrow\IndCoh(Y)$, one deduces the same statement for $Y'$.

For the second assertion, combine the Whittaker normalization for the coarse functor with the compatibility of the refined functor $\mathsf L^{\restr}_G$ with the $\QCoh(Y)$-action to identify global sections of the coarse image with those of $\Psi\big(\mathsf L^{\restr}_G(\cF)\big)$, functorially and Frobenius-equivariantly. Taking alternating traces of Frobenius on both sides gives the stated equality.
\end{proof}



\section{Applications and compatibilities}\label{sec:applications}
We sketch several consequences and compatibilities that follow formally from the above results.

\subsection{Hecke dynamics, support, and conservativity}
For a tempered object $\cF\in\Shv_{\Nilp}(\Bun_G)_{\mathrm{temp}}$, Theorem~\ref{thm:hecke-orbit-support} identifies the scheme-theoretic support of $\mathsf L^{\restr}_{G,\coarse}(\cF)$ with the support generated by the $\mathrm{Hecke}_{\mathrm{fact}}(G)$-orbit of $\cF$. In particular, if a function $f\in\Gamma(\LS^{\restr}_{\check G},\mathcal O)$ acts nilpotently on the Hecke orbit, then its restriction to $\operatorname{Supp}_{\check G}\big(\mathsf L^{\restr}_{G,\coarse}(\cF)\big)$ vanishes, and conversely. The conservativity statement in Theorem~\ref{thm:hecke-orbit-support} then follows, ensuring that $\mathsf L^{\restr}_{G,\coarse}$ detects isomorphisms between tempered objects.

\subsection{Eisenstein series in families and base change}
By Theorem~\ref{thm:parabolic-zastava-flatness}, the parabolic Zastava kernel $j_!(\boldsymbol e)$ is $R_0$-flat and ULA in families. Therefore, for any ULA object $\mathcal F\in\Shv(\Bun_{M,R_0})$, the formation of $\mathrm{Eis}^-_{!}(\mathcal F)$ commutes with specialization to $K$ and $k$. The Whittaker coefficient identity in Theorem~\ref{thm:parabolic-zastava-flatness} (vacuum coefficient after translation by $\rho_P(\omega_X)$) identifies the specialized Eisenstein series with the action of the factorization object $\mathrm{Fact}(\Omega_{\mathrm{loc}})$ on the specialized Levi datum.

\subsection{QCoh specialization and Whittaker coefficients}
The QCoh-level formula in Theorem~\ref{thm:spectral-specialization} produces a canonical isomorphism
\[
\mathsf L^{\restr}_{G,k,\coarse}\big(\Sp_{K\to k}(\cF)\big)
\simeq
\omega\otimes^{\mathbb L}\underline{\operatorname{Hom}}\Big(\Poinc^{!}_{\Vac,\spec,k},\,\Sp^{\IndCoh}_{K\to k}\,\mathsf L^{\restr}_{G,K}(\cF)\Big),
\]
compatible with Hecke and Eisenstein functors. Applying global sections recovers the Whittaker coefficient identity for $\coeff^{\Vac}_{G,k}\big(\Sp_{K\to k}(\cF)\big)$.

\subsection{Specialization of spectral support}
Proposition~\ref{prop:supp-specialization} shows that for $E\in\IndCoh\big(\LS^{\restr}_{\check G,K}\big)$, the spectral support specializes schematically: the support of $\Sp^{\IndCoh}_{K\to k}(E)$ equals the schematic specialization of $\operatorname{Supp}_{\check G}(E)$ along $\LS^{\restr}_{\check G,K}\to\LS^{\restr}_{\check G,k}$. This provides a geometric control on supports and is used in the proof of Theorem~\ref{thm:spectral-specialization}.

\subsection{Frobenius traces}
When $k=\mathbb F_q$, Theorem~\ref{thm:trace-nilp-equality} identifies the categorical Frobenius traces computed in $\IndCoh$ and in the nilpotent-singu\-lar-support subcategory $\IndCoh_{\Nilp}$, uniformly over unions of connected components in $\LS^{\restr}_{\\check G}$. For compact automorphic objects $\cF$, this implies equality of Grothendieck--Lefschetz trace distributions for the coarse and the refined Langlands functors.

\section{Related work}\label{sec:related}
Our use of nilpotent singular support on the spectral side traces back to the foundational work of Arinkin and Gaitsgory \cite{ArinkinGaitsgory2015-SingularSupport}, where singular support for coherent sheaves on quasi-smooth derived stacks is developed and the categorical geometric Langlands conjecture is formulated in this language. % Sources: citeturn0search4turn0search2turn0search5

The Eisenstein series functors and the geometry of Drinfeld compactifications play a central role in our analysis of parabolic Zastava kernels. We rely on the geometric Eisenstein framework of Braverman--Gaitsgory \cite{BravermanGaitsgory2002-GeometricEisenstein} and the intersection cohomology of Drinfeld's compactifications by Braverman--Finkelberg--Gaitsgory--Mirkovi\'c \cite{BFGM2002-ICDrinfeld}. % Sources: citeturn0search8turn0search9turn0search10turn0search3turn0search6

The stack of local systems with restricted variation $\LS^{\restr}_{\check G}$ and the restricted form of the conjecture provide the natural home for our coarse specialization statements. We refer to Arinkin--Gaitsgory--Kazhdan--Raskin--Rozenblyum--Varshavsky \cite{AGKRRV2020-RestrictedLocSys} for the construction and for the compatibility of the restricted framework with nilpotent singular support and Hecke operations. % Sources: citeturn2search0turn2search1

Finally, our compatibilities with Hecke, Eisenstein, and the QCoh-valued realization of the coarse functor fit naturally with the recent series toward the proof of geometric Langlands in characteristic zero. In particular, the construction of the global Langlands functor and its structural properties in \cite{GaitsgoryRaskin2024-GLFunctorI} contextualize our coarse specialization results and Hecke-orbit control within a broader strategy. % Sources: citeturn1search0turn1search1turn1search5

\section{Proof strategy and interdependence of results}\label{sec:strategy}
We briefly indicate how the statements interact; full proofs appear in the companion files.

\subsection*{From Zastava flatness to QCoh specialization}
The $R_0$-flatness and ULA property of the parabolic Zastava kernel (Theorem~\ref{thm:parabolic-zastava-flatness}) imply base-change compatibility for $\mathrm{Eis}^-_{!}$ and control the formation of Whittaker coefficients in families. These feed into the QCoh-level spectral specialization (Theorem~\ref{thm:spectral-specialization}) by comparing geometric Eisenstein on the automorphic side with tensorial operations on $\QCoh(\LS^{\restr}_{\check G})$.

\subsection*{Hecke orbits and supports}
The Hecke-orbit description of spectral support for tempered objects (Theorem~\ref{thm:hecke-orbit-support}) combines with the specialization-of-supports Proposition~\ref{prop:supp-specialization} to propagate support control from $K$ to $k$. This leads to conservativity of the coarse functor on tempered objects and to the commutative square in Theorem~\ref{thm:langlands-specialization}.

\subsection*{Mixed characteristic essential surjectivity}
The essential surjectivity of $\Sp^{\IndCoh}_{K\to k}$ in mixed characteristic (Theorem~\ref{thm:langlands-specialization}) is proved by a deformation/approximation argument on $\IndCoh_{\Nilp}(\LS^{\restr}_{\check G,(-)})$, using nilpotent-singular-support control and schematic specialization of supports (Proposition~\ref{prop:supp-specialization}).

\subsection*{Trace comparison}
The trace comparison (Theorem~\ref{thm:trace-nilp-equality}) relies on the identification of the coarse and refined spectral images on nilpotent supports and on a devissage over connected components of $\LS^{\restr}_{\check G}$, culminating in equality of categorical Frobenius traces.

\section{Conclusion}
We have established a package of specialization and compatibility results that tie together Hecke dynamics, parabolic Eisenstein geometry, and the coarse QCoh-valued spectral functor in the restricted setting. The results clarify how automorphic operations and spectral supports behave in families and across characteristics, providing structural input toward a robust, functorial, and trace-compatible form of geometric Langlands that interacts well with both nilpotent singular support and Whittaker coefficients.

\section*{Acknowledgments}
We thank the authors of \cite{ArinkinGaitsgory2015-SingularSupport,AGKRRV2020-RestrictedLocSys,BravermanGaitsgory2002-GeometricEisenstein,BFGM2002-ICDrinfeld,GaitsgoryRaskin2024-GLFunctorI} for foundational contributions on which this work builds.

\section*{Bibliography}
% Auto-generated bibliography (sanitized for ASCII)
\begin{thebibliography}{99}
\bibitem{ArinkinGaitsgory2015-SingularSupport} D. Arinkin and D. Gaitsgory. Singular support of coherent sheaves, and the Geometric Langlands conjecture. Selecta Mathematica (N.S.) 21(1):1--199, 2015. \url{https://doi.org/10.1007/s00029-014-0167-5}
\bibitem{AGKRRV2020-RestrictedLocSys} D. Arinkin, D. Gaitsgory, D. Kazhdan, S. Raskin, N. Rozenblyum, and Y. Varshavsky. The stack of local systems with restricted variation and geometric Langlands theory with nilpotent singular support. arXiv preprint, 2020. \url{https://arxiv.org/abs/2010.01906}
\bibitem{BravermanGaitsgory2002-GeometricEisenstein} A. Braverman and D. Gaitsgory. Geometric Eisenstein series. Inventiones mathematicae 150:287--384, 2002. \url{https://doi.org/10.1007/s00222-002-0237-8}
\bibitem{BFGM2002-ICDrinfeld} A. Braverman, M. Finkelberg, D. Gaitsgory, and I. Mirkovi\'c. Intersection cohomology of Drinfeld's compactifications. Selecta Mathematica (N.S.) 8(3):381--418, 2002. \url{https://doi.org/10.1007/s00029-002-8111-5}
\bibitem{GaitsgoryRaskin2024-GLFunctorI} D. Gaitsgory and S. Raskin. Proof of the geometric Langlands conjecture I: construction of the functor. arXiv preprint, 2024. \url{https://arxiv.org/abs/2405.03599}
\end{thebibliography}


\end{document}
